\documentclass[12pt]{article}
\usepackage[paper=letterpaper,margin=1.5cm]{geometry}
\usepackage{amsmath}
\usepackage{amssymb}
\usepackage{amsfonts}
\usepackage{mathtools}
%\usepackage[utf8]{inputenc}
%\usepackage{newtxtext, newtxmath}
\usepackage{lmodern}     % set math font to Latin modern math
\usepackage[T1]{fontenc}
\renewcommand\rmdefault{ptm}
%\usepackage{enumitem}
\usepackage[shortlabels]{enumitem}
\usepackage{titling}
\usepackage{graphicx}
\usepackage[colorlinks=true]{hyperref}
\usepackage{setspace}
\usepackage{subfigure} 
\usepackage{braket}
\usepackage{color}
\usepackage{tabularx}
\usepackage[table]{xcolor}
\usepackage{listings}
\usepackage{mathrsfs}
\usepackage{stackengine}
\usepackage{physics}
\usepackage{afterpage}
\usepackage{pdfpages}
\usepackage[export]{adjustbox}
\usepackage{biblatex}

\setstackEOL{\\}

\definecolor{dkgreen}{rgb}{0,0.6,0}
\definecolor{gray}{rgb}{0.5,0.5,0.5}
\definecolor{mauve}{rgb}{0.58,0,0.82}


\lstset{frame=tb,
  language=Python,
  aboveskip=3mm,
  belowskip=3mm,
  showstringspaces=false,
  columns=flexible,
  basicstyle={\small\ttfamily},
  numbers=none,
  numberstyle=\tiny\color{gray},
  keywordstyle=\color{blue},
  commentstyle=\color{dkgreen},
  stringstyle=\color{mauve},
  breaklines=true,
  breakatwhitespace=true,
  tabsize=3
}
\setlength{\droptitle}{-6em}

\makeatletter
% we use \prefix@<level> only if it is defined
\renewcommand{\@seccntformat}[1]{%
  \ifcsname prefix@#1\endcsname
    \csname prefix@#1\endcsname
  \else
    \csname the#1\endcsname\quad
  \fi}
% define \prefix@section
\newcommand\prefix@section{}
\newcommand{\prefix@subsection}{}
\newcommand{\prefix@subsubsection}{}
\renewcommand{\thesubsection}{\arabic{subsection}}
\makeatother
\DeclareMathOperator*{\argmin}{argmin}
\newcommand{\partbreak}{\begin{center}\rule{17.5cm}{2pt}\end{center}}
\newcommand{\alignbreak}{\begin{center}\rule{15cm}{1pt}\end{center}}
\newcommand{\tightalignbreak}{\vspace{-5mm}\alignbreak\vspace{-5mm}}
\newcommand{\hop}{\vspace{1mm}}
\newcommand{\jump}{\vspace{5mm}}
\newcommand{\R}{\mathbb{R}}
\newcommand{\C}{\mathbb{C}}
\newcommand{\N}{\mathbb{N}}
\newcommand{\G}{\mathbb{G}}
\renewcommand{\S}{\mathbb{S}}
\newcommand{\bt}{\textbf}
\newcommand{\xdot}{\dot{x}}
\renewcommand{\star}{^{*}}
\newcommand{\ydot}{\dot{y}}
\newcommand{\lm}{\mathrm{\lambda}}
\renewcommand{\th}{\theta}
\newcommand{\id}{\mathbb{I}}
\newcommand{\si}{\Sigma}
\newcommand{\Si}{\si}
\newcommand{\inv}{^{-1}}
\newcommand{\T}{^\intercal}
\renewcommand{\tr}{\text{tr}}
\newcommand{\ep}{\varepsilon}
\newcommand{\ph}{\varphi}
%\renewcomand{\norm}[1]{\left\lVert#1\right\rVert}
\definecolor{cit}{rgb}{0.05,0.2,0.45}
\addtolength{\jot}{1em}
\newcommand{\solution}[1]{

\noindent{\color{cit}\textbf{Solution:} #1}}

\newcounter{tmpctr}
\newcommand\fancyRoman[1]{%
  \setcounter{tmpctr}{#1}%
  \setbox0=\hbox{\kern0.3pt\textsf{\Roman{tmpctr}}}%
  \setstackgap{S}{-.9pt}%
  \Shortstack{\rule{\dimexpr\wd0+.1ex}{.9pt}\\\copy0\\
              \rule{\dimexpr\wd0+.1ex}{.9pt}}%
}

\newcommand{\Id}{\fancyRoman{2}}

% Enter the specific assignment number and topic of that assignment below, and replace "Your Name" with your actual name.
\title{STAT 37710: Homework 3}
\author{Caleb Derrickson}
\date{April 25, 2024}

\begin{document}
\onehalfspacing
\maketitle
\allowdisplaybreaks
{\color{cit}\vspace{2mm}\noindent\textbf{Collaborators:}} The TA's of the class, as well as Kevin Hefner, and Alexander Cram.

\tableofcontents
\newcommand{\scC}{\mathcal{C}}
\newcommand{\scX}{\mathcal{X}}

\newpage
\section{Problem 1}
An online algorithm, like the perceptron, is said to be conservative if it changes its hypothesis only when in makes a mistake. Let $\scC$ be a concept class and $A$ be a (not necessarily conservative) online algorithm which has a finite mistake bound $M$ on $\scC$. Prove that there is a conservative algorithm $A'$ for $\scC$ which also has mistake bound $M$.

\newpage
\section{Problem 2}
Give an example of a function $k: \scX \times \scX \to \R$ that is symmetric $(k(x, x') = k(x', x))$ and positive in the sense that $k(x, x') \geq 0$ for all $x, x' \in \scX$, but is not positive semidefinite. Conversely, give an example of a kernel that is positive semidefinite, but does not satisfy $k(x, x') \geq 0$ for all $x, x' \in \scX$.

\newpage
\section{Problem 3}
Given any function $\psi: \scX \to \scX'$, prove that if $k'$ is a psd kernel on $\scX'$, then $k(x, x') = k'(\psi(x), \psi(x'))$ is a psd kernel on $\scX$.

\newpage
\section{Problem 4}
Prove that if $k_1$ and $k_2$ are two positive semi-definite (psd) kernels on a space $\scX$, then 
\subsection{Problem 4, part a}
The function, $k(x, x') := k_1(x, x') + k_2(x, x')$ is a psd kernel on $\scX$;

\subsection{Problem 4, part b}
The function $k_\oplus ((x_1, x_2), (x_1', x_2')) = k_1(x_1, x_1') + k_2(x_2, x_2')$ is a psd kernel on $\scX \times \scX'$.

\subsection{Problem 4, part c}
Given any function $\psi: \scX \to \scX'$, prove that if $k'$ is a psd kernel on $\scX'$, then $k(x, x') = k'(\psi(x), \psi(x'))$ is a psd kernel on $\scX$.

\subsection{Problem 4, part d}
Let $\alpha(\textbf{x}, \textbf{x}')$ be the angle between $\textbf{x}, \textbf{x}' \in \R^n$. Prove that the cosine kernel $k_\angle (\textbf{x}, \textbf{x}') = \cos(\alpha(\textbf{x}, \textbf{x}'))$ is a psd kernel on $\scX = \R^n$.

\newcommand{\sgn}{\text{sgn}}
\newcommand{\scE}{\mathcal{E}}

\newpage
\section{Problem 5}
Recall that a training set $\{(x_1, y_1), (x_2, y_2), ..., (x_m, y_m)\}$ is said to have edge $\gamma$ over a set of weak classifiers $H$ if for any distribution $D$ over the training set, there is at least one weak learner $h \in H$ such that $\ep_h = \sum_{i = 1}^m D(i)\ell_{0/1}(h(x_i), y_i) \leq 1/2 - \gamma$.

\subsection{Problem 5, part a}
Using the inequality $\ell_{0/1}(z, 1) \leq e^{-z}$ prove that after $t$ rounds of boosting the running hypothesis $\hat{h}(x) = \sgn \left(\sum_{s = 1}^t \alpha_s h_s(x)\right)$ satisfies
\[\ell_{0/1} (\hat{h}(x_i), y_i) \leq m\left(\prod_{s = 1}^t Z_s\right) D_{t+1}(i)\]
for every example $i = 1, 2, ..., m$.

\subsection{Problem 5, part b}
Use this to show that  
\[\scE_{\text{train}}(\hat{h}) \leq \prod_{s = 1}^t 2\sqrt{\ep_s(1 - \ep_s)}.\]

\subsection{Problem 5, part c}
By plugging into the definition of the edge at round $s$, which is $\gamma_s = 1/2 - \ep_s$ and using the inequality $1 - z \leq e^{-z}$ prove that the training error decreases exponentially,
\[\scE_{\text{train}}(\hat{h}) \leq \exp(-2\gamma^2t),\]
as stated in a Theorem in class.

\end{document}