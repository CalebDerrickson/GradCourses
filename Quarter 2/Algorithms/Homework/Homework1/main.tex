\documentclass[12pt]{article}
\usepackage[paper=letterpaper,margin=1.5cm]{geometry}
\usepackage{amsmath}
\usepackage{amssymb}
\usepackage{amsfonts}
\usepackage{mathtools}
%\usepackage[utf8]{inputenc}
%\usepackage{newtxtext, newtxmath}
\usepackage{lmodern}     % set math font to Latin modern math
\usepackage[T1]{fontenc}
\renewcommand\rmdefault{ptm}
%\usepackage{enumitem}
\usepackage[shortlabels]{enumitem}
\usepackage{titling}
\usepackage{graphicx}
\usepackage[colorlinks=true]{hyperref}
\usepackage{setspace}
\usepackage{subfigure} 
\usepackage{braket}
\usepackage{color}
\usepackage{tabularx}
\usepackage[table]{xcolor}
\usepackage{listings}
\usepackage{mathrsfs}
\usepackage{stackengine}
\usepackage{physics}
\usepackage{afterpage}
\usepackage{pdfpages}
\usepackage[export]{adjustbox}
\usepackage{biblatex}

\setstackEOL{\\}

\definecolor{dkgreen}{rgb}{0,0.6,0}
\definecolor{gray}{rgb}{0.5,0.5,0.5}
\definecolor{mauve}{rgb}{0.58,0,0.82}


\lstset{frame=tb,
  language=Python,
  aboveskip=3mm,
  belowskip=3mm,
  showstringspaces=false,
  columns=flexible,
  basicstyle={\small\ttfamily},
  numbers=none,
  numberstyle=\tiny\color{gray},
  keywordstyle=\color{blue},
  commentstyle=\color{dkgreen},
  stringstyle=\color{mauve},
  breaklines=true,
  breakatwhitespace=true,
  tabsize=3
}
\setlength{\droptitle}{-6em}

\makeatletter
% we use \prefix@<level> only if it is defined
\renewcommand{\@seccntformat}[1]{%
  \ifcsname prefix@#1\endcsname
    \csname prefix@#1\endcsname
  \else
    \csname the#1\endcsname\quad
  \fi}
% define \prefix@section
\newcommand\prefix@section{}
\newcommand{\prefix@subsection}{}
\newcommand{\prefix@subsubsection}{}
\renewcommand{\thesubsection}{\arabic{subsection}}
\makeatother
\DeclareMathOperator*{\argmin}{argmin}
\newcommand{\partbreak}{\begin{center}\rule{17.5cm}{2pt}\end{center}}
\newcommand{\alignbreak}{\begin{center}\rule{15cm}{1pt}\end{center}}
\newcommand{\tightalignbreak}{\vspace{-5mm}\alignbreak\vspace{-5mm}}
\newcommand{\hop}{\vspace{1mm}}
\newcommand{\jump}{\vspace{5mm}}
\newcommand{\R}{\mathbb{R}}
\newcommand{\C}{\mathbb{C}}
\newcommand{\N}{\mathbb{N}}
\newcommand{\G}{\mathbb{G}}
\renewcommand{\S}{\mathbb{S}}
\newcommand{\bt}{\textbf}
\newcommand{\xdot}{\dot{x}}
\renewcommand{\star}{^{*}}
\newcommand{\ydot}{\dot{y}}
\newcommand{\lm}{\mathrm{\lambda}}
\renewcommand{\th}{\theta}
\newcommand{\id}{\mathbb{I}}
\newcommand{\si}{\Sigma}
\newcommand{\Si}{\si}
\newcommand{\inv}{^{-1}}
\newcommand{\T}{^\intercal}
\renewcommand{\tr}{\text{tr}}
\newcommand{\ep}{\varepsilon}
\newcommand{\ph}{\varphi}
%\renewcomand{\norm}[1]{\left\lVert#1\right\rVert}
\definecolor{cit}{rgb}{0.05,0.2,0.45}
\addtolength{\jot}{1em}
\newcommand{\solution}[1]{

\noindent{\color{cit}\textbf{Solution:} #1}}

\newcounter{tmpctr}
\newcommand\fancyRoman[1]{%
  \setcounter{tmpctr}{#1}%
  \setbox0=\hbox{\kern0.3pt\textsf{\Roman{tmpctr}}}%
  \setstackgap{S}{-.9pt}%
  \Shortstack{\rule{\dimexpr\wd0+.1ex}{.9pt}\\\copy0\\
              \rule{\dimexpr\wd0+.1ex}{.9pt}}%
}

\newcommand{\Id}{\fancyRoman{2}}

% Enter the specific assignment number and topic of that assignment below, and replace "Your Name" with your actual name.
\title{CMSC 37000: Homework 1}
\author{Caleb Derrickson}
\date{January 25, 2024}

\begin{document}
\onehalfspacing
\maketitle
\allowdisplaybreaks

\tableofcontents

\newpage
\section{Problem 1}
We are given $2n$ equally spaced points on a line. Half the points are black, and half are white. The goal is to connect every black point to a distinct white point with a wire so as to minimize the total wire length. \par
Below are two suggestions for greedy algorithms, which may or may not solve the problem correctly. For each suggested algorithm, if it is correct, prove its correctness. If it in incorrect, prove that by providing an input on which the algorithm does not find an optimal solution. \par
Each of the suggested algorithms performs $n$ iterations. In every iteration, it selects a single pair of points to connect and the deletes these two points from the line. In order to define each algorithm, it is now enough to specify the greedy rule for selecting the pair of points to connect.
\begin{enumerate}
    \item Greedy Rule 1: Find any pair $(a, b)$ of points, where $a$ is black and $b$ is white, and no other point lies between the two. Connect $a$ to $b$ and delete both points from the line.
    \item Greedy Rule 2: Let $a$ be the leftmost black point and let $b$ be the leftmost white point. Connect $a$ to $b$ and delete both points from the line. 
\end{enumerate}
\partbreak
\begin{solution}

    Greedy Rule number 1 mentions that $(a, b)$ can be \textit{any} pair of adjacent points. If this is the case, we can choose the pairings such that each successive pair tracks the distance of the previous. Figure \ref{fig:GR1 fail} showcases this, where the numbers according to each line correspond to the order in which they were chosen and subsequently deleted. For each iteration, the pairing scheme below does obey Greedy Rule number 1, since the prior iterations will be deleted before picking the next two points. It is also a feasible one, since each point is connected to one and only one other point. The length of wire used for this pairing is 16. We can compare this to Figure \ref{fig:GR1 better}, this pairing is feasible, and uses less wire, 8, than the one in Figure \ref{fig:GR1 fail}. Thus, the first pairing does not provide the optimal solution. Note that the pairing schema in Figure \ref{fig:GR1 better} just so happens to be Greedy Rule 2. This does not prove the optimality of Greedy Rule 2, I am just using it as a point of comparison.\par
    
    \vspace{10mm}
    \renewcommand{\d}{1.5}
    \begin{figure}[!h]
        \centering
        \begin{tikzpicture}
            \node[shape=circle, draw=black, fill=white] (1) at (0, 0) {};
            \node[shape=circle, draw=black, fill=white] (2) at (1*\d, 0) {};
            \node[shape=circle, draw=black, fill=black] (3) at (2*\d, 0) {};
            \node[shape=circle, draw=black, fill=black] (4) at (3*\d, 0) {};
            \node[shape=circle, draw=black, fill=white] (5) at (4*\d, 0) {};
            \node[shape=circle, draw=black, fill=white] (6) at (5*\d, 0) {};
            \node[shape=circle, draw=black, fill=black] (7) at (6*\d, 0) {};
            \node[shape=circle, draw=black, fill=black] (8) at (7*\d, 0) {};

            \draw (4) to[out=30, in=150] node[below, font=\scriptsize, inner sep=2pt] {1} (5);
            \draw (3) to[out=30, in=150] node[below, font=\scriptsize, inner sep=2pt] {2} (6);
            \draw (2) to[out=-30, in=-150] node[above, font=\scriptsize, inner sep=2pt] {3} (7);
            \draw (1) to[out=-30, in=-150] node[above, font=\scriptsize, inner sep=2pt] {4} (8);
        \end{tikzpicture}        
        \caption{A pairing of adjacent points which does not give an optimal solution.}
        \label{fig:GR1 fail}
    \end{figure}

    \newpage
    
    \begin{figure}[!h]
    \vspace{1.5cm}
        \centering
        \begin{tikzpicture}
            \node[shape=circle, draw=black, fill=white] (1) at (0, 0) {};
            \node[shape=circle, draw=black, fill=white] (2) at (1*\d, 0) {};
            \node[shape=circle, draw=black, fill=black] (3) at (2*\d, 0) {};
            \node[shape=circle, draw=black, fill=black] (4) at (3*\d, 0) {};
            \node[shape=circle, draw=black, fill=white] (5) at (4*\d, 0) {};
            \node[shape=circle, draw=black, fill=white] (6) at (5*\d, 0) {};
            \node[shape=circle, draw=black, fill=black] (7) at (6*\d, 0) {};
            \node[shape=circle, draw=black, fill=black] (8) at (7*\d, 0) {};

            \draw (1) to[out=30, in=150] node[above, font=\scriptsize, inner sep=2pt] {1} (3);
            \draw (2) to[out=-30, in=-150] node[below, font=\scriptsize, inner sep=2pt] {2} (4);
            \draw (5) to[out=30, in=150] node[above, font=\scriptsize, inner sep=2pt] {3} (7);
            \draw (6) to[out=-30, in=-150] node[below, font=\scriptsize, inner sep=2pt] {4} (8);
        \end{tikzpicture}        
        \caption{A more optimal pairing than above.}
        \label{fig:GR1 better}
    \end{figure}

    We will now prove the optimality of the pairing set $S$ given by Greedy Rule number 2. Suppose $I_k$ is the $k$-th iteration, which has removed $k < n$ pairings on the interval. Then $|I_k| = 2n - 2k = 2(n - k)$. Define $a_k, b_k$ as the black and white points chosen on the $k$-th iteration, and $d: I \times I \rightarrow \N$ be some distance metric. Note that measuring distances will be a proxy for the length of the wire used to connect two points. We will prove via induction on $k$ that $S$ is an optimal pairing set. 
    \begin{itemize}[-]
        \item \underline{Base Case}: $k = 1$

        \jump
        Suppose that $b_j$ is some other white point in $I_1$. Note that $b_1$ is chosen to be the leftmost white point on the interval. Then $d(a_1, b_j) = d(a_1, b_1) + d(b_1, b_j)$ by the triangle inequality.\footnote{The triangle inequality gives an equality in this point since the interval in consideration is only one dimensional, and $b_j$ lies to the right of $b_1$.} This implies that $d(a_1, b_j) \geq d(a_1, b_1)$ for any point $b_j \in I_1$. Then Greedy Rule number 2 gives an optimal pairing.

        \item \underline{Induction step}

        Suppose that Greedy Rule 2 gives an optimal pairing for iterations up to $k < n$. We will show that $k+1$ iteration also gives an optimal pairing. Note that $d(a_k, b_k) \leq d(a_k, b_j)$ for the iteration $k$ and interval $I_k$. Note that $a_i, b_i$ points have been removed from $I_k$ for all $i < k$. Then the points $a_k$ and $b_k$ have been removed from $I_k$ to produce $I_{k+1}$. Choose $a_{k+1}$ and $b_{k+1}$ to then be the leftmost black and white points of $I_{k+1}$, respectively. Then, for any $b_{k+1}' \in I_{k+1}$, $d(a_{k+1}, b_{k+1}') = d(a_{k+1}, b_{k+1}) + d(b_{k+1}, b_{k+1}')$. This then implies that $d(a_{k+1}, b_{k+1}) \leq d(a_{k+1}, b_{k+1}')$, so the pairing $(a_{k+1}, b_{k+1})$ provides the optimal pairing on $I_{k+1}$.         
    \end{itemize}

        Since each step provides an optimal pairing of points on the respective iterated interval $I_i$, then their summation, $\sum_{i = 1}^n d(a_i, b_i)$ is less than (or equal to) any other pairing schema, therefore Greedy Rule 2 provides an optimal pairing sheme. 
\end{solution}

\newpage
\section{Problem 2}
Construct the Huffman code for alphabet $\Sigma$ with 7 characters $\Sigma = \{ a, b, c, d, e, f, g\}$ that have frequencies $p(a) = 0.02, \ p(b) = 0.1, \ p(c) = 0.5, \ p(d) = 0.07, \ p(e) = 0.21, \ p(f) = 0.04,$ and $p(g) = 0.06$. Specifically, do the following:
\begin{itemize}
    \item Draw the Huffman tree for $\Sigma$.
    \item Label the leaves of the tree with characters form $\Sigma$.
    \item Write the codeword for each character. 
\end{itemize}
\partbreak
\begin{solution}

    The Problem mentions my answer need no explanations, so I will just give the tree and the codewords. Here the "$\bullet$" marker denotes an empty node. 
\alignbreak
\begin{center}
    \begin{minipage}{0.45\textwidth}
\usetikzlibrary {graphs,graphdrawing} \usegdlibrary {trees}
\tikz \graph [tree layout, minimum number of children=2,
               level distance=5mm, nodes={circle,draw, minimum size = 9mm}]
  { /$\boldsymbol{\Lambda}$ -- 
    { /\bullet -- 
        { /\bullet -- 
            {/\bullet -- 
                {{/\bullet -- 
                    {{a, f}}
                }, g},        
            { /\bullet -- {b, d}}}, 
        e}, 
    c}};
\end{minipage}
\vrule \hspace{2mm} %this is what draws the line between the minipages
\begin{minipage}{0.3\textwidth}
    \textcolor{red}{Code Words}
    \vspace{-2mm}
    \begin{itemize}[-]
        \itemsep0em 
        \item \textbf{a}: 00000
        \item \textbf{b}: 0010
        \item \textbf{c}: 1
        \item \textbf{d}: 0011
        \item \textbf{e}: 01
        \item \textbf{f}: 00001
        \item \textbf{g}: 0001
    \end{itemize}
\end{minipage}
\end{center}
\alignbreak
\end{solution}

\newpage
\section{Problem 3}
A freelance photographer prepares her schedule for the next $n$ days. She may work or rest on each of the days.
\begin{itemize}
    \item If she works on day $i$, she gets paid $p_i > 0$ dollars.
    \item However, she is not paid on those days she rests.
    \item She is not willing to work 3 days in a row.
\end{itemize}
Subject to these constraints, the photographer wants to maximize her pay.\par
Design a polynomial-time dynamic programming algorithm that given a sequence of $p_1, ..., p_n$ finds an optimal schedule. Describe your algorithm in detail. Prove its correctness. Specifically, do the following:
\begin{enumerate}
    \item Define a dynamic-programming table and explain the meaning of its entries.
    \item Write the initialization step of your algorithm.
    \item Write the recurrence formula for computing entries of the table.
    \item Explain the formula.
    \item Find the running time of your algorithm.
\end{enumerate}
\partbreak
\begin{solution}
    I have written some pseudocode below after the list for reference.     
    \begin{enumerate}
        \item I will define and keep track of two tables: T and W. Table T will keep track of the max amount of money for the iteration and Table W is a binary list that will keep track of the days to work. Note that the way T is defined, it is monotonic. Since the question asks for a schedule that returns the maximum amount of money, we return only W. If you want the maximum amount of money as well, we can return T[n].

        \item We initialize the two arrays with locally optimal values, meaning if we are considering only one day, we should work that day. Then the first entry of T would be p[0], and W[0] = 1. Note that the tables are initialized with zero in all entries. 

        \item The recurrence formulas for T and W are different, and should be handled separately. The formula for T is akin to those done in class, with T[k+1] = T[k] + max\{0, p[k-1] - p[k+1]\}. The modification to W is slightly different. Since the max number of days she is willing to work in a three day window is 2, we need to check the day two days prior. If p[k+1] is larger than p[k-1], we update the entry in W[k-1] to zero. Else, we reject it, setting W[k+1] = 0. If we have that W[k-1] = 0, then we can freely set W[k+1] = 1, regardless of T. 

        \newpage
        \item I will prove the algorithm's feasibility and optimality here.
        \alignbreak
        \begin{enumerate}[-]
            \item \underline{Feasibility}:
            
            \jump
            We will show this via induction on $|T| = n = |W|$. 
            \begin{enumerate}[-]
                \item \underline{Base Case}: $n = 1$
                
                \hop
                Then the array $p$ will consist of only 1 element. Thus, the optimal schedule is $T = \{p_1\}$\\ and $W = \{ 1\}$. This is handled accordingly in my algorithm.

                \item \underline{Induction Step}

                \hop
                Next, assume the algorithm provides a feasible schedule for some $k < n$ iteration. We will show the $k+1$ iteration holds. Note that since all cases before the $k+1$ case give a feasible schedule, we only need to consider three elements, that being $p_{k-1}, p_k,$ and $p_{k+1}$. Furthermore, since all cases before the $k+1$ cases are feasible, we can say that if $W[k-1] = 0$, then $W[k] = 1$ by the algorithm. This will 
            \end{enumerate}
        \end{enumerate}
        \alignbreak
        \item Since we are only looping over p once, and for each iteration of the for loop we are only doing O(1) operations, the overall running time of the algorithm is O(n). The memory complexity of our algorithm is slightly higher: since we are allocating memory for two arrays of length n, the memory complexity is O(2n), which in all intensive purposes, is O(n).
    \end{enumerate}
    \newpage
\begin{lstlisting}
THIS ISNT RIGHT. THINK BETTER!!
    function MaxMoney(array p)
        Let n = p.size
        Initialize T[n]
        Initialize W[n]

        set T[0] = p[0]
        set W[0] = 1

        for int k = 0...n-1 
            int value = max{0, p[k+1] - p[k-1]}
            T[k+1] = T[k] + value

            if W[k-1] == 1
                if value != 0
                    W[k-1] = 0
                    W[k+1] = 1
                else // value == 0
                    W[k+1] = 0
                endif
            else    // W[k-1] == 0
                W[k+1] = 1
            endif
        end for
        return W
    end function
\end{lstlisting}
    
\end{solution}
\end{document}