\documentclass[12pt]{article}
\usepackage[paper=letterpaper,margin=1.5cm]{geometry}
\usepackage{amsmath}
\usepackage{amssymb}
\usepackage{amsfonts}
\usepackage{mathtools}
%\usepackage[utf8]{inputenc}
%\usepackage{newtxtext, newtxmath}
\usepackage{lmodern}     % set math font to Latin modern math
\usepackage[T1]{fontenc}
\renewcommand\rmdefault{ptm}
%\usepackage{enumitem}
\usepackage[shortlabels]{enumitem}
\usepackage{titling}
\usepackage{graphicx}
\usepackage[colorlinks=true]{hyperref}
\usepackage{setspace}
\usepackage{subfigure} 
\usepackage{braket}
\usepackage{color}
\usepackage{tabularx}
\usepackage[table]{xcolor}
\usepackage{listings}
\usepackage{mathrsfs}
\usepackage{stackengine}
\usepackage{physics}
\usepackage{afterpage}
\usepackage{pdfpages}
\usepackage[export]{adjustbox}
\usepackage{biblatex}

\setstackEOL{\\}

\definecolor{dkgreen}{rgb}{0,0.6,0}
\definecolor{gray}{rgb}{0.5,0.5,0.5}
\definecolor{mauve}{rgb}{0.58,0,0.82}


\lstset{frame=tb,
  language=Python,
  aboveskip=3mm,
  belowskip=3mm,
  showstringspaces=false,
  columns=flexible,
  basicstyle={\small\ttfamily},
  numbers=none,
  numberstyle=\tiny\color{gray},
  keywordstyle=\color{blue},
  commentstyle=\color{dkgreen},
  stringstyle=\color{mauve},
  breaklines=true,
  breakatwhitespace=true,
  tabsize=3
}
\setlength{\droptitle}{-6em}

\makeatletter
% we use \prefix@<level> only if it is defined
\renewcommand{\@seccntformat}[1]{%
  \ifcsname prefix@#1\endcsname
    \csname prefix@#1\endcsname
  \else
    \csname the#1\endcsname\quad
  \fi}
% define \prefix@section
\newcommand\prefix@section{}
\newcommand{\prefix@subsection}{}
\newcommand{\prefix@subsubsection}{}
\renewcommand{\thesubsection}{\arabic{subsection}}
\makeatother
\DeclareMathOperator*{\argmin}{argmin}
\newcommand{\partbreak}{\begin{center}\rule{17.5cm}{2pt}\end{center}}
\newcommand{\alignbreak}{\begin{center}\rule{15cm}{1pt}\end{center}}
\newcommand{\tightalignbreak}{\vspace{-5mm}\alignbreak\vspace{-5mm}}
\newcommand{\hop}{\vspace{1mm}}
\newcommand{\jump}{\vspace{5mm}}
\newcommand{\R}{\mathbb{R}}
\newcommand{\C}{\mathbb{C}}
\newcommand{\N}{\mathbb{N}}
\newcommand{\G}{\mathbb{G}}
\renewcommand{\S}{\mathbb{S}}
\newcommand{\bt}{\textbf}
\newcommand{\xdot}{\dot{x}}
\renewcommand{\star}{^{*}}
\newcommand{\ydot}{\dot{y}}
\newcommand{\lm}{\mathrm{\lambda}}
\renewcommand{\th}{\theta}
\newcommand{\id}{\mathbb{I}}
\newcommand{\si}{\Sigma}
\newcommand{\Si}{\si}
\newcommand{\inv}{^{-1}}
\newcommand{\T}{^\intercal}
\renewcommand{\tr}{\text{tr}}
\newcommand{\ep}{\varepsilon}
\newcommand{\ph}{\varphi}
%\renewcomand{\norm}[1]{\left\lVert#1\right\rVert}
\definecolor{cit}{rgb}{0.05,0.2,0.45}
\addtolength{\jot}{1em}
\newcommand{\solution}[1]{

\noindent{\color{cit}\textbf{Solution:} #1}}

\newcounter{tmpctr}
\newcommand\fancyRoman[1]{%
  \setcounter{tmpctr}{#1}%
  \setbox0=\hbox{\kern0.3pt\textsf{\Roman{tmpctr}}}%
  \setstackgap{S}{-.9pt}%
  \Shortstack{\rule{\dimexpr\wd0+.1ex}{.9pt}\\\copy0\\
              \rule{\dimexpr\wd0+.1ex}{.9pt}}%
}

\newcommand{\Id}{\fancyRoman{2}}


% Enter the specific assignment number and topic of that assignment below, and replace "Your Name" with your actual name.
\title{STAT 31210: Homework 1}
\author{Caleb Derrickson}
\date{January 12, 2024}

\begin{document}
\onehalfspacing
\maketitle
\allowdisplaybreaks
{\color{cit}\vspace{2mm}\noindent\textbf{Collaborators:}} The TA's of the class, as well as Kevin Hefner, and Alexander Cram.

\tableofcontents

\newpage
\section{Problem 1.4}
Suppose that $(X, d_X)$  and $(Y, d_Y)$ are metric spaces. Prove that the Cartesian product $Z = X \times Y$ is a metric space with metric $d$ defined by 
$$d(z_1, z_2) = d_X(x_1, x_2) + d_Y(y_1, y_2),$$
where $z_1 = (z_1, y_1)$ and $z_2 = (z_2, y_2)$.
\partbreak

\begin{solution}

To show $(Z, d)$ is a metric space, we just need to show that $d$ is indeed a metric. This requires $d$ to satisfy the three stated properties for a metric (Definition 1.1). These will be shown below. 
\begin{itemize}[-]
    \item \underline{$d(z_1, z_2) \geq 0$ and $d(z_1, z_2) = 0 \iff z_1 = z_2$.}

    \jump
    Note since $d_X$ and $d_Y$ are both given as metrics on their respective spaces, they obey the given properties of a metric. Therefore, $d_X$ and $d_Y$ are both positive functions. Since the addition of two positive functions is positive, then $d_X + d_Y \geq 0$. But this is the definition for $d$, so $d \geq 0$. \par

    Next we need to show that $d(z_1, z_2) = 0 \iff z_1 = z_2.$ Again, since $d_X$ and $d_Y$ are metrics, then $d_X(x_1, x_2) = 0 \iff x_1 = x_2$ (same for $d_Y$). Therefore, the sum $d_X(x_1, x_2) + d_Y(y_1, y_2)$ equals zero if and only if $x_1 = x_2$ and $y_1 = y_2$ ($d_X = -d_Y$ is only valid when $d_X = 0, d_Y = 0$.) Therefore, $d(z_1, z_2) = d_X(x_1, x_2) + d_Y(y_1, y_2) = 0$ if and only if $z_1 = z_2$.

    \item \underline{$d(z_1, z_2) = d(z_2, z_1)$.}

    \jump
    We will again use the fact that $d_X$ and $d_Y$ are metrics. 
    \tightalignbreak
    \begin{align*}
        d(z_1, z_2) &= d_X(x_1, x_2) + d_Y(y_1, y_2) &\text{(Given.)}\\
        &= d_X(x_2, x_1) + d_Y(y_2, y_1) &(d_Z, d_Y \text{are metrics.)}\\
        &= d(z_2, z_1) &\text{(Given.)}
    \end{align*}
    \endtightalignbreak
\newpage
    \item \underline{$d(z_1, z_3) \leq d(z_1, z_2) + d(z_2, z_3)$.}

    \jump
    Defining $z_3 = (x_3, y_3)$, we can write the following. 
    \tightalignbreak
    \begin{align*}
        d(z_1, z_3) &= d_X(x_1, x_3) + d_Y(y_1, y_3) &\text{(Given.)}\\
        &\leq d_X(x_1, x_2) + d_X(x_2, x_3) + d_Y(y_1, y_3) &\text{($d_X$ is a metric.)}\\
        &\leq d_X(x_1, x_2) + d_X(x_2, x_3) + d_Y(y_1, y_2) + d_Y(y_2, y_3) &\text{($d_Y$ is a metric.)}\\
        &= d_X(x_1, x_2) + d_Y(y_1, y_2) + d_X(x_2, x_3) + d_Y(y_2, y_3) &\text{(Rearranging.)}\\
        &= d(z_1, z_2) + d(z_2, z_3) &\text{(By definition.)}
    \end{align*}
    \endtightalignbreak
\end{itemize}
\end{solution}

\newpage
\section{Problem 1.12}
Let $(X, d_X), (Y, d_Y),$ and $(Z, d_Z)$ be metric spaces and let $f: X \rightarrow Y$, and $g: Y \rightarrow Z$ be continuous functions. Show that the composition
$$h = g\circ f: X\rightarrow Z,$$
defined by $h(x) = g(f(x)),$ is also continuous.
\partbreak
\begin{solution}

    We will follow Definition 1.26 in the book. Let $x_0 \in X, y_0 = f(x_0) \in Y$. Since $f$ is continuous, $\forall \ \ep_1 > 0 \ \exists \ \delta_1 > 0$ such that for $d_X(x, x_0) < \delta_1 \implies d_Y(f(x), f(x_0)) < \ep_1$. Furthermore, $g$ is continuous, so $\forall \ \ep_2 > 0 \ \exists \ \delta_2 > 0$ such that for $d_Y(y, y_0) < \delta_2 \implies d_Z(g(y), g(y_0)) < \ep_2$. If we let $\delta_2 = \ep_1$, then $d_Y(f(x), f(x_0)) < \ep_1 \implies d_Z(g(f(x)), g(f(x_0))) < \ep_2$. Letting $\delta_1 = \ep_1$, then $d_X(x, x_0) < \delta_1 \implies d_Y(f(x), f(x_0)) < \ep_1 \implies d_Z(g(f(x)), g(f(x_0))) < \ep_2$. Therefore, $h = g\circ f$ is continuous at $x_0$. Since $x_0$ was chosen arbitrarily, then $h$ is continuous. \hfill \square
\end{solution}

\newpage
\section{Problem 1.15}
Prove that every compact subset of a metric space is closed and bounded. Prove that a closed subset of a compact space is compact. 
\partbreak
\begin{solution}

    \begin{itemize}[-]
        \item \underline{Every compact subset of a metric space is closed and bounded.}

        \jump
        Let $(X, d)$ be a metric space and $K \subset X$ is a compact subset. By Theorem 1.62, this equivalent to saying $K$ is sequentially compact. So let $x_n$ be a converging sequence in $\overline{K}$\footnote{The overline will denote the closure of a set.} with $x_n \rightarrow x$ as $n \rightarrow \infty$. If $K$ is closed, then $\overline{K} = K$. By Proposition 1.41, $x \in \overline{K}$. Let $x_{n_\alpha}$ be a subsequence of $x_n$. By definition, $x_{n_\alpha} \rightarrow k \in K$ as $\alpha \rightarrow \infty$. However, $x_{n_\alpha} \rightarrow x$ as $\alpha \rightarrow \infty$. Since the limit of a sequence is unique, then $k = x$. Therefore, $x \in K$, so $K$ is closed.

        \jump
        To show that $K$ is bounded, we wish to find $r > 0$ such that for any $x, y \in K, d(x, y) < r$. Let $x \in K$, and consider taking open balls around $x$, $B_a(x), a > 0$. Then the set $ \mathbb{O} = \{ B_a(x) : a > 0 \} $ defines an open cover over over $K$. Since $K$ is compact, there exists a finite subcover of $\mathbb{O}$ which covers $K$. Suppose $B_{r_1}(x), ... , B_{r_n}(x)$ is one such subcover. Let $r = \max \{ \ r_i : i \leq n \}$. Then for any $x, y \in K, \  d(x, y) < r$. Therefore, $K$ is bounded. 

        \item \underline{Prove that a closed subset of a compact space is compact.}

        \jump
        Suppose that $K$ is a compact subset of a metric space $(X, d)$, and $T \subseteq K$, where $T$ is closed. We need to show that $T$ is compact. Suppose $x_n$ is a sequence in $T$ such that $x_n \rightarrow x \in X$. Since $T \subseteq K$ and $K$ is compact (i.e. sequentially compact), then there exists a subsequence $x_{n_\alpha} \in K$ where $x_{n_\alpha} \rightarrow x$ as $\alpha \rightarrow \infty$. Note that $T$ is closed, therefore, $x \in T$. Therefore, $T$ is sequentially compact, therefore compact.    
    \end{itemize}
    
\end{solution}

\newpage
\section{Problem 1.16}
Suppose that $F$ and $G$ are closed and open subsets of $\R^n$, respectively, such that $F \subset G$. Show that there is a continuous function $f: \R^n \rightarrow \R$ such that:
\begin{enumerate}[(a)]
    \item $0 \leq f(x) \leq 1$;
    \item $f(x) = 1$ for $x \in F$;
    \item $f(x) = 0$ for $x \in G^c$.
\end{enumerate}
\partbreak
\begin{solution}

    Consider the function
    \[
    f(x) = \frac{d(x, G^c)}{d(x, G^c) + d(x, F)}
    \]
    where $d(x, F) = \min \{ d(x, y) : y \in F \}$, similarly for $G^c$. Explicitly showing this function is continuous, we would need to provide an $\ep -\delta$ proof. However, since $f(x)$ is a ratio of two continuous functions\footnote{Here I am taking for granted that $d$ is a continuous function. This follows from the definition of a metric.}, and the denominator is never zero, $f(x)$ does not limit to $\infty$ for any $x \in \R^n$, thus is continuous. Note that the denominator is never zero since, if it was, then there exists an $x \in \R^n$ where $x \in F\cap G^c$. However, since $F \subset G$, $F \not\subset G^c$, so $F\cap G^c = \phi$\footnote{I denote $\phi$ as the empty set.}. Therefore, $x \in \phi$, which is a contradiction. Therefore, the denominator is never zero. We just now need to show that $f$ obeys the enumerated properties.

    \begin{itemize}[(a)]
        \item Note that $f(x)$ can be rewritten as
        \[
        f(x) = 1 - \frac{d(x, F)}{d(x, G^c) + d(x, F)}.
        \]
        Since the second term is $\geq 0$ (the denominator is $>0$ and the numerator is $\geq 0$), this implies that $f(x) \leq 1$. Furthermore, we can also ration out that $f(x)$ is bounded below by zero, since $f(x)$ is a ratio of two functions which are positive. Therefore, $0 \leq f(x) \leq 1$.
        \item Note that if $x \in F$, $d(x, F) = 0$. Therefore, $f(x)$ simplifies to 
        \[
        f(x \in F) = \frac{d(x, G^c)}{d(x, G^c)} = 1.
        \]
        \item If $x \in G^c$, $d(x, G^c) = 0$. Therefore, $f(x)$ simplifies to 
        \[
        f(x \in G^c) = \frac{0}{d(x, F)} = 0.
        \]
    \end{itemize}
\end{solution}
\newpage
\section{Problem 1.20}
Let $X$ be a normed linear space. A series $\sum x_n$ in $X$ is \textit{absolutely convergent} if $\sum \norm{x_n}$ converges to a finite value in $\R$. Prove that $X$ is a Banach space if and only if every absolutely convergent series converges.
\partbreak

\begin{solution}

\begin{itemize}
    \item [\underline{$\implies$}:] Suppose $X$ is a Banach space. Then $X$ is complete with respect to the metric $d(x, y) = \norm{x - y}.$ We wish to show that every absolutely convergent series converges.\par

    Let $\sum x_n \in X$ be an absolutely convergent series. Denote $s_n = \sum_{i = 1}^n x_n$. Note that since $\sum \norm{x_n} < \infty$, there then exists $N \in \N$ such that for any $\ep > 0$, $\sum_{i = N}^\infty \norm{x_n} < \ep$. Notice that $\norm{s_n - s_m} = \norm{\sum_{i = m+1}^n x_i} \leq \sum_{i = m+1}^n\norm{x_i}$ for any $n, m$. If we choose $n, m \geq N$, then $\norm{s_n - s_m} < \ep$, by the bound shown above. Thus $s_n$ is a Cauchy sequence in $X$, which by assumption means that $s_n$ converges. Therefore, $\sum x_n$ is a convergent sum. 

    \item [\underline{$\impliedby$}:] Here we assume that every absolutely convergent series converges in $X$, we then need to show that $X$ is a Banach space. Let $x_n$ be a Cauchy sequence in $X$, and a subsequence $x_{n_k}$ for $k \geq 1$. Since $x_n$ is Cauchy, we can find a subsequence for which $\norm{x_{n_{k+1}} - x_{n_k}} < \ep$, namely $\norm{x_{n_{k+1}} - x_{n_k}} \leq 2^{-k}$. Note that 
    \[
    \sum_{k \geq 1} \norm{x_{n_{k+1}} - x_{n_k}} \leq \sum_{k \geq 1} 2^{-k} = 1.
    \]
    Define a new sequence $y_k$ such that $y_1 = x_{n_1}$, $y_{k+1} = x_{n_{k+1}} - x_{n_k}$. Then by above, 
    \[
    \sum_{k \geq 1}\norm{y_k} = \norm{x_{n_1}} + \sum_{k \geq 1} \norm{x_{n_{k+1}} - x_{n_k}} \leq \norm{x_{n_1}} + 1.
    \]
    Thus $\sum y_k$ is an absolutely convergent series in $X$, thus is a convergent series. Since $x_n$ has a convergent subsequence ($y_k$), then $x_n$ converges in $X$, thus $X$ is a Banach space. 
\end{itemize}
\end{solution}

\newpage
\section{Problem 1.27}
Suppose that $(x_n)$ is a sequence in a compact metric space with the property that every convergent subsequence has the same limit $x$. Prove that $x_n \rightarrow x$ as $n \rightarrow \infty$.
\partbreak
\begin{solution}
    
\end{solution}
\end{document}