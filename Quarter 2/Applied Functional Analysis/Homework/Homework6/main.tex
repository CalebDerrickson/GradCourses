\documentclass[12pt]{article}
\usepackage[paper=letterpaper,margin=1.5cm]{geometry}
\usepackage{amsmath}
\usepackage{amssymb}
\usepackage{amsfonts}
\usepackage{mathtools}
%\usepackage[utf8]{inputenc}
%\usepackage{newtxtext, newtxmath}
\usepackage{lmodern}     % set math font to Latin modern math
\usepackage[T1]{fontenc}
\renewcommand\rmdefault{ptm}
%\usepackage{enumitem}
\usepackage[shortlabels]{enumitem}
\usepackage{titling}
\usepackage{graphicx}
\usepackage[colorlinks=true]{hyperref}
\usepackage{setspace}
\usepackage{subfigure} 
\usepackage{braket}
\usepackage{color}
\usepackage{tabularx}
\usepackage[table]{xcolor}
\usepackage{listings}
\usepackage{mathrsfs}
\usepackage{stackengine}
\usepackage{physics}
\usepackage{afterpage}
\usepackage{pdfpages}
\usepackage[export]{adjustbox}
\usepackage{biblatex}

\setstackEOL{\\}

\definecolor{dkgreen}{rgb}{0,0.6,0}
\definecolor{gray}{rgb}{0.5,0.5,0.5}
\definecolor{mauve}{rgb}{0.58,0,0.82}


\lstset{frame=tb,
  language=Python,
  aboveskip=3mm,
  belowskip=3mm,
  showstringspaces=false,
  columns=flexible,
  basicstyle={\small\ttfamily},
  numbers=none,
  numberstyle=\tiny\color{gray},
  keywordstyle=\color{blue},
  commentstyle=\color{dkgreen},
  stringstyle=\color{mauve},
  breaklines=true,
  breakatwhitespace=true,
  tabsize=3
}
\setlength{\droptitle}{-6em}

\makeatletter
% we use \prefix@<level> only if it is defined
\renewcommand{\@seccntformat}[1]{%
  \ifcsname prefix@#1\endcsname
    \csname prefix@#1\endcsname
  \else
    \csname the#1\endcsname\quad
  \fi}
% define \prefix@section
\newcommand\prefix@section{}
\newcommand{\prefix@subsection}{}
\newcommand{\prefix@subsubsection}{}
\renewcommand{\thesubsection}{\arabic{subsection}}
\makeatother
\DeclareMathOperator*{\argmin}{argmin}
\newcommand{\partbreak}{\begin{center}\rule{17.5cm}{2pt}\end{center}}
\newcommand{\alignbreak}{\begin{center}\rule{15cm}{1pt}\end{center}}
\newcommand{\tightalignbreak}{\vspace{-5mm}\alignbreak\vspace{-5mm}}
\newcommand{\hop}{\vspace{1mm}}
\newcommand{\jump}{\vspace{5mm}}
\newcommand{\R}{\mathbb{R}}
\newcommand{\C}{\mathbb{C}}
\newcommand{\N}{\mathbb{N}}
\newcommand{\G}{\mathbb{G}}
\renewcommand{\S}{\mathbb{S}}
\newcommand{\bt}{\textbf}
\newcommand{\xdot}{\dot{x}}
\renewcommand{\star}{^{*}}
\newcommand{\ydot}{\dot{y}}
\newcommand{\lm}{\mathrm{\lambda}}
\renewcommand{\th}{\theta}
\newcommand{\id}{\mathbb{I}}
\newcommand{\si}{\Sigma}
\newcommand{\Si}{\si}
\newcommand{\inv}{^{-1}}
\newcommand{\T}{^\intercal}
\renewcommand{\tr}{\text{tr}}
\newcommand{\ep}{\varepsilon}
\newcommand{\ph}{\varphi}
%\renewcomand{\norm}[1]{\left\lVert#1\right\rVert}
\definecolor{cit}{rgb}{0.05,0.2,0.45}
\addtolength{\jot}{1em}
\newcommand{\solution}[1]{

\noindent{\color{cit}\textbf{Solution:} #1}}

\newcounter{tmpctr}
\newcommand\fancyRoman[1]{%
  \setcounter{tmpctr}{#1}%
  \setbox0=\hbox{\kern0.3pt\textsf{\Roman{tmpctr}}}%
  \setstackgap{S}{-.9pt}%
  \Shortstack{\rule{\dimexpr\wd0+.1ex}{.9pt}\\\copy0\\
              \rule{\dimexpr\wd0+.1ex}{.9pt}}%
}

\newcommand{\Id}{\fancyRoman{2}}

% Enter the specific assignment number and topic of that assignment below, and replace "Your Name" with your actual name.
\title{STAT 31210: Homework 6}
\author{Caleb Derrickson}
\date{February 16, 2024}

\begin{document}
\onehalfspacing
\maketitle
\allowdisplaybreaks
{\color{cit}\vspace{2mm}\noindent\textbf{Collaborators:}} The TA's of the class, as well as Kevin Hefner, and Alexander Cram.

\tableofcontents



\newpage
\section{Problem 7.1}
Let $\ph_n$ be the function defined as
\[\ph_n(x) = c_n(1 + cos(x))^n.\]

\subsection{Problem 7.1, part a}
Prove (7.5) for $\ph_n$, that is, 
\[\lim_{n\rightarrow \infty} \int_{\delta \leq |x| \leq \pi} \ph_n (x) \ dx = 0 \quad \text {for every } \delta > 0.\]
\partbreak
\begin{solution}

    To evaluate this expression, we will consider the integral separately. Note that over the given domain, we can separate this integral into two, depending on our approach to $\delta$.
    \[\int_{\delta \leq |x| \leq \pi} \ph_n(x) \ dx = \int_{-\pi}^{-\delta} \ph_n(x) \ dx + \int_{\delta}^\pi \ph_n(x) \ dx\]
    Since these two integrals are the same, up to bound evaluations, we will consider only solving an indefinite integral, then applying the two bounds. We then have
    \[\int \ph_n(x) \ dx = \int (1 + \cos(x))^n \ dx\]
    Applying the Binomial Theorem, we can write
    \[\int (1 + \cos(x))^n \ dx = \int \sum_{k = 0}^n {n\choose k} \cos^k (x) \ dx\]
    Note that I have simplified the powers of 1 to just 1. We can then express $\cos(x)$ as a sum of complex exponentials. That is, $\cos(x) = \frac{1}{2}(e^{ix} + e^{-ix})$. Plugging this is, we have
    \[\int \sum_{k = 0}^n {n\choose k} \cos^k (x) \ dx = \int \sum_{k = 0}^n {n\choose k} \left(\frac{1}{2} (e^{ix} + e^{-ix}) \right)^k \ dx = \int \sum_{k = 0}^n {n\choose k} \frac{1}{2^k}\left( e^{ix} + e^{-ix} \right)^k \ dx\]
    Applying the Binomial Theorem once more, we have
    \[\int \sum_{k = 0}^n {n\choose k} \frac{1}{2^k}\left( e^{ix} + e^{-ix} \right)^k \ dx = \int \sum_{k = 0}^n {n\choose k} \frac{1}{2^k}\sum_{l = 0}^k {k \choose l} \left( e^{ix}\right)^{(k - l)} \left( e^{-ix}\right)^l  \ dx\]
    As we are raising an exponential by a power in the inner summation, we can simplify the expression depending only on $x$ by
    \[\left( e^{ix}\right)^{(k - l)} \left( e^{-ix}\right)^l = e^{i(k - l)x}  e^{-ilx}=  e^{i(k - 2l)x}\]
    Therefore, we have 
    \[\int \sum_{k = 0}^n {n\choose k} \frac{1}{2^k}\sum_{l = 0}^k {k \choose l} \left( e^{ix}\right)^{(k - l)} \left( e^{-ix}\right)^l  \ dx = \int \sum_{k = 0}^n {n\choose k} \frac{1}{2^k}\sum_{l = 0}^k {k \choose l} e^{i(k - 2l)x} \ dx\]
    Since both summations are finite, we are free to switch the order of summation and integration. We then need to evaluate the following: 
    \[\sum_{k = 0}^n {n\choose k} \frac{1}{2^k}\sum_{l = 0}^k {k \choose l} \int e^{i(k - 2l)x} \ dx\]
    This is a simple integral to take. By standard calculus, we have 
    \[\int e^{i(k - 2l)x} \ dx = \frac{-i}{k - 2l} e^{i(k - 2l)x}\]
    I will omit the constant of integration, since I am immediately applying the bounds. For each integral, we get the same result, up to applying bounds. Since there is no explicit need to have the two \textit{finite} summations separate, we can sum over them at the same time to get
    \[\int_{-\pi}^{-\delta} \ph_n(x) \ dx + \int_{\delta}^\pi \ph_n(x) \ dx = \sum_{k = 0}^n {n\choose k} \frac{1}{2^k}\sum_{l = 0}^k {k \choose l}\left(\frac{-i}{k - 2l}\right)\left[ e^{i(k - 2l)x}\Big|_{-\pi}^{-\delta} + e^{i(k - 2l)x}\Big|_{\delta}^\pi\right]\]
    Applying bounds, 
    \[ \sum_{k = 0}^n {n\choose k} \frac{1}{2^k}\sum_{l = 0}^k {k \choose l}\left(\frac{-i}{k - 2l}\right)\left[ e^{-i(k - 2l)\delta} - e^{-i(k - 2l)\pi} + e^{i(k - 2l)\pi} - e^{i(k - 2l)\delta} \right]\]
    Note that the value of the integral only depends on the evaluation of the exponential inside the brackets. So we will investigate them. Note that we can rewrite the exponentials as sines via
    \[e^{-i(k - 2l)\delta} - e^{-i(k - 2l)\pi} + e^{i(k - 2l)\pi} - e^{i(k - 2l)\delta} = -2i\sin((k -2l)\delta) +2i\sin(k - 2l)\pi)\]
    
    Using the sine angle difference formula, $\sin(\alpha -\beta) = \sin(\alpha)\cos(\beta) - \cos(\alpha)\sin(\beta)$, we get
    \[\sin(k\delta)\cos(2l\delta) - cos(k\delta)\sin(2l \delta) + \sin(k\pi)\cos(2l\pi) - \cos(k\pi) \sin(2l\pi)\]
    No matter what the values of $k$ and $l$ are, we always have $\sin(k\pi) = 0$, and $\sin(2\pi l) = 0$. Therefore, both terms will equal zero, effectively canceling everything out. Also, this should hold for all $\delta > 0$, so  it should hold for the bound of zero. Therefore, we have that 
    \[\int_{-\pi}^{-\delta} \ph_n(x) \ dx + \int_{\delta}^\pi \ph_n(x) \ dx = 0.\]
    Applying the limit as $n \into \infty$, we have thus shown  
    \[\lim_{n\rightarrow \infty} \int_{\delta \leq |x| \leq \pi} \ph_n (x) \ dx = 0 \quad \text {for every } \delta > 0.\]
\end{solution}


\newpage
\subsection{Problem 7.1, part b}
Prove that if the set $\scP$ of trigonometric polynomials is dense in the space of periodic functions on $\scT$ with the uniform norm, then $\scP$ is dense in the space of all continuous functions on $\scT$ with the $L^2$-norm.
\partbreak
\begin{solution}

    Suppose that $f$ is a periodic function. Since $\scP$ is dense in the space of periodic functions, there exists a sequence $f_n \in \scP$ such that $\norm{f_n - f}_{\infty} \into 0$ as $n \into \infty$. That is, there exists a $N \in \N$ such that when $n \geq N \implies \sup_{x \in \R}|f_n(x) - f(x)| < \frac{\ep}{\sqrt{2\pi}}$. If the supremum of the sequence $|f_n(x) - f(x)| < \frac{\ep}{\sqrt{2\pi}}$, then $|f_n(x) - f(x)| < \frac{\ep}{\sqrt{2\pi}}$ for every $x \in \R$. Moreover, $|f_n(x) - f(x)|^2 < \left(\frac{\ep}{\sqrt{2\pi}}\right)^2$. Then, 
    \[\norm{f_n - f}_{L^2}^2 = \int_{\scT} |f_n(x) - f(x)|^2 \ dx < \int_{\scT} \frac{\ep^2}{2\pi} \ dx= \ep^2.\]
    Therefore, $\norm{f_n - f}_{L^2} < \ep$, implying that $\scP$ is also dense with respect to the $L^2$-norm. 

\end{solution}


\newpage
\subsection{Problem 7.1, part c}
Is $\scP$ is dense in the space of all continuous functions on $[0, 2\pi]$ with the uniform norm?
\partbreak
\begin{solution}

    Unless I'm missing something, this is the result of Theorem 7.3, which states that the set of trigonometric polynomials are dense in $C(\scT)$ with respect to the uniform norm. The only thing that is different is $\scT$ is defined on the interval $[-\pi, \pi]$, which for every function $f$ defined on $\scT$, we can define $g(x) := f(x+\pi)$. Since $f(x)$ is continuous and $2\pi$ periodic, then so is $g(x)$. Therefore, the claim has been shown.  
\end{solution}
\newpage
\section{Problem 7.2}
Suppose that $f: \scT \into \C$ is a continuous function, and 
\[S_N = \frac{1}{\sqrt{2\pi}} \sum_{n = -N}^{N} \hat{f}_n e^{inx} \]
is the $N$th partial sum of its Fourier series

\subsection{Problem 7.2, part a}
Show that $S_N = D_N \ast f$, where $D_N$ is the \textit{Dirichlet kernel}
\[D_N = \frac{1}{2\pi}\frac{\sin [(N + 1/2)x]}{\sin(x/2)}\]
\partbreak
\begin{solution}

    We aim to show
    \[D_N \ast f = \int_\scT D_N(x - y)f(y) \ dy = S_N.\]
    Note that the form given for the Dirichlet kernel can be rewritten as follows (this will be proven after its use):
    \[D_N = \frac{1}{2\pi}\frac{\sin [(N + 1/2)x]}{\sin(x/2)} = \frac{1}{2\pi}\sum_{n = -N}^{N} e^{inx}\]
    The latter is far more applicable to our problem. Applying this, we have the following:
    \tightalignbreak
    \begin{align*}
        &D_N\ast f = \frac{1}{2\pi}\int_{\scT} D_n(x - y) f(y) \ dy &\text{(Given.)}\\
        &= \frac{1}{2\pi}\int_{\scT} \sum_{n =-N}^N e^{in(x-y)} f(y) \ dy &\text{(By above.)}\\
        &= \frac{1}{2\pi} \sum_{n =-N}^N \int_{\scT}e^{in(x-y)} f(y) \ dy &\text{(Finite sum.)}\\
        &= \frac{1}{2\pi} \sum_{n =-N}^N \int_{\scT}e^{inx}e^{-iny} f(y) \ dy &\text{(Product of exponentials.)}\\
        &= \frac{1}{2\pi} \sum_{n =-N}^N e^{inx}\int_{\scT}e^{-iny} f(y) \ dy &\text{(Constant of integration.)}\\
        &=\frac{1}{\sqrt{2\pi}}\sum_{n =-N}^N \left( \frac{1}{\sqrt{2\pi}}\int_{\scT}e^{-iny} f(y) \ dy\right)e^{inx}  &\text{(Rearranging.)}\\
        &=\frac{1}{\sqrt{2\pi}}\sum_{n =-N}^N \hat{f}_ne^{inx}  &\text{(By definition.)}\\
        &= S_N &\text{(Given.)}
    \end{align*}    
    \vspace{-12mm}\alignbreak
    We will next show that the Dirichlet kernel can be rewritten in such a form. Using the geometric series expansion, we can write
    \[\frac{\sin [(n + 1/2)x]}{\sin(x/2)} = \frac{-2i\sin [(n + 1/2)x]}{-2i\sin(x/2)} = \frac{e^{-(n+1/2)ix} - e^{(n+1/2)ix}}{e^{-ix/2} - e^{ix/2}} = \sum_{k = -n}^n e^{ikx}\]
    The intermediate step employed $\sin(x) = \frac{1}{2i}(e^{ix} - e^{-ix})$. Usually the geometric series is written from \textit{zero} to n. That is, 
    \[\sum_{k = 0}^n ar^k = a\left[ \frac{1 - r^{n+1}}{1 - r}\right]\]
    This can be extended to negative terms via
    \[\sum_{k = -n}^n r^k = r^{-n}\left[ \frac{1 - r^{2n+1}}{1-r}\right].\]
    
\end{solution}


\newpage
\subsection{Problem 7.2, part b}
Let $T_N$ be the mean of the first $N+1$ partial sums,
\[T_N = \frac{1}{N+1} \left \{ S_0 + S_1 + \dots + S_N\right \}.\]
Show that $T_N = F_N \ast f$, where $F_N$ is the \textit{Fej\'er kernel}
\[F_N(x) = \frac{1}{2\pi (N+1)}\left( \frac{\sin [(N + 1)x/2]}{\sin(x/2)}\right)^2 \]
\partbreak
\begin{solution}
    
    The Fej\'er kernel can be written in terms of the Dirichlet kernel in the following manner (this will be proven after its use):
    \[F_N(x) = \frac{1}{N+1}\sum_{k = 0}^N D_k(x)\]
    Using this information, we can write the following:
    \tightalignbreak
    \begin{align*}
        &F_N \ast f = \int_T F_N(x - y)f(y) \ dy &\text{(Given.)}\\
        &= \frac{1}{2\pi (N+1)}\int_\scT \sum_{k = 0}^N D_k(x - y) f(y) \ dy &\text{(By above.)}\\
        &= \frac{1}{N+1}\sum_{k = 0}^N\int_{\scT}D_k(x - y) f(y) \ dy &\text{(Finite sum.)}\\
        &= \frac{1}{N+1}\sum_{k = 0}^NS_k \ dy &\text{(By part a.)}\\
        &= \frac{1}{N+1}\{ S_0 + S_1 + \cdots + S_N\} &\text{(Expanding.)}\\
        &= T_N &\text{(Given.)}
    \end{align*}
    \vspace{-12mm}\alignbreak
    \newpage
    We next need to show that the Fej\'er kernel can be written in such a manner. This can be shown by the following:
    \tightalignbreak
    \begin{align*}
        &F_N(x) = \frac{1}{N+1}\left(\frac{\sin\left[ (N+1)x/2\right]}{\sin(x/2)}\right)^2 &\text{(Given.)}\\
        &= \frac{1}{N+1}\frac{1}{\sin^2(x/2)}\left(\sin\left[ (N+1)x/2\right]\right)^2 &\text{(Rearranging.)}\\
        &= \frac{1}{N+1}\frac{1}{\sin^2(x/2)}\frac{1 - \cos \left[ (N+1)x\right]}{2} &\text{(Trig identity.)}\\
        &= \frac{1}{N+1}\frac{1}{2\sin^2(x/2)}\sum_{k = 0}^N \left[ \cos(kx) - \cos((k+1)x)\right] &\text{(Reinterpretation.)}\\
        &= \frac{1}{N+1}\frac{1}{\sin^2(x/2)}\sum_{k = 0}^N \left[ \sin((k+1/2)x) \sin(x/2)\right] &\text{(Trig identity.)}\\
        &= \frac{1}{N+1}\frac{1}{\sin(x/2)}\sum_{k = 0}^N \left[ \sin((k+1/2)x) \right] &\text{(Simplifying.)}\\
        &= \frac{1}{N+1}\sum_{k = 0}^N \frac{\left[ \sin((k+1/2)x) \right]}{\sin(x/2)} &\text{(Rearranging.)}\\
        &= \frac{1}{N+1}\sum_{k = 0}^N D_k(x) &\text{(Given.)} 
    \end{align*}
    \vspace{-12mm}\alignbreak
\end{solution}

\newpage
\subsection{Problem 7.2, part c}
Which of the families $(D_N)$ and $(F_N)$ are the approximate identities as $N\rightarrow \infty$? What can you say about the uniform convergence of the partial sums $S_N$ and the averaged partial sums $T_N$ to $f$?
\partbreak
\begin{solution}

    Firstly, the Dirichlet kernel fails to be strictly positive, which can be seen on its Wikipedia page. So the Dirichlet kernel immediately fails at being an approximate identity. By the form of the Fej\'er kernel given, its easy to see that its strictly positive, passing the first required condition. Next, we want to show that its integral equals one for all $N \in \N$. This requires separate handling for the $n = 0$ case, as we will see when evaluating the integral. From part b, we had found an equivalent form for writing the Fej\'er kernel in terms of a sum over Dirichlet kernels. That is,
    \[F_N = \frac{1}{2\pi (N+1)}\sum_{k = 0}^N D_k(x) = \frac{1}{2\pi(N+1)} \sum_{k = 0}^N\sum_{n = -k}^k e^{inx}\]
    Supposing $n \neq 0$ in any of the terms, we can write
    \[\int_{-\pi}^\pi F_N(x) \ dx  = \frac{1}{2\pi(N+1)} \sum_{k = 0}^N\sum_{n = -k}^k \int_{-\pi}^\pi e^{inx} \ dx = \left[\dots\right] \frac{1}{in}e^{inx}\Big|_{-\pi}^\pi = \left[\dots\right] \frac{2}{n}\sin (n\pi) = 0.  \]
    For the case when $n = 0$, we see that this happens for every term in the series. Since there are $N+1$ terms in the series in which we are evaluating the integral at $n = 0$, we will need to multiply in the $N+1$ at the end. Note that $e^{i(0)x} = 1$ for all $x$, so 
    \[\int_{-\pi}^\pi 1 \ dx = 2\pi.\]
    Thus, adding up all the zero terms, we get that the value of the integral is equal $2\pi(N+1)$, implying that $\int F_N = 1$ for all $N \in \N$. Next, we need to show that 
    \[\lim_{n \into \infty} \int_{\delta \leq |x| \leq \pi}\ph_n(x) \ dx = 0, \quad \text{ for every } \delta > 0.\]
    This is slightly more involved. We can note that instead of evaluating the one integral from $-\pi$ to $\pi$, we have 
    \[\int_{\delta \leq |x| \leq \pi} F_N(x) \ dx = \int_{-\pi}^{-\delta} F_N(x) \ dx + \int_{\delta}^{\pi} F_N(x) \ dx,\]
    where we will take $N \into \infty$ after integral evaluation. We will employ the same trick as was performed earlier, where we evaluate an indefinite integral and apply different bounds then. Since we have to worry about the case when $n = 0$, we will evaluate this first. Everything up to integral evaluation is the same as the previous part, so
    \[\int F_N(x) = \frac{1}{2\pi(N+1)}\sum_{k = 0}^N\sum_{n = -k}^k \left[\int_{-\pi}^{-\delta} 1 \ dx + \int_{\delta}^\pi 1 \ dx \right]= \frac{1}{2\pi(N+1)} (N+1) 2(\pi - \delta) = 1 - \frac{\delta}{\pi}.\]
    We now consider the case when $n \neq 0$, which will mean the the integral is nontrivial, but easy, so
    \begin{align*}
        &\int F_N(x) = \frac{1}{2\pi(N+1)}\sum_{k = 0}^N\sum_{n = -k}^k \left[\int_{-\pi}^{-\delta} e^{inx}+ \int_{\delta}^\pi e^{inx} \right]= \left[\cdots\right] \frac{1}{in}\left[ e^{-in\delta} - e^{-in\pi} + e^{in\pi} - e^{in\delta}\right]\\
        &= \frac{1}{2\pi(N+1)}\sum_{k = 0}^N\sum_{n = -k}^k \frac{1}{n}\left[ 2\sin(n\pi) - 2\sin(n\delta)\right]
    \end{align*}
    Note that for any integer of $\pi$, $\sin(n\pi) =0$, so this term can be thrown away. We can simplify then to
    \[\int F_N(x) \ dx = -\frac{1}{\pi(N+1)}\sum_{k = 0}^N\sum_{n = -k}^k \frac{1}{n}\sin(n\delta)\]
    Since the sine function is odd, and $1/n$ is also odd, then their product is even, so instead of the summation running from $-k$ to $k$, we can double the summation, and just run from $1$ to $k$, since the zero term has already been evaluated. We then have 
    \[\int F_N(x) \ dx = -\frac{2}{\pi(N+1)}\sum_{k = 0}^N\sum_{n = 1}^k \frac{1}{n}\sin(n\delta)\]
    Ignoring the terms at the front, the summation acts like 
    \[\sin(\delta) + \sin{\delta} + \frac{1}{2}\sin{\delta} + \sin{\delta} + \frac{1}{2}\sin{\delta} +\frac{1}{3}\sin{3\delta} + \cdots \]
    Grouping the terms together, we have that the double summation can be simplified to  
    \[\sum_{n = 0}^N \left( \frac{N}{n+1}  -1\right)\sin((n+1)\delta)\]
    I finally have that
    \[\int_{\delta \leq |x| \leq \pi} F_N(x) \ dx = 1 - \frac{\delta}{\pi} - \frac{2}{\pi(N+1)}\sum_{n = 0}^N \left( \frac{N}{n+1}  -1\right)\sin((n+1)\delta). \]
    It seems as though the summation doesn't converge as $N \into \infty$, since it spikes for $\delta$ sufficiently close to zero. I'm not entirely sure what's wrong, but I know that the Fej\'er kernel is an approximate identity. 
\end{solution}

\newpage
\section{Problem 8.1}
If $M$ is a linear subspace of a linear space $X$, then the \textit{quotient space} $X/M$ is the set $\{ x + M : x \in X\}$ of affine spaces
\[x + M = \{x + y : y \in M\}\]

\subsection{Problem 8.1, part a}
Show that $X/M$ is a linear space with respect to the operations 
\[\lm(x + M) = \lm x + M, \quad (x + M) + (y + M) = (x + y) + M.\]
\partbreak
\begin{solution}

    Fix $x \in X$, and let $y \in M$. Then since $M \subseteq X$ is a linear subspace of $X$, $x + y \in X$. Therefore, $\lm(x + y) = \lm x + \lm y \in X$. Since $M$ is a linear subspace, then $\lm y \in M$. Therefore, $\lm(x + M) = \lm x + M \in X/M$. \par
    
    \jump
    Similarly, fix $x , y \in X$. Then for the quotient space, there exists $u, v \in M$ for which $x+u, y+v$ are in the affine space. Since $M$ is a linear subspace of $X$, then $(x + u) + (y + v) = (x + y) + (u + v)$. Again, since $M$ is a linear subspace, $u + v \in M$, and $x + y \in X$, therefore, $(x + M) + (y + M) = (x + y) + M \in X/M$. Therefore, $X/M$ is a linear space.   
\end{solution}
\newpage
\subsection{Problem 8.1, part b}
Suppose that $X = M \oplus N$. Show that $N$ is linearly isomorphic to $X/M$. 
\partbreak
\begin{solution}

    Let $T: N \into X/M$ be defined as $T(y) = y + M$ for any $y \in N$. To show that this is a linear isomorphism, we need to show three things: $T$ is injective, $T$ is surjective, and $T$ is linear. To show injectivity, let $y_1, y_2 \in N$ be such that $Ty_1 = Ty_2$. This implies that $y_1 + M = y_2 + M$, and in particular, $y_1 \in y_2 + M$. Because of this, there exists an $x \in M$ such that $y_1 = y_2 + x$. Rearranging, we have $y_1 - y_2 = x$. Note that $N$ is linear, so $y_1 - y_2 \in N$. Since $M \cap N = \{0\}$, this implies that $x = 0$, therefore $y_1 = y_2$. Therefore, $T$ is injective. \par

    \jump
    To show surjectivity, let $z + M \in X / M$. Since $X = M \oplus N$, then $z = x + y$ for $x \in M, y \in N$. Since $x \in M$, and by the linearity of the quotient space shown above, we have that $(x + y) + M = (x + M) + (y + M) = y + M.$ Therefore, $z + M = y + M$, implying that $T$ is surjective. \par
    
    \jump
    Finally we show linearity of the operator $T$. By the work we did in part a, this is almost immediate, since for $\mu , \nu$ constants and $y_1, y_2 \in N$, we have
    \[T(\mu y_1 + \nu y_2) = (\mu y_1 + \nu y_2) + M = \mu (y_1 + M) + \nu (y_2 + M) = \mu T(y_1) + \nu T(y_2).\]
\end{solution}

\newpage
\subsection{Problem 8.1, part c}
The \textit{codimension} of $M$ in $X$ is the dimension of $X/M$. Is a subspace of a Banach space with finite codimension necessarily closed? 
\partbreak
\begin{solution}

    Suppose we have a projection $P$ which projects onto $N$ along $M$. By theorem 8.2, $X = \range P \oplus \ker P$, with $N = \range P$ and $M = \ker P$. In the last part, we showed that $N$ is linearly isomorphic to $X / M$, which we are supposing has finite dimension. Since there is an isomorphism between $\range P$ and $X / M$, then $\range P$ has finite dimension. Note that the projection $P$ is continuous as well, and is a linear mapping. Therefore, by Theorem 5.25, the kernel of $P$, $M$, is closed. So yes, $M$ is necessarily closed.  
\end{solution}

\newpage
\section{Problem 8.3}
Let $\scM, \scN$ be closed subspaces of a Hilbert space $\scH$ and $P, Q$ the orthogonal projections with range$(P) = \scM$, range$(Q) = \scN$. Prove that the following are equivalent:
\begin{enumerate}[a)]
    \item $\scM \subset \scN$;
    \item $QP = P$;
    \item $PQ = P$;
    \item $\norm{Px} \leq \norm{Qx}$ for all $x \in \scH$;
    \item $\braket{x}{Px} \leq \braket{x}{Qx}$ for all $x \in \scH$.
\end{enumerate}
\partbreak
\begin{solution}

    \begin{enumerate}
        \item[] \underline{$a) \implies b)$}:

        \hop
        Let $x \in \scH$. Since $P$ is a projection over a linear space, by Theorem 8.2, $\scH$ can be decomposed into a direct sum of the kernel and range of $P$. In particular, $\scH = \range (P) \oplus \ker (P)$. This implies any element of our Hilbert space can be uniquely decomposed into $x = z + y; \  z \in \range(P), \ y \in \ker (P)$. Since $\scM \subset \scN, \ \range(P) \subset \range(Q)$. This implies that $Q(z) = z$. Therefore, the following can be said:
        \[QP(x) = Q(P(z) + P(y)) = Q(z) + 0 = z.\]
        Note that $P(x) = P(z) + P(y) = z$. Therefore, the operations $QP$ and $P$ are equivalent. 

        \item[] \underline{$b) \implies a)$}:

        \hop
        If $QP = P$, then for any $z \in \range(P)$, 
        \[QP(z) = P(z) \implies Q(z) = z \implies z \in \range(Q).\]
        Therefore, $\range(P) \subset \range(Q)$, so $\scM \subset \scN$. Thus, we have shown the statement $(b)$ is equivalent to $(a)$.

        \newpage
        \item [] \underline{$a) \implies c)$}:

        \hop
        Let $x \in \scH$. By Theorem 8.2, $\scH = \range (Q) \oplus \ker (Q)$. Therefore, $x$ can be uniquely decomposed into $x = z + y; \ z \in \range (Q), \ y \in \ker(Q)$. Then, 
        \[PQ(x) = P(Q(z) + Q(y)) = P(z).\]
        If $\range(P) \subset \range(Q)$, then $\ker(Q) \subset \ker(P)$, since $\scH$ can be decomposed into direct sums according to the operator respectively. This implies that for $y \in \ker(Q), \ y \in \ker (P)$. Therefore, 
        \[P(x) = P(z + y) = P(z) + P(y) = P(z).\]
        Therefore, $PQ = P$.

        \item[] \underline{$c) \implies a)$}:

        \hop
        Take $x = z + y; \ z \in \range(Q), \ y \in \ker(Q)$. Since $PQ = P$, then the following can be said
        \[PQ(x) = P(x) \implies P(Qz + Qy) = P(z) + P(y) \implies Pz = Pz + Py \implies Py = 0.\]
        This implies that $\ker (Q) \subset \ker(P)$, which implies that $\range(P) \subset \range(Q)$, by the same logic form the previous implication. 
    
        \item[] \underline{$c) \implies d)$}:

        \hop
        This is the easiest way to show. Since $PQ = Q$, then the following can be said:
        \[\norm{Px} = \norm{QPx} \leq \norm{Q}\norm{Px} = \norm{Qx}\]
        The last inequality is via Proposition 8.4, that is, for a nonzero orthogonal projection, $\norm{Q} = 1$. 

        \item[] \underline{$d) \implies e)$}:

        \hop
        Since $\norm{Px} \leq \norm{Qx},$ then $\norm{Px}^2 \leq \norm{Qx}^2$. Then by the inner product norm definition, 
        \[\norm{Px}^2 = \braket{Px}{Px} = \braket{x}{P^2x} = \braket{x}{Px}\]
        Similarly, $\norm{Qx}^2 \leq \braket{x}{Qx}$. Therefore $\braket{x}{Px} \leq \braket{x}{Qx}$.

        \newpage
        \item[] \underline{$e) \implies a)$}:

        \hop
        Suppose false, that is, there exists an $x \in \scH$ where $x \in \scM$, but $x \not\in \scN$. We will assume that $(e)$ holds as well. Then $x \in \range (P), \ x\not \in \range(Q)$. Since $\scH = \range (Q) \oplus \ker (Q)$, this implies $x \in \ker(Q)$.  Therefore, 
        \[\braket{x}{Px} \leq \braket{x}{Qx} = 0 \implies \braket{x}{Px} \leq 0\]
        Since $P$ is an orthogonal projection, then 
        \[\braket{x}{Px} = \braket{x}{P^2x} = \braket{Px}{Px} = \norm{Px}^2 \geq 0.\]
        Since $\norm{Px}^2 \leq 0$ and $\norm{Px}^2 \geq 0$, then $\norm{Px}^2 = 0$, meaning $Px = 0$. Since we require $x \in \range(P)$, this implies that $x = 0$. But $0 \in \range(Q)$, since $\range(Q)$ is a linear vector subspace of $\scH$. Therefore, we have a contradiction. This implies that there does not exist an $x \in \scH$ such that $x \in \scM$, but $x \not \in \scN$. Therefore, $\scM \subset \scN$. 
    \end{enumerate}

    From the above implications, we have the following graph of implications. We can see that we can get to any node from any other node. 
    \usetikzlibrary {graphs}
    \usetikzlibrary{positioning, arrows.meta, bending}
    \renewcommand{\d}{2.5}
    \begin{center}
        \begin{tikzpicture}[
            node distance=2cm,
            arrow/.style={-Stealth, shorten >=1pt, shorten <=1pt},
            mynode/.style={draw, circle, fill=lightgray},
            edgelabel/.style={font=\footnotesize, sloped}
            ]
            
            % Nodes
            \node[mynode, label=$a$] at (0,0) (a) {};
            \node[mynode, label=$b$] at (\d,0) (b) {};
            \node[mynode, label=right:$c$] at (0,-\d) (c) {};
            \node[mynode, label=below:$d$] at (-\d, -\d) (d) {};
            \node[mynode, label=$e$] at (-\d,0) (e) {};
            
            % Edges
            %\draw[arrow] (s) -- node[edgelabel, above] {\textcolor{red}{6} (10)} (u);
            \draw[arrow] (a) -> (b);
            \draw[arrow] (b) -> (a);
            \draw[arrow] (a) -> (c);
            \draw[arrow] (c) -> (a);
            \draw[arrow] (c) -> (d);
            \draw[arrow] (d) -> (e);
            \draw[arrow] (e) -> (a);
        \end{tikzpicture}
    \end{center}

\end{solution}


\newpage
\section{Problem 8.5}
Let $\scH = L^2(\scT; \R^3)$ be the Hilbert space of $2\pi$-periodic, square integrable, vector-valued functions $\textbf{u}: \scT^3 \into \R^3$, with the inner product
\[\braket{\textbf{u}}{\textbf{v}} = \int_{\scT^3} \textbf{u}(\textbf{x}) \cdot \textbf{v}(\textbf{x}) \ d\textbf{x}.\]
We define the subspaces $\scV$ and $\scW$ of $\scH$ by 
\begin{align*}
    &\scV \  =  \ \{ \textbf{v} \in C^\infty (\scT^3;\R^3) : \grad \cdot \textbf{v} = 0\},\\
    &\scW  \ =  \ \{ \textbf{w} \in C^\infty (\scT^3;\R^3) : \textbf{w} = \grad\ph \text{ for some } \ph : \scT^3 \into \R\}.
\end{align*}
\subsection{Problem 8.5, part 1}
Show that $\scH = \scM \oplus \scN$ is the orthogonal direct sum of $\scM = \overline{\scV}$ and $\scN = \overline{\scW}$.
\partbreak
\begin{solution}

    We will first show that $\braket{\textbf{v}}{\textbf{w}} = 0$, for $\textbf{v} \in \scV$ and $\textbf{w} \in \scW$. Note first by integration by parts, we have
    \[\grad \cdot (\ph\textbf{v}) = \ph(\grad \cdot \textbf{v}) + \textbf{v}\cdot \grad\ph\]
    Since $\grad \cdot \textbf{v} = 0$, we can rewrite the inner product as 
    \[\braket{\textbf{v}}{\textbf{w}} = \int_{\scT^3} \grad \cdot(\ph \textbf{v}) \ d^3\textbf{x}\]
    By the Divergence Theorem of Vector calculus, 
    \[\int_{\scT^3} \grad \cdot(\ph \textbf{v}) \ d^3\textbf{x} = \int_{\partial \scT^3} \ph\textbf{v} \cdot d\textbf{S}\]
    where $\partial \scT^3$ is the boundary of our Torus, and $d\textbf{S}$ is the surface indicated by the normal vector to the torus at a point. Integration over any simple loop on our torus will return zero, since the torus is $2\pi$ periodic, and no external forces act upon it which returns a nonzero value. Therefore, 
    \[\int_{\partial \scT^3} \ph\textbf{v} \cdot d\textbf{S} = 0\]
    which implies $\scV \cap \scW = \{0\}$. Note that this implies $\scV \subseteq \scW^\perp, \ \scW \subseteq \scV^\perp.$ Via the fact that $\scH = \overline{\scW} \oplus \scW^\perp$, we just need to show that $\scW^\perp = \overline{\scV}$. Since $\overline{\scV}$ is the smallest closed subset which contains $\scV$ and $\scV \subseteq \scW^\perp$, $\overline{\scV} \subseteq \scW^\perp$. We next need to show that $\overline{\scV} \subseteq \scW^\perp$. \par
    
    \hop
    Suppose false, that is, there exists a $\textbf{u} \in \scW^\perp$ such that $u \not \in \overline{\scV}$, yet we have that $\braket{\textbf{u}}{\textbf{w}} = 0$ for any $\textbf{w} \in \scW$. Let $\textbf{w} \in \scW$. Therefore, there exists a $\ph : \scT^3 \into \R$ such that $\textbf{w} = \grad \ph$. Since $\braket{\textbf{u}}{\textbf{w}} = 0,$ then,
    \[\int \textbf{u} \cdot \grad \ph = 0.\]
    By integration by parts, we have that 
    \[\int \grad \cdot (\textbf{u}\ph) = \int \ph(\grad \cdot \textbf{u}).\]
    From the previous calculations, we found that the left integral equals zero, therefore, 
    \[\int \ph (\grad \cdot \textbf{u}) = 0.\]
    This should hold for any $\ph$, which implies $\grad \cdot u = 0$, giving us a contradiction. Therefore, $\scW^\perp = \overline{\scV}$, so $\scH = \overline{\scV} \oplus \overline{\scW}$.
\end{solution}

\newpage
\subsection{Problem 8.5, part 2}
\hop
Let $P$ be the orthogonal projection onto $\scM$. The velocity $\textbf{v}(\textbf{x}, t) \in \R^3$ and pressure $p(\textbf{x}, t) \in \R$ of an incompressible, viscous fluid satisfy the \textit{Navier-Stokes equations}
\vspace{-4mm}\begin{align*}
    &\textbf{v}_t + \textbf{v} \cdot \grad \textbf{v} + \grad p = \nu \Delta \textbf{v},\\
    &\grad \cdot \textbf{v} = 0.
\end{align*}\vspace{-10mm}

Show that the velocity \textbf{v} satisfies the nonlocal equation
\[\textbf{v}_t + P[\textbf{v}\cdot \grad \textbf{v}] = \nu \Delta \textbf{v}.\]
\vspace{-16mm}
\partbreak
\vspace{-3mm}
\begin{solution}

    Na\"ively applying the projection to the above equation, we have 
    \[P\left[ \frac{\partial \textbf{v}}{\partial t} + \textbf{v} \cdot \grad \textbf{v} + \grad p\right] = P\left[ \nu \Delta\textbf{v}\right].\]
    Clearly, $P$ is a linear operator, as well as $\grad p \in \overline{\scW}$. We can then simplify to,
    \[P\left[ \frac{\partial \textbf{v}}{\partial t}\right] + P\left[\textbf{v} \cdot \grad \textbf{v} \right] = P\left[ \nu \Delta\textbf{v}\right].\]
    For this equation to be equal to the given equation, we need to show that $\textbf{v}_t, \ \nu\Delta \textbf{v} \in \overline{\scV}$. Since the spatial differential operator acts independently on the temporal differential operator, we have 
    \[\grad \cdot (\textbf{v}_t) = \grad \cdot \frac{\partial \textbf{v}}{\partial t} = \frac{\partial}{\partial t}(\grad \cdot \textbf{v}) = 0. \quad \implies \frac{\partial \textbf{v}}{\partial t} \in \overline{\scV}.\]
    Furthermore, by the definition of the Laplacian operator\footnote{I am most comfortable with denoting the Laplacian operator as $\grad^2$ over $\Delta$.}, we have that 
    \[\grad^2 \textbf{v} = \grad (\grad \cdot \textbf{v}) - \grad \cross (\grad \cross \textbf{v}).\]
    Taking the divergence of a curl is equal to zero, plugging that in, we have
    \[\grad \cdot \grad^2\textbf{v} = \grad \cdot \grad (\grad \cdot \textbf{v}) = \grad^2(\grad \cdot \textbf{v}).\]
    Therefore, the divergence operator commutes with the Laplacian. Since $\grad \cdot \textbf{v} = 0$, this implies that $\nu \grad^2 \textbf{v} \in \overline{\scV}$. Therefore, the projection $P$ will do nothing to these two terms. The equation then simplifies to 
    \[\frac{\partial \textbf{v}}{\partial t} + P\left[ \textbf{v}\cdot \grad \textbf{v}\right] = \nu \grad^2 \textbf{v},\]
    which is what we wanted to find.
\end{solution}


\newpage
\section{Problem 8.7}
If $\ph_y$ is the bounded linear functional defined as $\ph_y (x) = \braket{y}{x}$, prove that $\norm{\ph_y} = \norm{y}$.
\partbreak
\begin{solution}
    We will first begin with the definition of a linear operator. Let $x$ be chosen nonzero, then 
    \[\norm{\ph_y(x)} = \sup\frac{|\ph_y(x)|}{\norm{x}} = \sup\frac{|\braket{y}{x}|}{\norm{x}}.\]
    Via Cauchy Schwartz, the numerator is bounded by $\norm{y}\norm{x}$. Then
    \[\norm{\ph_y(x)} \leq \sup\frac{\norm{x}\norm{y}}{\norm{x}} = \sup\norm{y} = \norm{y}.\]
    Next, we need to find a value in $\scH$ which saturates this bound. Substituting in $x \mapsto y$, we see
    \[\frac{|\ph_y(y)|}{\norm{y}} = \frac{\norm{y}^2}{\norm{y}} = \norm{y}.\]
    Therefore, the bound can be achieved, implying $\norm{\ph_y} = \norm{y}$. Note if $x = 0$, then clearly, $\ph_y(x) = 0$. This value will be bounded by $\norm{y}$, since $\norm{y}\geq 0$. The same logic of setting $y = 0$ shows this bound can be achieved.  
\end{solution}

\newpage
\section{Problem 8.9, part 1}
Let $A \subset \scH$ be such that
\[\scM = \{ x \in \scH : x \text{ is a finite linear combinations of elements in $A$}\}\]
is a dense linear subspace of $\scH$. Prove that any bounded linear functional on $\scH$ is uniquely determined by its values on $A$. 
\partbreak
\begin{solution}

    Since any element in $\scM$ can be expressed as a finite linear combination of elements in $A$, take $U$ to be the maximally linear independent set of vectors in $A$. Then $U$ spans $A$, as well as $\scM$. We then have that any $x \in \scM$ can be defined uniquely by a finite linear combination of elements in the basis set $U$. Define $\psi$ as a linear functional which acts on $\scM$ where $\norm{\psi} = M$ is bounded. Since every elements of $\scM$ requires finite combinations of elements in our basis set, $\scM$ has finite dimension. Therefore, by Hahn Banach, we can extend our functional $\psi$ to a bounded linear functional $\ph : \scH \into \R$ which acts on the entire Hilbert Space. Since $\ph$ is a bounded linear functional on $\scH$, the by the Riesz Representation theorem, Theorem 8.12, there is a unique vector $y \in \scH$ such that $\ph(x) = \braket{y}{x}$. Since $\scM$ is dense in $\scH$, $y$ can be taken as a limit of a sequence of elements in $\scM$, which are uniquely determined by a linear combination of elements in $U$. Thus the claim has been shown. 

\end{solution}

\newpage
\subsection{Problem 8.9, part 2}
If $\{ u_\alpha\}$ is an orthonormal basis, find a necessary and sufficient condition on a family of complex numbers $c_\alpha$ for there to be a bounded linear functional $\ph$ such that $\ph(u_\alpha) = c_\alpha$.
\partbreak
\begin{solution}

    I claim that a necessary and sufficient condition for there to exists such a bounded linear functional is for $\{ c_\alpha \} \in \ell^2$, that is $\sum_{\alpha \in I}|c_\alpha|^2 < \infty$ for $I$ being an arbitrary index set. We will break this up into two cases, based on the size of the index set.

    \begin{itemize}
        \item \underline{$I$ is countable}:

        \hop
        Since $\{u_\alpha\}$ is an orthonormal basis, then by Parseval's equality, for any $x \in \scH$, $x = \sum_{\alpha \in I} \braket{u_\alpha}{x}u_\alpha$. For sufficienly, we suppose $\sum_{\alpha \in I} |c_\alpha|^2 < \infty$. Define $c = (\alpha_1, \dots, \alpha_n, \dots)$ to be the vector whose $\alpha$-th element is $c_{\alpha}$ in the basis given. Then,
        \[\ph_{c}(u_\alpha) = \braket{c}{u_\alpha} = c_\alpha = \ph(u_\alpha)\]
        Therefore, by Problem 8.7, $\norm{\ph} = \norm{\ph_c} = \norm{c} < \infty$ by assumption. therefore, $\ph$ is bounded.  \par
        
        \jump
        For necessity, we suppose that $\norm{\ph} < \infty.$ Define $c_n = (c_1, \dots, c_n, 0, \dots 0)$ to have nonzero entries in only the first $n$ entries (with respect to the given orthonormal basis). Then,
        \[\norm{c_n}^2 = \sum_{\alpha \in I} |\braket{u_\alpha}{c_\alpha}|^2 = \sum_{\alpha =1}^n |{c_\alpha}|^2 = \ph(c_n) \leq \norm{\ph}\norm{c_n}.\]
        Dividing by $\norm{c_n}$ on both sides, this implies $\norm{c_n} \leq \norm{\phi}$. Since this should hold for all $n$, we see that $\sum_{\alpha \in I} |c_{\alpha}|^2 < \infty,$ satisfying necessity. 

        \item \underline{$I$ is uncountable}:

        \hop
        The same analysis can be done as in the case when $I$ is countable. By the previous part, we can take $A$ to be the span of $\{u_\alpha\}$, for $\alpha$ in some countable index set. Then we see the same necessary and sufficient condition to hold as in the countable case. 
    \end{itemize}
\end{solution}
\end{document}