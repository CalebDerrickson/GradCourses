\documentclass[12pt]{article}
\usepackage[paper=letterpaper,margin=1.5cm]{geometry}
\usepackage{amsmath}
\usepackage{amssymb}
\usepackage{amsfonts}
\usepackage{mathtools}
%\usepackage[utf8]{inputenc}
%\usepackage{newtxtext, newtxmath}
\usepackage{lmodern}     % set math font to Latin modern math
\usepackage[T1]{fontenc}
\renewcommand\rmdefault{ptm}
%\usepackage{enumitem}
\usepackage[shortlabels]{enumitem}
\usepackage{titling}
\usepackage{graphicx}
\usepackage[colorlinks=true]{hyperref}
\usepackage{setspace}
\usepackage{subfigure} 
\usepackage{braket}
\usepackage{color}
\usepackage{tabularx}
\usepackage[table]{xcolor}
\usepackage{listings}
\usepackage{mathrsfs}
\usepackage{stackengine}
\usepackage{physics}
\usepackage{afterpage}
\usepackage{pdfpages}
\usepackage[export]{adjustbox}
\usepackage{biblatex}

\setstackEOL{\\}

\definecolor{dkgreen}{rgb}{0,0.6,0}
\definecolor{gray}{rgb}{0.5,0.5,0.5}
\definecolor{mauve}{rgb}{0.58,0,0.82}


\lstset{frame=tb,
  language=Python,
  aboveskip=3mm,
  belowskip=3mm,
  showstringspaces=false,
  columns=flexible,
  basicstyle={\small\ttfamily},
  numbers=none,
  numberstyle=\tiny\color{gray},
  keywordstyle=\color{blue},
  commentstyle=\color{dkgreen},
  stringstyle=\color{mauve},
  breaklines=true,
  breakatwhitespace=true,
  tabsize=3
}
\setlength{\droptitle}{-6em}

\makeatletter
% we use \prefix@<level> only if it is defined
\renewcommand{\@seccntformat}[1]{%
  \ifcsname prefix@#1\endcsname
    \csname prefix@#1\endcsname
  \else
    \csname the#1\endcsname\quad
  \fi}
% define \prefix@section
\newcommand\prefix@section{}
\newcommand{\prefix@subsection}{}
\newcommand{\prefix@subsubsection}{}
\renewcommand{\thesubsection}{\arabic{subsection}}
\makeatother
\DeclareMathOperator*{\argmin}{argmin}
\newcommand{\partbreak}{\begin{center}\rule{17.5cm}{2pt}\end{center}}
\newcommand{\alignbreak}{\begin{center}\rule{15cm}{1pt}\end{center}}
\newcommand{\tightalignbreak}{\vspace{-5mm}\alignbreak\vspace{-5mm}}
\newcommand{\hop}{\vspace{1mm}}
\newcommand{\jump}{\vspace{5mm}}
\newcommand{\R}{\mathbb{R}}
\newcommand{\C}{\mathbb{C}}
\newcommand{\N}{\mathbb{N}}
\newcommand{\G}{\mathbb{G}}
\renewcommand{\S}{\mathbb{S}}
\newcommand{\bt}{\textbf}
\newcommand{\xdot}{\dot{x}}
\renewcommand{\star}{^{*}}
\newcommand{\ydot}{\dot{y}}
\newcommand{\lm}{\mathrm{\lambda}}
\renewcommand{\th}{\theta}
\newcommand{\id}{\mathbb{I}}
\newcommand{\si}{\Sigma}
\newcommand{\Si}{\si}
\newcommand{\inv}{^{-1}}
\newcommand{\T}{^\intercal}
\renewcommand{\tr}{\text{tr}}
\newcommand{\ep}{\varepsilon}
\newcommand{\ph}{\varphi}
%\renewcomand{\norm}[1]{\left\lVert#1\right\rVert}
\definecolor{cit}{rgb}{0.05,0.2,0.45}
\addtolength{\jot}{1em}
\newcommand{\solution}[1]{

\noindent{\color{cit}\textbf{Solution:} #1}}

\newcounter{tmpctr}
\newcommand\fancyRoman[1]{%
  \setcounter{tmpctr}{#1}%
  \setbox0=\hbox{\kern0.3pt\textsf{\Roman{tmpctr}}}%
  \setstackgap{S}{-.9pt}%
  \Shortstack{\rule{\dimexpr\wd0+.1ex}{.9pt}\\\copy0\\
              \rule{\dimexpr\wd0+.1ex}{.9pt}}%
}

\newcommand{\Id}{\fancyRoman{2}}

% Enter the specific assignment number and topic of that assignment below, and replace "Your Name" with your actual name.
\title{STAT 31210: Homework 3}
\author{Caleb Derrickson}
\date{January 26, 2024}

\begin{document}
\onehalfspacing
\maketitle
\allowdisplaybreaks
{\color{cit}\vspace{2mm}\noindent\textbf{Collaborators:}} The TA's of the class, as well as Kevin Hefner, and Alexander Cram.

\tableofcontents

\newpage
\section{Problem 5.3}
Let $\delta: C([0, 1]) \rightarrow \R$ be the linear functional that evaluated an function at the origin: $\delta(f) = f(0)$. If $C([0, 1])$ is equipped with the sup-norm,
\[\norm{f}_1 = \int_0^1 |f(x)| \ dx,\]
show that $\delta$ is unbounded.
\partbreak
\begin{solution}
    
    Note that $\delta f \in \R$, so $|\delta f| = |f(0)|$. By definition, $|f(0)| \leq \sup_{x \in [0, 1]} |f(x)| = \norm{f}_\infty$. Then 
    \[
   \norm{\delta} = \sup \frac{\norm{\delta f}}{\norm{f}_\infty} = \sup \frac{|f(0)|}{\norm{f}_\infty} = \sup \frac{|f(0)|}{\sup |f(x)|} \leq 1
    \]
    This implies that $\delta$ is bounded. To compute the norm, we just need to find a function that achieves its max value at $x = 0$. The simplest case is to take $f$ be a nonzero constant function, i.e. $f(x) = 2$ for all $x \in [0, 1]$. Then $|f(0)| = 2$, and $\sup |f(x)| = |f(0)| = 2$. Then $\norm{\delta} = 1$.
    \par
    
    \jump
    To show that when $C([0, 1])$ equipped with the one-norm makes $\norm{\delta}$ unbounded, we can consider the family of functions 
    \[\mathbb{O} = \{ \{1 - nx : x \in \left[0, \frac{1}{n}\right], \text{else } 0\}, n \in \N\}.\]
    Taking a memeber of that family, we can note that $\norm{\delta f} = |f(0)| = 1$, and
    \[\norm{f}_1 = \int_0^1 |f(x)| = \int_0^\frac{1}{n} 1 - nx \ dx = \frac{1}{2n}\]
    This implies $\norm{\delta} = \sup \{2n\} = \infty$, which is unbounded. 
\end{solution}

\newpage
\section{Problem 5.7}
Find the kernel and range of the linear operator $K: C([0, 1]) \rightarrow C([0, 1])$ defined by 
\[ Kf(x) = \int_0^1 \sin\left( \pi (x - y)\right)f(y) \ dy.\]
\partbreak
\begin{solution}

    In finding the kernel of this functional, we need to find all functions such that
    \[-\int_0^1 e^{-\pi(x - y)} f(y) \ dy = \int_0^1 e^{-i\pi(x - y)}f(y) \ dy\]
    As a first suggestion, since the transformation $y \rightarrow -y$ in the right integral transforms it to the first integral with the upper bound being -1, we can first try an even function on $[0, 1]$. As a first guess, we can try $f(y) = \cos(\pi y)$. But when checking this, we see that 
    \[\int_0^1 \sin \left( \pi(x - y)\right)\cos(\pi y) \ dy = \frac{1}{2}\sin(\pi x)\]
    This is obviously not zero. the kernel is a functions orthogonal to cosine (pi x) and sine (pi x).
\end{solution}

\newpage
\section{Problem 5.8}

\newpage
\section{Problem 5.14}

\newpage
\section{Problem 5.15}

\end{document}