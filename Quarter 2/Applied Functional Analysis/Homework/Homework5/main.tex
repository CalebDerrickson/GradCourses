\documentclass[12pt]{article}
\usepackage[paper=letterpaper,margin=1.5cm]{geometry}
\usepackage{amsmath}
\usepackage{amssymb}
\usepackage{amsfonts}
\usepackage{mathtools}
%\usepackage[utf8]{inputenc}
%\usepackage{newtxtext, newtxmath}
\usepackage{lmodern}     % set math font to Latin modern math
\usepackage[T1]{fontenc}
\renewcommand\rmdefault{ptm}
%\usepackage{enumitem}
\usepackage[shortlabels]{enumitem}
\usepackage{titling}
\usepackage{graphicx}
\usepackage[colorlinks=true]{hyperref}
\usepackage{setspace}
\usepackage{subfigure} 
\usepackage{braket}
\usepackage{color}
\usepackage{tabularx}
\usepackage[table]{xcolor}
\usepackage{listings}
\usepackage{mathrsfs}
\usepackage{stackengine}
\usepackage{physics}
\usepackage{afterpage}
\usepackage{pdfpages}
\usepackage[export]{adjustbox}
\usepackage{biblatex}

\setstackEOL{\\}

\definecolor{dkgreen}{rgb}{0,0.6,0}
\definecolor{gray}{rgb}{0.5,0.5,0.5}
\definecolor{mauve}{rgb}{0.58,0,0.82}


\lstset{frame=tb,
  language=Python,
  aboveskip=3mm,
  belowskip=3mm,
  showstringspaces=false,
  columns=flexible,
  basicstyle={\small\ttfamily},
  numbers=none,
  numberstyle=\tiny\color{gray},
  keywordstyle=\color{blue},
  commentstyle=\color{dkgreen},
  stringstyle=\color{mauve},
  breaklines=true,
  breakatwhitespace=true,
  tabsize=3
}
\setlength{\droptitle}{-6em}

\makeatletter
% we use \prefix@<level> only if it is defined
\renewcommand{\@seccntformat}[1]{%
  \ifcsname prefix@#1\endcsname
    \csname prefix@#1\endcsname
  \else
    \csname the#1\endcsname\quad
  \fi}
% define \prefix@section
\newcommand\prefix@section{}
\newcommand{\prefix@subsection}{}
\newcommand{\prefix@subsubsection}{}
\renewcommand{\thesubsection}{\arabic{subsection}}
\makeatother
\DeclareMathOperator*{\argmin}{argmin}
\newcommand{\partbreak}{\begin{center}\rule{17.5cm}{2pt}\end{center}}
\newcommand{\alignbreak}{\begin{center}\rule{15cm}{1pt}\end{center}}
\newcommand{\tightalignbreak}{\vspace{-5mm}\alignbreak\vspace{-5mm}}
\newcommand{\hop}{\vspace{1mm}}
\newcommand{\jump}{\vspace{5mm}}
\newcommand{\R}{\mathbb{R}}
\newcommand{\C}{\mathbb{C}}
\newcommand{\N}{\mathbb{N}}
\newcommand{\G}{\mathbb{G}}
\renewcommand{\S}{\mathbb{S}}
\newcommand{\bt}{\textbf}
\newcommand{\xdot}{\dot{x}}
\renewcommand{\star}{^{*}}
\newcommand{\ydot}{\dot{y}}
\newcommand{\lm}{\mathrm{\lambda}}
\renewcommand{\th}{\theta}
\newcommand{\id}{\mathbb{I}}
\newcommand{\si}{\Sigma}
\newcommand{\Si}{\si}
\newcommand{\inv}{^{-1}}
\newcommand{\T}{^\intercal}
\renewcommand{\tr}{\text{tr}}
\newcommand{\ep}{\varepsilon}
\newcommand{\ph}{\varphi}
%\renewcomand{\norm}[1]{\left\lVert#1\right\rVert}
\definecolor{cit}{rgb}{0.05,0.2,0.45}
\addtolength{\jot}{1em}
\newcommand{\solution}[1]{

\noindent{\color{cit}\textbf{Solution:} #1}}

\newcounter{tmpctr}
\newcommand\fancyRoman[1]{%
  \setcounter{tmpctr}{#1}%
  \setbox0=\hbox{\kern0.3pt\textsf{\Roman{tmpctr}}}%
  \setstackgap{S}{-.9pt}%
  \Shortstack{\rule{\dimexpr\wd0+.1ex}{.9pt}\\\copy0\\
              \rule{\dimexpr\wd0+.1ex}{.9pt}}%
}

\newcommand{\Id}{\fancyRoman{2}}

% Enter the specific assignment number and topic of that assignment below, and replace "Your Name" with your actual name.
\title{STAT 31210: Homework 5}
\author{Caleb Derrickson}
\date{February 8, 2024}

\begin{document}
\onehalfspacing
\maketitle
\allowdisplaybreaks
{\color{cit}\vspace{2mm}\noindent\textbf{Collaborators:}} The TA's of the class, as well as Kevin Hefner, and Alexander Cram.

\tableofcontents

\newpage
\section{Problem 12.6}
Use the Dominated Convergence Theorem to prove Corollary 12.36 for differentiation under an integral sign. 
\partbreak
\begin{solution}

    I will include the Dominated Convergence Theorem and Corollary 12.36 for reference.

    \begin{center}\rule{16.5cm}{0.5pt}\end{center}
    \begin{quote}
    \vspace{-6mm}
        \textbf{Dominated Convergence Theorem}: Suppose that $(f_n)$ is a sequence of integrable functions, $f_n: X \into \overline{\R}$, on a measure space $(X, A, \mu)$ that converges pointwise to a limiting function $f: X \into \overline{\R}$. If there is an integrable function $g: X \into [0, \infty]$ such that
        \[|f_n(x)| \leq g(x) \quad \text{for all $x \in X$ and $n \in \N$},\]
        then $f$ is integrable and 
        \[\lim_{n \into \infty} \int f_n \ d\mu = \int f \ d\mu.\]
    \end{quote}
    \vspace{-10mm}
    \begin{center}\rule{16.5cm}{0.5pt}\end{center}
    \jump

    \begin{center}\rule{16.5cm}{0.5pt}\end{center}
    \begin{quote}
    \vspace{-6mm}
        \textbf{Corollary 12.36}: Suppose that $(X, A , \mu)$ is a complete measure space, $I \subset \R$ is an open interval, and $f : X \times I \into \overline{\R}$ is a measurable function such that:
        \begin{itemize}
            \item $f(\cdot, t)$ is integrable on $X$ for each $t \in I$;
            \item $f(x, \cdot)$ is differentiable in $I$ for each $x \in X \setminus N$, where $\mu(N) = 0$;
            \item there is an integrable function $g: X \into [0, \infty]$ such that 
            \[ \left| \frac{\partial f}{\partial t}(x, t) \right| \leq g(x) \quad \text{a.e. in $X$ for every $t \in I$.}\]
            Then
            \[\ph (t) = \int_X f(x, t) \ d\mu(x)\]
            is a differentiable function of $t$ in $I$, and 
            \[\frac{d\ph}{dt} (t) = \int_X \frac{\partial f}{\partial t}(x, t) \ d\mu(x).\]
        \end{itemize}
        
        \end{quote}
    \vspace{-10mm}
    \begin{center}\rule{16.5cm}{0.5pt}\end{center}

    \newpage
    Suppose we have a sequence of functions $d_n : X \times I \into \overline{\R}$ defined as
    \[d_n(x, t) = \frac{f(x, t + \frac{1}{n}) - f(x, t)}{1 / n}\]
    Note that this is the $n$-th approximation to $\frac{\partial f}{\partial t}$. Since for some $n$ sufficiently large, there exists some $\ep > 0$ for which $|d_n - \frac{\partial f}{\partial t}| < \ep$. By the reverse triangle inequality, we see that 
    \[|d_n - \frac{\partial f}{\partial t}| < \ep \implies |d_n| < |\frac{\partial f}{\partial t}| + \ep.\]
    Since the partial derivative is bounded by some $g: X \into [0, \infty]$, we have that $|d_n| \leq g(x)$, since $\ep$ is arbitrary. Therefore, we have 
    \[\lim_{n \into \infty}\int_X d_n \ d\mu(x) = \int_X \lim_{x \into \infty} d_n \ d\mu(x) = \int_X \frac{\partial f}{\partial t}(x, t) \ d\mu(x).\]
    Note that $d_n$ is differentiable in $I$ for all $n$, thus the derivative is defined. This then proves Corollary 12.36. 
\end{solution}


\newpage
\section{Problem 12.8}
Let $f_n : X \into \C$. be a sequence of measurable functions converging to $f$ pointwise almost everywhere. Suppose there exists $g \in L^p(X)$ such that $|f_n| \leq g$ almost everywhere. Then $f_n \into f$ in the $L^p$-norm. 
\partbreak
\begin{solution}

    We can first note that $\lim_{n \into \infty} |f_n(x)| = |f(x)| \leq |g(x)|$. This implies that $|f_n(x)| \leq |g(x)|$, so we can rewrite the convergence of $f_n$ as
    \[|f_n - f|^p \leq (|f_n| + |f|)^p \leq (|g| + |g|)^p = 2^p|g|^p.\]
    Since $g \in L^p$, we have that $\int |g|^p < \infty$, therefore, Theorem 12.35 applies. Therefore, we can write
    \[\lim_{n \into \infty} \norm{f_n - f}_p^p = \lim_{n \into \infty} \int |f_n - f|^p \ d\mu = \int \lim_{n \into \infty} |f_n - f|^p = 0.\]
    Therefore, $f_n \into f$ in the $L^p$ norm.
\end{solution}


\newpage
\section{Problem 12.12}
Prove the following generalization of H\"older's inequality: if $1 \leq p_i \leq \infty$, where $i = 1, ..., n$ satisfy
\[\sum_{i = 1}^n \frac{1}{p_i} = 1\]
and $f_i \in L^{p_i}(X, \mu)$, then $f_1 \cdots f_n \in L^1(X, \mu)$ and 
\[\left| \int f_1 \cdots f_n \ d\mu\right| \leq \norm{f_1}_{p_1} \cdots \norm{f_n}_{p_n}.\]
\partbreak
\begin{solution}

    This claim will be proven via induction on $n$. 
    \begin{itemize}
        \item \underline{Base Case}: $n = 1$

        \hop
        Then the summation runs only over one $p_i$, in particular $p_1$. Since $\frac{1}{p_1} = 1$, this implies that $p_1 = 1$. Then 
        \[\left| \int f_1 \ d\mu\right| = \norm{f_1}_1 \leq \norm{f_1}_1\]
        Since equality is a subcase of ``$\leq$".

        \item \underline{Induction Hypothesis}: 
        
        \hop
        Next we assume the inequality above holds for some cases up to and including the case $k$, $k > 1$. This implies that 
        \[\sum_{i = 1}^k \frac{1}{p_i} = 1,\]
        and
        \[\left| \int f_1 \cdots f_k \ d\mu \right| \leq \norm{f_1}_{p_1} \cdots \norm{f_k}_{p_k}.\]
        We will now show that the subsequent case, $k+1$, holds.

        \item \underline{Induction Case}: $n = k+1$.

        \hop
        Since each $1 \leq p_i \leq \infty$, we need to consider the case when $p_{k+1} = \infty$. If this is true, then 
        \[\sum_{i = 1}^{k+1}\frac{1}{p_i} = \sum_{i = 1}^{k}\frac{1}{p_i} = 1.\]
        Thus, H\"older's inequality can be applied, where we separate the function $f_{k+1}$ from the first $k$ functions. Then, 
        \[\left| \int f_1 \cdots f_{k+1} \ d\mu \right| \leq \norm{f_1\cdots f_{k}}_1 \norm{f_{k+1}}_\infty.\]
        Then, we can apply the induction hypothesis to get 
        \[\left| \int f_1 \cdots f_{k+1} \ d\mu \right| \leq \norm{f_1}_{p_1}\cdots \norm{f_k}_{p_k} \norm{f_{k+1}}_\infty\]
        Which is what we wanted to show. \par

        \hop
        We next assume that $p_{k+1} \neq \infty$. Define the following H\"older conjugates:
        \[p := \frac{p_{k+1}}{p_{k+1} - 1}, \quad q := p_{k+1}\]
        Applying H\"older's Inequality in a similar fashion to the previous case gives, 
        \[\norm{f_1\cdots f_{k}f_{k+1}}_1 \leq \norm{f_1\cdots f_k}_p \norm{f_{k+1}}_q\]
        Which by the Induction Hypothesis, satisfies the given equality. Next, we need to show that the summation is satisfied. From the definition of $p, q$, we have
        \[\sum_{i =1}^{k+1}\frac{1}{p_i} = \frac{1}{p} + \frac{1}{q} = 1 - \frac{1}{p_{k+1}} + \frac{1}{p_{k+1}} = 1\]
    \end{itemize}
\end{solution}

\newpage
\section{Problem 12.15}
If $f \in L^p (\R^n) \cap L^q(\R^n)$, where $p < q$, prove that $f \in L^r(\R^n)$ for any $p < r < q$, and show that 
\[\norm{f}_r \leq (\norm{f}_p)^{\frac{1 / r - 1 / q}{1 / p - 1 / q}} (\norm{f}_q)^{\frac{1 / p - 1 / r}{1 / p - 1 / q}},\]
This result is one of the simplest examples of an \textit{interpolation inequality}.
\partbreak
\begin{solution}

    Suppose there are constants $\alpha \in (0, 1)$, and $m, n > 1$ be conjugates. By H\"older's Inequality, we then have 
    \[\norm{f}_r^r = \int |f|^r = \int |f|^{\alpha r} |f|^{(1 - \alpha)r} \leq \left( \int |f|^{\alpha m r}\right)^{1 / m} \left( \int |f|^{(1 - \alpha)nr}\right)^{1 / n} \]

    We wish to have the right hand side to have the form given, that is, we need to choose $\alpha, m , n$ such that
    \begin{center}
        \begin{align}
            \alpha m r &= p\\
            (1 - \alpha)nr &= q\\
            \frac{1}{m} + \frac{1}{n} &= 1        
        \end{align}
    \end{center}
    We will first solve for $\alpha$. Note that (1) implies that $\alpha = \frac{p}{mr}$. (2) implies $\frac{1}{n} = \frac{(1 - \alpha)r}{q}$, and (3) implies $\frac{1}{m} = 1 - \frac{1}{n}$. Plugging this all in, we have that 
    \[\alpha = \frac{p}{r}\left(1 - \frac{(1 - \alpha)r}{q}\right).\]
    We can then solve for $\alpha$.

    \tightalignbreak
    \begin{align*}
        &\alpha = \frac{p}{r}\left(1 - \frac{(1 - \alpha)r}{q}\right)&\text{(Given.)}\\
        &\alpha = \frac{p}{r}\left( 1 - \frac{r}{q}\right) + \frac{\alpha p}{q} &\text{(Distributing.)}\\
        &\alpha\left( 1 - \frac{p}{q} \right) = \frac{p}{r}\left(1 - \frac{r}{q}\right) &\text{(Rearranging.)}\\
        \alpha &= \left( \frac{pq - rp}{rq}\right)\left( \frac{q - p}{q}\right)\inv &\text{(Rearranging.)}\\
        &= \left( \frac{pq - rp}{rq}\right)\left( \frac{q}{q - p}\right) &\text{(Taking inverse.)}\\
        &= \frac{pq - rp}{rq - rp} &\text{(Simplifying.)}\\
        &= \frac{1/r - 1/q}{1/p-1/q} &\text{(Dividing by $rpq$.)}
    \end{align*}
    \vspace{-12mm}\alignbreak
    From the original equation, we now have that 
    \[\norm{f}_r \leq \left((\norm{f}^p_p)^{1/m} (\norm{f}_q^q)^{1/n}\right)^{1/r} = \norm{f}_p^{p/mr}\norm{f}_q^{q/nr}.\]
    From (1) and (2), we can substitute the powers by $\alpha$ and $1 - \alpha$, respectively. Note that
    \[1 - \alpha = 1 - \frac{1/r-1/q}{1/p - 1/q} = \frac{1/p-1/q}{1/p-1/q} - \frac{1/r-1/q}{1/p - 1/q} = \frac{1/p - 1/r}{1/p - 1/q}.\]
    We therefore have the desired inequality. 
\end{solution}

\newpage
\section{Problem 12.17}
Prove that the unit ball in $L^p([0, 1])$, where $1 \leq p \leq \infty$, is not strongly compact.
\partbreak
\begin{solution}

    As a counterexample, we let $n \in \N$, so that $I_n = (2^{-n}, 2^{-(n-1)})$, and $f_n = 2^{n/p}\chi_{I_n}.$ Note that 
    \[\norm{f_n}_p = \int_0^1 |2^{n/p}\chi_{I_n}(x)| \ d\mu(x) = \int^{2^{-n+1}}_{2^{-n}} 2^n \ d\mu(x) = 2^n(2^{-n+1} - 2^{-n}) = 2 - 1 = 1.\]
    Similarly, we have that $\norm{f_n - f_m}\infty = 1$. For $m \neq n$. For $p \in [1, \infty)$, we have that
    \[\norm{f_n - f_m}_p = \left(2^n \chi_{I_n}(x) - 2^m \chi_{I_m}(x) \ d\mu(x)\right)^{1/p} = (1 + 1)^{1/p} = 2^{1/p}.\]
    Note that the integral above can be broken into two integrals, which are both of the previous integral. Since we have the difference under the sup-norm is 1, we cannot take infinitely large $p$ to maintain the p-norm less than $\ep$. Therefore, no subsequence of $(f_n)$ can be Cauchy, so none can converge. 
\end{solution}


\newpage
\section{Problem 12.18}
Give an example of a bounded sequence in $L^1([0, 1])$ that does not have a weakly convergent subsequence. Why does this not contract the Banach-Analoglu Theorem?
\partbreak
\begin{solution}

    The choice of $f_n$ is similar to Problem 2.17, with $f_n = 2^n \chi_{I_n}$. Suppose that $(f_{n_j})$ is a subsequence of $(f_n)$, and define a function $g \in L^\infty$ by 
    \[g = \sum_{j = 1}^\infty (-1)^j \chi_{I_{n_j}}.\]
    Take $\ph \in (L^1)^*$ as $\ph (f) = \int fg.$ Then,
    \[\int_0^1 2^{n_j} \chi_{I_{n_j}}(x) \sum_{k = 1}^\infty (-1)^k\chi_{I_{n_k}}(x) \ d\mu(x) = \int_{2^{-n_j}}^{2^{-n_j + 1}} 2^{n_j} (-1)^j \ d\mu(x) = 2^{n_j}(2^{-n_j} - 2^{-n_j + 1})(-1)^j = (-1)^j. \]
    Note that when taking the sum, the only term that will survive with respect to the outside indicator function is the one related to $n_j$. Therefore, $(f_{n_j})$ cannot converge weakly. Note that this does not contradict the Banach-Alaoglu Theorem since $L^1$ is not reflexive.
\end{solution}

\newpage
\section{Problem 6.2}
Consider $C([0, 1])$ with the sup-norm. Let 
\[N = \left \{ f \in C([0, 1]) : \int_0^1 f(x) \ dx = 0\right\}\]
be the closed linear subspace of $C([0, 1])$ of functions with zero mean. Let 
\[X = \left\{ f \in C([0, 1]) : f(0) = 0\right\}\]
and define $M = N \cap X$. 
\subsection{Problem 6.2, part a}
If $u \in C([0, 1])$, prove that 
\[d(u, N) = \inf_{n \in N} \norm{u - n} = |\overline{u}|\]
where $|\overline{u}| = \int_0^1 u(x) \ dx$ is the mean of $u$, so the infimum is attained when $n = u - \overline{u} \in N$. 
\partbreak
\begin{solution}

    We first consider two cases, where $u \in C([0, 1])\setminus N, u \in N$. If $u \in N$, then $d(u, N) = \inf_{n \in N} \norm{u - n} = 0$, since the norm is positive function, and equals zero only when the term inside is equal zero. This implies $n = u$ is the unique element. If $u \in C([0, 1]) \setminus N$, we then have by Theorem 6.13 that there us a unique closest element for $u$ in $N$, denoted $y$, such that 
    \[\norm{u - y} = \min_{z \in N} \norm{u - z}.\]
    Furthermore, the element $y$ is the unique element of $N$ with the property that $(u - y) \perp N$. Denote $y = u - \overline{u}$. Note that since $u \not \in N$, $\int_0^1 u(x) \ dx \neq 0$, thus $\overline{u} \not \in N$. We next need to show that $y \in N$. This is shown by the following:
    \[\int_0^1 u(x) - \overline{u} \ dx = \int_0^1 u(x) \ dx - \int_0^1\overline{u} \ dx = \overline{u} - \overline{u} = 0.\]
    Next we need to show that $(u - y) \perp N$. Take $n \in N$, then
    \[\int_0^1 (u - y)(x) n(x) \ dx = \overline{u} \int_0^1 n(x) = 0.\]
    Therefore, $d(u, N) = |\overline{u}|$.
\end{solution}


\newpage
\subsection{Problem 6.2, part b}
If $u(x) = x \in X$, show that
\[d(x, M) = \inf_{m \in M}\norm{u - m} = 1/2,\]
but that the infimum is not attained for any $m \in M$.
\partbreak
\begin{solution}

    From part a, we see that 
    \[d(x, M) = \left|\int_0^1 x \ dx\right| = \frac{1}{2}.\]
    Therefore, we choose $y = x - \frac{1}{2}$. Note that $y \not\in M$, however, since setting $x =0$, then $y = -\frac{1}{2}$. This violates the property any element has in $M$. Therefore, the claim has been shown.
\end{solution}

\newpage
\section{Problem 6.5}
Suppose that $\{H_n : n \in \N\}$ is a set of orthogonal closed subspaces of a Hilbert space $H$. We define the infinite direct sum
\[\bigoplus_{n = 1}^\infty H_n = \left \{ x_n : x_n \in H_n \text{ and } \sum_{n=1}^\infty \norm{x_n}^2 < \infty\right\}.\]
Prove that $\bigoplus_{n = 1}^\infty H_n$ is a closed linear subspace of $H$.
\partbreak
\begin{solution}

    For the sake of simplicity, let $H = \bigoplus_{n =1}^\infty$. To show that $H$ closed, take a sequence $x_n$ in $H$. We want to show that $x_n \into x \in H$. Note that, since $x_n \in H$, then $x_n$ can be written as a summation of sequences $y^k_n$, for which each $y^k_n \in H_k$. Then $x_n = \sum_{k}y^k_n$. Then, 
    \[\norm{x_n - x_m}^2 = \norm{\sum_{k} y^k_n - y^k_m}^2 = \sum_{k} \norm{y^k_n - y^k_m}^2\]
    Note that since each $H_k$ is closed, then $\sum_{k} \norm{y^k_n - y^k_m}^2 \into 0$. Note that summing over all $k$ implies 
    \[\norm{y^k_n - y_m^k}^2 \leq \sum_k\norm{y^k_m - y^k_n}^2 \into 0.\]
    This tells us $\norm{y^k_n - y_m^k} \into 0$ for all $k$. Since each $H_k$ is closed, we have that $y_n^k \into y^k$ for some $y^k \in H_k$. We can then show that $x_n \into \sum_k y^k$.
    \[\norm{x_n - \sum_k y^k}^2 = \norm{\sum_k (y^k_n - y_n)}^2 = \sum\norm{y^k_n - y_n}^2 \leq \liminf_{m \into \infty} \sum_k \norm{y^k_n - y^k_m}^2.\]
    Since each $\norm{y^k_n - y^k_m}$ can be made less than $\ep$ for sufficiently large $m, n$, we have that $\norm{x_n - \sum_k y^k}^2 < \ep$. Therefore, $x = \sum_k y^k$. We now need to show that $y_k \in H$. Note that
    \[\sum_k\norm{y^k}^2 \leq \liminf_{m \into \infty}\sum_k\norm{y^k_m}^2 = \liminf_{m \into \infty} \norm{x_m}^2 = \norm{x}^2 < \infty.\]
    Therefore, $y_k \in H$, so $H$ is closed. 
\end{solution}


\newpage
\section{Problem 6.11}
Prove that if $M$ is a dense linear subspace of a Hilbert space $H$, then $H$ has an orthonormal basis consisting of elements in $M$. Does the same result hold for arbitrary dense subsets of $H$?
\partbreak
\begin{solution}

    Since we are not given information on the dimensionality of $H$, we need to consider two separate cases of when $H$ is finite or infinite dimensional. We will first consider the case when $H$ is finite dimensional. Since any subspace of a linear space is closed in finite dimensions, the only dense linear subspace of $H$ is the space $H$ itself. The claim that $H$ then has an orthonormal basis from $M$ holds since $H$ has an orthonormal basis. \par
    
    \hop 
    Next consider the case when $H$ is infinite dimesnsional. Since we suppose that $M$ is a dense linear subspace of $H$, we can take $H$ to be separable. Because of this, there is a countable dense subset $\{x_n : n \in \N\}$ of $H$. Note that this subset is agnostic to $M$. Since $M$ itself is dense in $H$, we can take a sequence $x_{m, n} \in M$ such that $x_{m, n} \into x_n$ as $m \into \infty.$ Then the set $\{ x_{m, n} : m, n \in \N\}$ is a countable subset of $M$, which is dense in $H$. \par

    \hop
    To connect this new set to $H$, suppose that we have an arbitrary subset $P$ of $M$ that is dense in $H$. Let $P_B = \{x_n : n \in \N\}$ then be the largest linearly independent subset of $P$. Then since $P_B$ spans every element of $B$, $B \subseteq P_B$, thus $P_B$ is dense in $H$. Therefore, the closure of $P_B$ is then equal to $H$. Via Gram-Schmidt, we can take the basis $P_B$ and transform it into an orthonormal set of vectors, $P_N$ whose closed span is equal to the span of $P_B$. This implies that $P_N$ is an orthonormal basis of $H$. Since we have that elements of $P_N$ are linear combinations of elements of $M$, $P_N \subseteq M$, thus $P_N$ is an orthonormal basis of $H$ whose elements belong to the dense linear subspace $M$. \par

    \hop
    To give a counterexample for any arbitrary dense subset, take $H = \R^2$, and consider $P_B$ to be any rotation of the basis vectors, $e_1, e_2$ by an irrational angle, say $n\pi$ degrees for any $n \in \N$. Then, $P_B$ is dense in $H$, however, $P_B$ does not have a finite set of linearly independent vectors. Since $\pi$ is irrational, rotations of the basis vectors will never overlap since, if they did, then that would imply that $\pi$ was irrational, which is not the case. Therefore, the result cannot hold for arbitrary dense subsets of a Hilbert space.  
\end{solution}

\newpage
\section{Problem 6.14}
Define the Hermite polynomials $H_n$ by 
\[H_n(x) = (-1)^ne^{x^2} \frac{d^n}{dx^n}\left( e^{-x^2}\right).\]
\subsection{Problem 6.14, part a}
Show that 
\[\ph_n(x)  = e^{-x^2/2}H_n(x)\]
is an orthogonal set in $L^2(\R)$.
\partbreak
\begin{solution}

    First, since each $\ph_n$ is orthogonal to every function of the form $e^{(-x^2/2)}p_m$, where $p_m$ is a polynomial of lower degree of $n$, we just need to show that $e^{-x^2/2}x^m$ is orthogonal to $\ph_n$. Integrating by parts $m$ times, and taking each residual from integration by parts to go to zero, we can see that
    \[\int_\R e^{-x^2/2}\ph_n(x) \ dx = (-1)^n\int_\R x^m \frac{d^n}{dx^n} \left( e^{-x^2}\right) \ dx = (-1)^{m+n}m!\int_\R \frac{d^{n-m}}{dx^{n-m}}\left(e^{-x^2}\right) \ dx = 0.\]
    The last integral is equal zero since the differentiated function vanishes as $|x| \into \infty$.
\end{solution}

\newpage
\subsection{Problem 6.14, part b}
show that the $n$-th Hermite function $\ph_n$ is an eigenfunction of the linear operator
\[H = -\frac{d^2}{dx^2} + x^2\]
with eigenvalue $\lm_n = 2n+1$.
\partbreak
\begin{solution}

    First, let 
    \[A = \frac{d}{dx} + x, \quad A^* = -\frac{d}{dx} + x.\]
    We will first show that $AA^* - 1 = H$. Taking a test function $\psi$, we see that,
    \tightalignbreak
    \begin{align*}
        &(AA^*)\psi = A(A^* \psi) \\
        &= A\left( -\frac{d\psi}{dx} + x\psi \right) \\
        &= -A\left(\frac{d\psi}{dx}\right) + A\left( x\psi\right)\\
        &= -\left( \frac{d^2\psi}{dx^2} + x\frac{d\psi}{dx}\right) + \frac{d}{dx}\left( x\psi\right) + x^2\psi\\
        &= -\left( \frac{d^2\psi}{dx^2} + x\frac{d\psi}{dx}\right) + x\frac{d\psi}{dx} + \psi + x^2\psi\\
        &= - \frac{d^2\psi}{dx^2} + \psi + x^2\psi\\
        &= \left(- \frac{d^2}{dx^2} + 1 + x^2\right)\psi
    \end{align*}\vspace{-10mm}\alignbreak
    Therefore, the action that $AA^*$ preforms on $\psi$ is equivalent to the form above, which is equivalent to $H + 1$. This implies that $AA^* - 1 = H$, which is what we wanted to show.\par

    \newpage
    Next, we need to prove the following recurrence relation between Hermite Polynomials:
    \[\frac{dH_n}{dx} = 2n H_{n-1} = -H_{n+1} + 2xH_n\]
    We can relate the first and the third relations together via:
    \begin{align*}
        \frac{dH_n}{dx} &= (-1)^n\frac{d}{dx}\left[ e^{x^2} \frac{d^n}{dx^n}\left(e^{-x^2}\right)\right]\\
        &= (-1)^n e^{x^2} \frac{d^{n+1}}{dx^{n+1}}\left(e^{-x^2}\right) + (-1)^n 2xe^{x^2}\frac{d^n}{dx^n}\left( e^{-x^2} \right)\\
        &= -H_{n+1} + 2xH_n
    \end{align*}

    Next, we can observe the differentiation term in the $n+1$ Hermite polynomial.
    \begin{align*}
        \frac{d^{n+1}}{dx^{n+1}}\left( e^{-x^2}\right) &= \frac{d^n}{dx^n}\left(-2xe^{-x^2}\right)\\
        &= -2x\frac{d^n}{dx^n}\left( e^{-x^2}\right) - 2n \frac{d^{n-1}}{dx^{n-1}}\left( e^{-x^2}\right)
    \end{align*}
    We can then multiply both sides by $(-1)^{n+1}e^{x^2}$ to get 
    \[H_{n+1} = 2xH_n - 2nH_{n-1}\]
    Using the found equation above, we can remove the $H_n$ term from both sides to get
    \[\frac{dH_n}{dx} = 2nH_{n-1}\]
    Next, we need to investigate the actions $A$ and $A^*$ have on on $\ph_n$. From the found relations above, we have
    \begin{align*}
        A\ph_n &= \left( \frac{d}{dx} + x\right)\left( e^{-x^2/2}H_n\right)\\
        &= \frac{d}{dx}\left[ e^{-x^2/2}H_n\right] + xe^{-x^2/2}H_n\\
        &= -xe^{-x^2/2}H_n + e^{-x^2/2}\frac{dH_n}{dx} + xe^{-x^2/2}H_n\\
        &= e^{-x^2/2}\frac{dH_n}{dx}\\
        &= 2ne^{-x^2/2}H_{n-1}\\
        &= 2n\ph_{n-1}
    \end{align*}
    \newpage
    Similarly,
    \begin{align*}
        A^*\ph_n &= \left(-\frac{d}{dx} + x\right)\left(e^{-x^2/2}H_n\right)\\
        &= -\frac{d}{dx}\left[ e^{-x^2/2}H_n\right] + xe^{-x^2/2}H_n\\
        &= xe^{-x^2/2}H_n - e^{-x^2/2}\frac{dH_n}{dx} + xe^{-x^2/2}H_n\\
        &= e^{-x^2/2}\left(-\frac{dH_n}{dx} + 2xH_n\right)\\
        &= e^{-x^2/2}H_{n+1}\\
        &= \ph_{n+1}
    \end{align*}

    We can now finally see that action $H$ has on $\ph_n$. Via everything we have shown above, we can write
    \begin{align*}
        H\ph_n &= (AA^* - 1)\ph_n\\
        &= AA^*\ph_n - \ph_n\\
        &= A(\ph_{n+1}) - \ph_n\\
        &= 2(n+1)\ph_n - \ph_n\\
        &= (2n+1)\ph_n
    \end{align*}
    Therefore, $\ph_n$ is an eigenfunction of $H$ with eigenvalue $2n+1$.
\end{solution}
\end{document}