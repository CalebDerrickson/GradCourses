\documentclass[12pt]{article}
\usepackage[paper=letterpaper,margin=1.5cm]{geometry}
\usepackage{amsmath}
\usepackage{amssymb}
\usepackage{amsfonts}
\usepackage{mathtools}
%\usepackage[utf8]{inputenc}
%\usepackage{newtxtext, newtxmath}
\usepackage{lmodern}     % set math font to Latin modern math
\usepackage[T1]{fontenc}
\renewcommand\rmdefault{ptm}
%\usepackage{enumitem}
\usepackage[shortlabels]{enumitem}
\usepackage{titling}
\usepackage{graphicx}
\usepackage[colorlinks=true]{hyperref}
\usepackage{setspace}
\usepackage{subfigure} 
\usepackage{braket}
\usepackage{color}
\usepackage{tabularx}
\usepackage[table]{xcolor}
\usepackage{listings}
\usepackage{mathrsfs}
\usepackage{stackengine}
\usepackage{physics}
\usepackage{afterpage}
\usepackage{pdfpages}
\usepackage[export]{adjustbox}
\usepackage{biblatex}

\setstackEOL{\\}

\definecolor{dkgreen}{rgb}{0,0.6,0}
\definecolor{gray}{rgb}{0.5,0.5,0.5}
\definecolor{mauve}{rgb}{0.58,0,0.82}


\lstset{frame=tb,
  language=Python,
  aboveskip=3mm,
  belowskip=3mm,
  showstringspaces=false,
  columns=flexible,
  basicstyle={\small\ttfamily},
  numbers=none,
  numberstyle=\tiny\color{gray},
  keywordstyle=\color{blue},
  commentstyle=\color{dkgreen},
  stringstyle=\color{mauve},
  breaklines=true,
  breakatwhitespace=true,
  tabsize=3
}
\setlength{\droptitle}{-6em}

\makeatletter
% we use \prefix@<level> only if it is defined
\renewcommand{\@seccntformat}[1]{%
  \ifcsname prefix@#1\endcsname
    \csname prefix@#1\endcsname
  \else
    \csname the#1\endcsname\quad
  \fi}
% define \prefix@section
\newcommand\prefix@section{}
\newcommand{\prefix@subsection}{}
\newcommand{\prefix@subsubsection}{}
\renewcommand{\thesubsection}{\arabic{subsection}}
\makeatother
\DeclareMathOperator*{\argmin}{argmin}
\newcommand{\partbreak}{\begin{center}\rule{17.5cm}{2pt}\end{center}}
\newcommand{\alignbreak}{\begin{center}\rule{15cm}{1pt}\end{center}}
\newcommand{\tightalignbreak}{\vspace{-5mm}\alignbreak\vspace{-5mm}}
\newcommand{\hop}{\vspace{1mm}}
\newcommand{\jump}{\vspace{5mm}}
\newcommand{\R}{\mathbb{R}}
\newcommand{\C}{\mathbb{C}}
\newcommand{\N}{\mathbb{N}}
\newcommand{\G}{\mathbb{G}}
\renewcommand{\S}{\mathbb{S}}
\newcommand{\bt}{\textbf}
\newcommand{\xdot}{\dot{x}}
\renewcommand{\star}{^{*}}
\newcommand{\ydot}{\dot{y}}
\newcommand{\lm}{\mathrm{\lambda}}
\renewcommand{\th}{\theta}
\newcommand{\id}{\mathbb{I}}
\newcommand{\si}{\Sigma}
\newcommand{\Si}{\si}
\newcommand{\inv}{^{-1}}
\newcommand{\T}{^\intercal}
\renewcommand{\tr}{\text{tr}}
\newcommand{\ep}{\varepsilon}
\newcommand{\ph}{\varphi}
%\renewcomand{\norm}[1]{\left\lVert#1\right\rVert}
\definecolor{cit}{rgb}{0.05,0.2,0.45}
\addtolength{\jot}{1em}
\newcommand{\solution}[1]{

\noindent{\color{cit}\textbf{Solution:} #1}}

\newcounter{tmpctr}
\newcommand\fancyRoman[1]{%
  \setcounter{tmpctr}{#1}%
  \setbox0=\hbox{\kern0.3pt\textsf{\Roman{tmpctr}}}%
  \setstackgap{S}{-.9pt}%
  \Shortstack{\rule{\dimexpr\wd0+.1ex}{.9pt}\\\copy0\\
              \rule{\dimexpr\wd0+.1ex}{.9pt}}%
}

\newcommand{\Id}{\fancyRoman{2}}

% Enter the specific assignment number and topic of that assignment below, and replace "Your Name" with your actual name.
\title{STAT 31210: Homework 2}
\author{Caleb Derrickson}
\date{PUT DATE HERE}

\begin{document}
\onehalfspacing
\maketitle
\allowdisplaybreaks
{\color{cit}\vspace{2mm}\noindent\textbf{Collaborators:}} The TA's of the class, as well as Kevin Hefner, and Alexander Cram.

\tableofcontents

\newpage
\section{Problem 1}
Suppose that $f: G\rightarrow \R$ is a uniformly continuous function defined on an open subset $G$ of a metric space $X$. Prove that $f$ has a unique extension to a continuous function $\overline{f} : \Bar{G} \rightarrow \R$ defined on the closure $\Bar{G}$ of $G$. Show that such an extension need not exist if $f$ is continuous but not uniformly continuous on $G$.

\newcommand{\fbar}{\Bar{f}}

\partbreak
\begin{solution}

    Suppose we have a sequence $x_n \in \Bar{G}$. Since $\Bar{G}$ is closed, we can assume this sequence converges to the value $x \in \Bar{G}$. Define the extension $\fbar : \Bar{G} \rightarrow \R$ as $\fbar (x) = \lim_{n \rightarrow \infty} f(x_n)$. Note for all $n \in \N, x_n \in G$. And since $f$ is uniformly continuous on $G$, and also $x_n$ is Cauchy (it converges), $f(x_n)$ is then Cauchy. This is because for any sufficiently large $n, m \in \N$, we can make $d(x_n, x_m)$ arbitrarily small, i.e. less than some $\ep > 0$. $f$ is uniformly continuous, so for points close enough in the domain (take $x, y$), then their distance in the target space is also arbitrarily small. Therefore, for sufficiently large $N \in \N$ and $n, m \geq N$, we see that $d(x_n, x_m) < \ep$ for any $\ep > 0$. Therefore, $d(f(x_n), f(x_m)) < \ep$. \par

    \jump
    Note that the target space, $\R$, is complete, so stating $\fbar(x \in \Bar{G}) = \lim_{n\rightarrow \infty} f(x_n)$ is well defined. We need to show now that $\fbar$ is unique and uniformly continuous. 

    \tightalignbreak
    \begin{itemize}[-]
        \item \underline{$\fbar$ is unique.}

        \jump
        Since $\R$ is complete and $f(x_n)$ is Cauchy, then the sequence has a limit, so this is well defined. If $a, b \in \R$, and $a = \lim_{n \rightarrow \infty} \fbar(x_n) = b$, then by the triangle inequality, 
        \[d(a, b) \leq d(a, f(x_n)) + d(f(x_n), b).\]
        Since $f(x_n) \rightarrow a$ and $f(x_n) \rightarrow b$, these two distances will approach zero in the limit as $\ep \rightarrow 0$. Therefore, $d(a, b) \leq \ep \rightarrow 0$, so $d(a, b) = 0 \iff a = b$. Therefore, $\fbar$ is unique.

        \item \underline{$\fbar$ is uniformly continuous.}

        \jump
        Since $f$ is continuous, there exists a $\delta > 0$ such that for $d(x, y) < \delta / 3, \ d(f(x), f(y)) < \ep / 3$, for some $\ep > 0$, and for any $x, y \in \Bar{G}$. Since $\Bar{G}$ is closed, there exists sequences $x_n \in G, \ y_n \in G$ for which $x_n \rightarrow x$ and $y_n \rightarrow y$ as $n \rightarrow \infty$. Explicitly, we see when $n \geq N_1 \in \N$, we can have $d(x_n, x) < \delta / 3$ and $d(y_n, y) < \delta / 3$. \par

        \jump
        When $n \geq N_1$, we see $d(x_n, y_n) \leq d(x_n, x) + d(x, y) + d(y, y_n)$ by the triangle inequality. These can all be made less than $\delta / 3$, therefore $d(x_n, y_n) < \delta$. This will imply that their distances under the transformation can be made arbitrarily small. In this case, we let $d(f(x_n), f(y_n)) < \ep / 3$. \par

        \jump
        Since $f(x_n)$ and $f(y_n)$ are both Cauchy sequences in $\R$, we see that $f(x_n) \rightarrow \fbar(x)$ and $f(y_n) \rightarrow \fbar(y)$. Therefore, there exists an $\N_2 \in \N$ such that, when $n \geq N_2$, implies both $d(f(x_n), \fbar(x))$ and $d(f(y_n), \fbar(y)$ can be made less than $\delta / 3$. Choose $N = \max \{ N_1, N_2 \}$. Therefore, for any $n \geq N$, 
        \[d(\fbar(x), \fbar(y)) \leq d(\fbar(x), f(x_n)) + d(f(x_n), f(y_n)) + d(f(y_n), \fbar(y)).\]
        All terms on the right hand side were shown to be less than $\ep / 3$. Therefore, $d(\fbar(x), \fbar(y)) < \ep$, implying $\fbar$ is uniformly continuous.
    \end{itemize}
\end{solution}

\newpage
\section{Problem 2.6}
Show that the space $C([a, b])$ equipped with the $L^1$-norm $\norm{\cdot}_1$ defined as 
\[\norm{f}_1 = \int_a^b |f(x)| dx\]
is incomplete. Show that if $f_n \rightarrow f$ with respect to the sup-norm, then $f_n \rightarrow f$ with respect to $\norm{\cdot}_1$. Give a counterexample to show the converse is not true.
\partbreak
\begin{solution}
    
\end{solution}
\end{document}