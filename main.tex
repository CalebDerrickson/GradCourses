\documentclass[12pt]{article}
\usepackage[paper=letterpaper,margin=1.5cm]{geometry}
\usepackage{amsmath}
\usepackage{amssymb}
\usepackage{amsfonts}
\usepackage{mathtools}

\usepackage{lmodern}     % set math font to Latin modern math
\usepackage[T1]{fontenc}
\renewcommand\rmdefault{ptm}
\usepackage[shortlabels]{enumitem}
\usepackage{titling}
\usepackage{graphicx}
\usepackage[colorlinks=true]{hyperref}
\usepackage{setspace}
\usepackage{subfigure} 
\usepackage{braket}
\usepackage{color}
\usepackage{tabularx}
\usepackage[table]{xcolor}
\usepackage{listings}
\usepackage{mathrsfs}
\usepackage{stackengine}
\usepackage{physics}
\usepackage{afterpage}
\usepackage{tikz}
\usepackage{pdfpages}
\usepackage[export]{adjustbox}
\usepackage{biblatex}

\setstackEOL{\\}

\definecolor{dkgreen}{rgb}{0,0.6,0}
\definecolor{gray}{rgb}{0.5,0.5,0.5}
\definecolor{mauve}{rgb}{0.58,0,0.82}


\lstset{frame=tb,
  language=Python,
  aboveskip=3mm,
  belowskip=3mm,
  showstringspaces=false,
  columns=flexible,
  basicstyle={\small\ttfamily},
  numbers=none,
  numberstyle=\tiny\color{gray},
  keywordstyle=\color{blue},
  commentstyle=\color{dkgreen},
  stringstyle=\color{mauve},
  breaklines=true,
  breakatwhitespace=true,
  tabsize=3
}
\setlength{\droptitle}{-6em}

\makeatletter
% we use \prefix@<level> only if it is defined
\renewcommand{\@seccntformat}[1]{%
  \ifcsname prefix@#1\endcsname
    \csname prefix@#1\endcsname
  \else
    \csname the#1\endcsname\quad
  \fi}
% define \prefix@section
\newcommand\prefix@section{}
\newcommand{\prefix@subsection}{}
\newcommand{\prefix@subsubsection}{}
\renewcommand{\thesubsection}{\arabic{subsection}}
\makeatother
\DeclareMathOperator*{\argmin}{argmin}
\newcommand{\partbreak}{\begin{center}\rule{17.5cm}{2pt}\end{center}}
\newcommand{\alignbreak}{\begin{center}\rule{15cm}{1pt}\end{center}}
\newcommand{\tightalignbreak}{\vspace{-5mm}\alignbreak\vspace{-5mm}}
\newcommand{\hop}{\vspace{1mm}}
\newcommand{\jump}{\vspace{5mm}}
\newcommand{\R}{\mathbb{R}}
\newcommand{\C}{\mathbb{C}}
\newcommand{\N}{\mathbb{N}}
\newcommand{\G}{\mathbb{G}}
\renewcommand{\S}{\mathbb{S}}
\newcommand{\bt}{\textbf}
\newcommand{\xdot}{\dot{x}}
\newcommand{\ydot}{\dot{y}}
\newcommand{\lm}{\mathrm{\lambda}}
\renewcommand{\th}{\theta}
\newcommand{\id}{\mathbb{I}}
\newcommand{\si}{\Sigma}
\newcommand{\Si}{\si}
\newcommand{\inv}{^{-1}}
\newcommand{\T}{^{\intercal}}
\renewcommand{\tr}{\text{tr}}
\newcommand{\ep}{\varepsilon}
\newcommand{\ph}{\varphi}
\newcommand{\range}{\text{range}}
\newcommand{\scP}{\mathcal{P}}
\newcommand{\scT}{\mathbb{T}}
\newcommand{\into}{\rightarrow}
\newcommand{\scM}{\mathcal{M}}
\newcommand{\scH}{\mathcal{H}}
\newcommand{\scN}{\mathcal{N}}
\newcommand{\scV}{\mathcal{V}}
\newcommand{\scW}{\mathcal{W}}
\renewcommand{\grad}{\nabla}
\renewcommand{\star}{^{*}}

\definecolor{cit}{rgb}{0.05,0.2,0.45}
\addtolength{\jot}{1em}
\newcommand{\solution}[1]{


\noindent{\color{cit}\textbf{Solution:} #1}}

\newcounter{tmpctr}
\newcommand\fancyRoman[1]{%
  \setcounter{tmpctr}{#1}%
  \setbox0=\hbox{\kern0.3pt\textsf{\Roman{tmpctr}}}%
  \setstackgap{S}{-.9pt}%
  \Shortstack{\rule{\dimexpr\wd0+.1ex}{.9pt}\\\copy0\\
              \rule{\dimexpr\wd0+.1ex}{.9pt}}%
}

\newcommand{\Id}{\fancyRoman{2}}

% Enter the specific assignment number and topic of that assignment below, and replace "Your Name" with your actual name.
\title{STAT 31050: Homework 1}
\author{Caleb Derrickson}
\date{April 1, 2024}

\begin{document}
\onehalfspacing
\maketitle
\allowdisplaybreaks

\tableofcontents

\newcommand{\barP}{\Bar{P}}
\newcommand{\barQ}{\Bar{Q}}
\renewcommand{\braket}[2]{\langle #1, #2 \rangle}

\newpage
\section{Exercise 1}
Prove that the Trapezoid rule follows the following error inequality for integration:
\[|T_n(f) - I(f)| \leq \frac{C}{n^2}\norm{f''}_\infty.\]
\partbreak
\begin{solution}

    The overview of my solution will be first showing the error for $n = 1$ for generic bounds $(a, b)$, then applying that to the case for $n \neq 1$ (I will explain this better when we get to it). For $n = 1$, and assuming our function $f$ is at least twice differentiable, then by Taylor's Theorem, \footnote{I am citing the Wikipedia page for Taylor's Theorem.} 
    \[f(x) = f(a) + f'(a)(x - a) + R_1(x).\]
    Here, the term $R_1(x)$ represents the remainder term, in particular the mean value form of it. This is given as 
    \[R_1(x) = \frac{f''(\xi_L)}{2}(x - a)^2,\]
    where $a \leq \xi_L \leq b$. Since the point we are expanding from in Taylor's Theorem is generic ( within $[a, b]$), we can also expand from the point $x = b$ to get 
    \[f(x) = f(b) + f'(b)(x - b) + R_1(x).\]
    The remainder term is slightly different in this case, where we have\footnote{The value $\xi_L$ is somewhat loose here; I am assuming it is equal over different Taylor expansions, since the remainder term, or the Peano remainder term at least, is unique. This does not change the calculation, since we would just sum over the sup norm at the end.} 
    \[R_1(x) = \frac{f''(\xi_L)}{2}(x - a)^2.\]
    Since these two expansions are equivalent, we can sum them both and divide to get the original function, so
    \[f(x) = \frac{1}{2}(f(a) + f(b)) + \frac{1}{2}\left[f'(a)(x - a) +f'(b)(x - b)\right] + \frac{f''(\xi_L)}{4}[(x - a)^2 + (x - b)^2]\]
    Integrating from $a$ to $b$, and after some factorization, we get 
    \begin{align*}
        \int_a^b f(x) \ dx &= \frac{1}{2}(f(a) + f(b)) + \frac{1}{2}f'(a) \left( \frac{1}{2}(b - a)^2\right) + \frac{1}{2}f'(b) \left( -\frac{1}{2}(b - a)^2\right) + \frac{f''(\xi_L)}{6}(b-a)^3\\
        &= \frac{1}{2}(f(a) + f(b))+ \frac{(b - a)^2}{4}[f'(a) - f'(b)] + \frac{f''(\xi_L)}{6}(b - a)^3.
    \end{align*}
    Since $f \in C^2([a, b])$, for some $\eta \in [a, b]$, we can apply the Mean Value Theorem to the first term to get
    \[\int_a^b f(x) \ dx = \frac{1}{2}(f(a) + f(b)) - \frac{f''(\eta)}{4}(b - a)^3 + \frac{f''(\xi_L)}{6}(b - a)^3,\]
    note the minus sign, which comes from inverting the Mean Value Theorem. Comparing with the first term of Trapezoid integration, 
    \[T_1(f) = \frac{f(a) + f(b)}{2},\]
    we see that this is the first term in integration. To get the error in Trapezoid integration, we then have
    \[E_1 = |I(f) - T_1(f)| = (b - a)^3\left|\frac{f''(\eta)}{4} - \frac{f''(\xi_L)}{6}\right|,\]
    where I assume $a < b$. Since $f''$ is continuous, there exists a value, which I will denote as $m$, which attains the maximum value for $f''$ on the interval $[a, b]$. We then have
    \[E_1 = |I(f) - T_1(f)| \leq  \frac{(b - a)^3}{12}|f''(m)| = \frac{(b - a)^3}{12}\norm{f''}_\infty.\]
    
    Therefore, the first order Trapezoid error is bounded by the above result. When we go to the case where $n \neq 1$, that is, when we consider summing over any number of trapezoid approximations, we can apply this error to each interval $[x_{j-1}, x_j]$ in question. That is, if we consider partitioning the interval $[0, 1]$ into intervals $[x_{j-1}, x_j]$ and approximating the function $f(x)$ over that interval by the Trapezoid method, then we can apply the error term for each interval, and sum over them. This is what I take as "gluing," which is mentioned in the hint. \par

    \jump
    Since the function $f$ is continuous, we can then split up the integral with respect to the above mentioned partition. Note that I have not mentioned the explicit form of the $x_i$ terms - we will take them to be evenly spaced between $[0, 1]$ by the formula $x_j = jh$, where $j \in \N$ with $j \leq n$ and $h = \frac{1}{n}$. We then have 
    \[I(f) \leadsto \sum_{j = 1}^n \int_{x_{j-1}}^{x_j} f(x) \ dx.\]
    On each interval $[x_{j-1}, x_j]$, we approximate the function by a trapezoid, which will incur error by the above result. Setting bounds $a = x_{j-1}$ and $b = x_j$, we have the individual error terms as 
    \[E([x_{j-1}, x_j]) = \frac{h^3}{12}\norm{f''}_{\infty, j}, \quad x_{j-1} \leq c_j \leq x_j.\]

    Note the sup norm is over the specific interval rather than the total interval. The total error in integration is then summing over all $j$, which will give us 
    \[E_n(f) = \frac{h^3}{12}\sum_{j = 1}^n \norm{f''}_{\infty, j}.\]

    The max value attained on each interval is certainly smaller than the maximum value over the total interval, so we can bound the error by swapping the sup values. Noticing that the sum is independent of $j$, we get that 
    \[E_n(f) \leq \frac{nh^3}{12}\norm{f''}_{\infty} = \frac{1}{12n^2}\norm{f''}_{\infty},\]

    which, when setting $M = \frac{1}{12}$, we get the desired form. 
    
\end{solution}

\newpage
\section{Exercise 3}
Assuming that $f \in C^4(0, 1)$, construct an improved quadrature rule which computes the integral of $f$ on $(0, 1)$ with an error which is $O(\frac{1}{n^4})$. Your rule should only involve a constant number of modifications to the standard left-hand or right-hand Riemann sums as $n \into \infty$. Implement your quadrature rule and plot the convergence for:
\begin{itemize}
    \item $f(x) = e^x$
    \item $f(x) = \sqrt{x}$
\end{itemize}
Compare your numerical results with those obtained from Simpson’s rule – using a cubic approximation on each subinterval, where the cubic is chosen to agree with f at the two endpoints and the midpoint of each subinterval.
\partbreak
\begin{solution}

    After some tweaking of coefficients in Python, I found a suitable quadrature rule which only uses slight modifications of Left and Right Reimann sums and scales the same as Simpon's 1/3 composite rule ($O(h^4)$). My proposed quadrature rule, $M_n$, is as follows:
    \[M_n = \frac{h}{3}\left[\frac{f(0) + f(1)}{2}  + 3\sum_{k = 1}^{n-1}f(k/n)\right]\]
    Note this is suitably scaled over $[0, 1]$, scaling like terms of the Left and Right Reimann sums by 3, and averaging the end points. The results are given below, where we take $n \to 100$. It seems as though Simpson's rule creates a jagged form of our approximation method, which implies that our proposed quadrature rule scales as Simpon's 1/3 composite rule.
\end{solution}

\begin{figure}[htbp]
    \centering
    \subfigure{\includegraphics[width=0.4\textwidth]{Figures/Problem3 ex.png}}
    \hfill
    \subfigure{\includegraphics[width=0.4\textwidth]{Figures/Problem3 sqrtx.png}}
\end{figure}

\newcommand{\scI}{\mathcal{I}}

\clearpage
\newpage
\section{Exercise 4}
Let $\{v_j\}_{j = 1}^\infty$ be an orthonormal basis of a Hilbert space $\scH$. Let $A$ be a bounded operator from $\scH$ to itself, with bounded inverse. Consider the sequence of vectors $w_j := Av_j$. Is $\{w_j\}_{j = 1}^\infty$ an orthonormal basis? A Schauder basis? If so, can you bound its basis constant?
\partbreak
\begin{solution}

    We can see that $w_j = Av_j$ gives us a transformation between $w_j$ and $v_j$. Since $A$ is a bounded operator with bounded inverse, it a continuous operator, it is bijective. Therefore, for any $x \in \scH$, we can write 
    \begin{align*}    
        &x = \sum_{i \in \scI} \braket{x}{v_i}v_i = \sum_{i \in \scI} \braket{x}{v_i}A\inv w_i\\
        &= A\inv\left(\sum_{i \in \scI} \braket{x}{v_i} w_i\right)\\
        \implies & Ax = \sum_{i \in \scI} \braket{x}{v_i}w_i
    \end{align*}
    Since $A$ is bijective, then we can write $Ax = y$ for $y \in \scH$. We can further write
    \begin{align*}
        &y = \sum_{i \in \scI} \braket{A\inv y}{A\inv w_i}w_i\\
        &= \sum_{i \in \scI} \braket{y}{(A\inv)\star A\inv w_i}w_i\\
        &= \norm{A\inv}_2^2 \sum_{i \in \scI} \braket{y}{w_i}w_i
    \end{align*}
    Thus, each vector in $\scH$ can be represented with the $w_j$ vectors, implying that $w_j$ is an orthonormal basis. To prove that it is a Schauder basis, we first note that $v_j$ is a Schauder basis for the Hilbert space $\scH$, since it is an orthonormal basis (here we assume $\scH$ is separable). this implies for any $x \in \scH$, there exists unique coefficients $x_1, ...$ for which 
    \[x = \sum_{i \in \scI} x_i v_i = \lim_{n \to \infty}\sum_{i = 1}^n x_iv_i\]
    Since $A$ is bijective, we can multiply through to get 
    \[Ax = \lim_{n \to \infty}\sum_{i = 1}^n x_iAv_i = \lim_{n \to \infty}\sum_{i = 1}^n x_iw_i\]
    We can also furthermore write $y = Ax$ for $y \in \scH$. This implies that any $y \in \scH$ is expressible as a linear combination of the $w_i$ components, implying that $w_i$ is a Schauder basis. By the above calculations, we have that $\norm{A\inv}_2 = 1$, therefore, $A\inv$ is a unitary matrix.  
\end{solution} 

\newcommand{\scX}{\mathcal{X}}

\newpage
\section{Exercise 5}
Let $\psi : \R \to \R$ be the function defined by 
\[\psi(x) = \begin{cases}
    0, &x \not \in (0, 1),\\
    2x, &x \in [0, 1/2],\\
    2(1 - x), &x \in [1/2, 1].
\end{cases}\]
For $i = 0, ..., \infty$ and $k = 0, ..., 2^i - 1$ define $\psi_{i, k}$ by 
\[\psi_{i, k}(x) = \psi(2^ix + k)\]
Consider the collection of functions $\scX = \{1, x, \psi_{0, 0}, \psi_{1, 0}, \psi_{1, 1}, \psi_{2, 0}, ...\}$. Sketch the first five functions of $\scX$ on $[0, 1]$. Show that $\scX$ is a Schauder basis of $C([0, 1])$. 
\partbreak
\begin{solution}

    I do not like adding hand-written sketches in homework, since I find them to be too messy for any meaningful interpretation. Instead, I have included a plot of the first five functions of $\scX$ in Figure \ref{fig:Exercise5} using Python. We will next show that $\scX$ is a Schauder basis of $C([0, 1])$. That is, we need to show that there exists unique coefficients $\alpha_1, \alpha_2, ...$ for which
    \[v = \sum_{j = 1}^\infty \alpha_j \scX_j \quad \text{for all } v \in C([0, 1]).\]
    Note that $C([0, 1])$ equipped with the sup-norm is a Banach space, since the space of continuous functions on a compact interval (the interval $[0, 1]$ is closed and bounded) is a Banach Space. \par
    
    \jump
    Suppose we have a linear combination of $N$ of these functions, which I will denote by $p_N$. Note that, the coefficients of $p_N$ can be chosen such that at the crest of each function $\psi_{i, k}$. We will show then that $p_N \to f$ uniformly on $[0, 1]$. Let $\ep > 0$. Note that $f \in C([0, 1])$, implying it is uniformly continuous. Suppose we have, for some $N_0 \in \N$, we have that for any $|x - y| \leq 2^{-N_0}$, then $|f(x) - f(y)| < \ep$. Note that for $x \in [0, 1]$ and $N \geq N_0$, we have that 
    \[\frac{k}{2^N} \leq x \leq \frac{k+1}{2^N},\]
    Note these are the crests of the functions $\psi_{N,k}$ and $\psi_{N, k+1}$ respectively. Denote these two points as $y_1, y_2$. We then have the following:
    \newpage
    \begin{align*}
        |p_N(x) - f(x)| &\leq |p_N(x) - p_N(y_1)| + |p_N(y_1) - f(y_1)| + |f(y_1) - f(x)| &\text{(Triangle Inequality.)}\\
        &\leq |p_N(y_1) - p_N(y_2)| + |p_N(y_1) - f(y_1)| + |f(y_1) - f(x)| &\text{(By above.)}\\
        &= |f(y_1) - f(y_2)| + |f(y_1) - f(y_1)| + |f(y_1) - f(x)| &\text{(Agreement of $p_N$ and $f$.)}\\
        &|f(y_1) - f(y_2)| + |f(y_1) - f(x)| &\text{(Simplifying.)}\\
        &< \ep &\text{(By assumption.)}
    \end{align*}
    This implies that $p_N \to f$ uniformly. Therefore, we have shown that every function in $C([0, 1])$ can be expressed as a linear combination of the $\psi_{i, k}$. To show it is a Schauder basis, we need to show these coefficients are unique. Suppose this is not the case. That is, there exists expansions in this basis for which, 
    \[f = \sum_{i = 0}^\infty c_i\scX_i = \sum_{i = 0}^\infty a_i\scX_i\]
    Suppose that after the $N$-th term, these expansions agree, that is, 
    \[\sum_{i = N}^\infty (c_i - a_i) f_i = 0, \quad \forall x \in [0, 1]\]
    Since this should hold for all $x \in [0, 1]$, we analyse the $x$ which maximizes the function $f_i$, that is, at the point $\frac{k}{2^M}$, where $f_N = \psi_{M, k}$. We then see that 
    \[0 = \sum_{i = N}^\infty (c_i - a_i)f_i\left(\frac{k}{2^M}\right) = c_N - a_N\]
    This implies that $c_N = a_N$, which is a contradiction. Therefore, the expansion is unique, implying that $\psi_{i, k}$ is a Schauder basis. 
\end{solution}

\begin{figure}[!h]
    \centering
    \includegraphics[width = 0.425\textwidth]{Figures/Exercise5.png}
    \caption{First five functions of $\scX$ for Exercise 5.}
    \label{fig:Exercise5}
\end{figure}
\clearpage

\newpage
\section{Exercise 6}
Let $\scH$ be a Hilbert space and let $P, Q: \scH \to \scH$ be two projections. In the following, set $C = i[P, Q] = i(PQ - QP)$. We also assume that $P$ and $Q$ are orthogonal, $i.e.$ $P\star = P$ and $Q\star = Q$.
\subsection{Exercise 6, part a}
Find explicit examples of $P$ and $Q$ which do not commute. Make sure to give clear definitions of both them and $\scH$. Sketch a geometric characterization of when this happens. 
\partbreak
\begin{solution}

    For an easy example, we examine the space $\R^2$ equipped with the standard inner product, and consider the two linear Projections (in matrix form)
    \[P = \mqty[1&0\\0&0], \quad Q = \frac{1}{2}\mqty[1&1\\1&1].\]
    Indeed, these two matrices represent projections on $\R^2$, since 
    \begin{align*}
        &P^2 = \mqty[1&0\\0&0]\mqty[1&0\\0&0] = \mqty[1&0\\0&0] = P\\
        &Q^2 = \frac{1}{4}\mqty[1&1\\1&1]\mqty[1&1\\1&1] = \frac{1}{4}\mqty[2&2\\2&2] = \frac{1}{2}\mqty[1&1\\1&1] = Q.
    \end{align*}
    Furthermore, both $P$ and $Q$ are orthogonal, i.e. $P\star = P, Q\star = Q$. We can then say, for any $x, y \in \scH$, 
    \[\braket{Px}{y} = x\T P\T y = x\T P y = \braket{x}{Py}, \braket{Qx}{y} = x\T Q\T y = x\T Q y = \braket{x}{Qy}.\]
    Therefore, $P$ and $Q$ are indeed projections. I have included a figure below to demonstrate their action on the Unit circle in $\R^2$. The two operators project onto the vectors (1, 1) and (1, 0), respectively.
\end{solution}
\vspace{3mm}
\begin{figure}[!h]
    \centering
    \includegraphics[width = 0.3\textwidth]{Figures/P and Q.png}
\end{figure}

\renewcommand{\ker}{\text{ker }}

\renewcommand{\range}{\text{range }}
\newpage
\subsection{Exercise 6, part b}
Show that $0 \leq \braket{v}{Pv} \leq 1$ for all $v \in \scH$. 
\partbreak
\begin{solution}

    For the lower bound, we use the property of idempotence of Projections,
    \[\braket{v}{Pv} = \braket{v}{P^2 v} = \braket{Pv}{Pv} = \norm{Pv}^2 \geq 0.\]
    Next, fix $v \in \scH$. Since $P$ is an orthogonal projection on $\scH$, we have that $\scH = \range P \oplus  \ker P$. Therefore, there exists $x \in \range P, y \in \ker P$ for which $v = x + y$. Then, 
    \[\braket{v}{Pv} = \braket{x+y}{x} = \braket{x}{x} + \braket{y}{x} = \norm{x}^2.\]
    Note that $\braket{y}{x} = 0$, since 
    \[\braket{y}{x} = \braket{y}{Px} = \braket{Py}{x} = \braket{0}{x} = 0.\]
    Therefore, we have that
    \[\frac{\braket{Px}{Px}}{\braket{x}{x}} = \frac{\norm{Px}^2}{\norm{x}^2}= 1\]
    But $\norm{v} \geq \norm{x}$, and since $\norm{Px} = \norm{Pv}$, we have
    \[\frac{\norm{Pv}}{\norm{v}} \leq 1. \]
    Here, we will assume that $\norm{v} = 1$ since for any non unitary $v \in \scH$, we can repeat the process for the vector $w = \alpha v$, where $\alpha = 1/\norm{v}$. Squaring both sides then gives us
    \[\norm{Pv}^2 \leq 1 \implies \braket{Pv}{Pv} \leq 1 \iff \braket{v}{Pv} \leq 1.\]
    The absolute value can be dropped on the above process, since we have shown that $\braket{v}{Pv} \geq 0$. Thus, the statement has been shown.
\end{solution}


\newpage
\subsection{Exercise 6, part c}
For any $v \in \scH$, set $\barP(v) = \braket{v}{Pv}, \barQ(v) = \braket{v}{Qv}, \ \sigma^2_P(v) = \braket{v}{(P - \barP \id)^2 v}$ and $\sigma_Q^2 = \braket{v}{(Q - \barQ \id)^2 v}$. Show that $\sigma_P, \sigma_Q \leq \norm{v} / 2$. Here, $\id$ is the identity operator on $\scH$.  
\partbreak
\begin{solution}

    Note that the above property need to be shown only for one projection, via symmetry of their respective definitions. I wasn't able to get exactly the form given, but I got something very similar.
    \tightalignbreak
    \begin{align*}
        &\sigma_P^2 = \braket{v}{(P - \barP \id)^2v} &\text{(Given.)}\\
        &= \braket{v}{P^2v - 2\barP Pv + \barP^2 v} &\text{(Expanding.)}\\
        &= \braket{v}{P^2v} - 2\barP \braket{v}{Pv} + \barP^2 \braket{v}{v} &\text{(Expanding.)}\\
        &= \norm{Pv}^2 - 2\barP \braket{v}{Pv} + \barP^2 \norm{v}^2 &\text{(Projection Properties.)}\\
        &= \norm{Pv}^2 - 2\barP\norm{Pv}^2 + \barP^2\norm{v}^2 &\text{(Projection Properties.)}\\
        &= \norm{v}^2\left(\frac{\norm{Pv}^2}{\norm{v}^2} - 2\barP\frac{\norm{Pv}^2}{\norm{v}^2} + \barP^2\right) &\text{(Grouping.)}\\
        &\leq \norm{v}^2 (1 - 2\barP + \barP^2) &\text{(By last part.)}\\
        &= \norm{v}^2(1 - \barP)^2 &\text{(Grouping.)}\\
        \implies &\sigma_P \leq \norm{v}(1 - \barP) &\text{(Taking square root.)}
    \end{align*}\vspace{-12mm}\alignbreak
    Note that the absolute value can be dropped from the last line since from part b, $ 0 \leq \barP \leq 1$. My only idea would be to break up $v$ into $v = x + y$ as in the last part, and since $\norm{Pv}^2 = \norm{x}^2$, we can say that $\norm{v} \geq \norm{Pv} = \norm{x}$ (By Triangle inequality). This does not guarantee however, that $\barP \geq 1/2$, which would bridge the gap between the two. However we can say for $v \in \scH$ for which $\norm{x} \geq 1/2$, then this would hold. This is assuming that $\norm{v} = 1$, by rescaling.  
\end{solution}

\newpage
\subsection{Exercise 6, part d}
If $C$ is the commuatator of $P$ and $Q$, show that $\frac{1}{4}|\braket{v}{Cv}|^2 \leq \sigma_P^2(v) \sigma_Q^2(v).$ Argue that $\norm{C} \leq 1/2$. For the last part, you can use without proof that $\norm{C} = \sup_{\norm{v}\leq 1} |\braket{v}{Cv}|$ (this follows from the fact that $C$ is Hermitian).
\partbreak
\begin{solution}

    I could not get a satisfying answer to this. I will provide all that I have gotten.
    \tightalignbreak
    \begin{align*}
        &\frac{1}{4}|\braket{v}{Cv}|^2 = \frac{1}{4}\braket{v}{Cv}\braket{v}{C\star v} &\text{(Given.)}\\
        &= \frac{1}{4}\braket{v}{i(PQ - QP)v}\braket{v}{-i(PQ-QP)v} &\text{(Given.)}\\
        &= (\braket{v}{PQv} - \braket{v}{QPv})(\braket{v}{PQv} - \braket{v}{QPv})&\text{(Expanding.)}\\
        &= \braket{v}{PQv}^2 + \braket{v}{QPv}^2 - 2\braket{v}{QPv}\braket{v}{PQv} &\text{(Simplifying.)}
    \end{align*}\vspace{-12mm}\alignbreak
    
    Here is where I am stuck. I could take the approach that $\range QP \subseteq \range P$, and take a bound involving the first two terms, but I would then get stuck on the last term, involving the cross projections.  
\end{solution}
\end{document}