\documentclass[12pt]{article}
\usepackage[paper=letterpaper,margin=1.5cm]{geometry}
\usepackage{amsmath}
\usepackage{amssymb}
\usepackage{amsfonts}
\usepackage{mathtools}

\usepackage{lmodern}     % set math font to Latin modern math
\usepackage[T1]{fontenc}
\renewcommand\rmdefault{ptm}
\usepackage[shortlabels]{enumitem}
\usepackage{titling}
\usepackage{graphicx}
\usepackage[colorlinks=true]{hyperref}
\usepackage{setspace}
\usepackage{subfigure} 
\usepackage{braket}
\usepackage{color}
\usepackage{tabularx}
\usepackage[table]{xcolor}
\usepackage{listings}
\usepackage{mathrsfs}
\usepackage{stackengine}
\usepackage{physics}
\usepackage{afterpage}
\usepackage{tikz}
\usepackage{pdfpages}
\usepackage[export]{adjustbox}
\usepackage{biblatex}

\setstackEOL{\\}

\definecolor{dkgreen}{rgb}{0,0.6,0}
\definecolor{gray}{rgb}{0.5,0.5,0.5}
\definecolor{mauve}{rgb}{0.58,0,0.82}


\lstset{frame=tb,
  language=Python,
  aboveskip=3mm,
  belowskip=3mm,
  showstringspaces=false,
  columns=flexible,
  basicstyle={\small\ttfamily},
  numbers=none,
  numberstyle=\tiny\color{gray},
  keywordstyle=\color{blue},
  commentstyle=\color{dkgreen},
  stringstyle=\color{mauve},
  breaklines=true,
  breakatwhitespace=true,
  tabsize=3
}
\setlength{\droptitle}{-6em}

\makeatletter
% we use \prefix@<level> only if it is defined
\renewcommand{\@seccntformat}[1]{%
  \ifcsname prefix@#1\endcsname
    \csname prefix@#1\endcsname
  \else
    \csname the#1\endcsname\quad
  \fi}
% define \prefix@section
\newcommand\prefix@section{}
\newcommand{\prefix@subsection}{}
\newcommand{\prefix@subsubsection}{}
\renewcommand{\thesubsection}{\arabic{subsection}}
\makeatother
\DeclareMathOperator*{\argmin}{argmin}
\newcommand{\partbreak}{\begin{center}\rule{17.5cm}{2pt}\end{center}}
\newcommand{\alignbreak}{\begin{center}\rule{15cm}{1pt}\end{center}}
\newcommand{\tightalignbreak}{\vspace{-5mm}\alignbreak\vspace{-5mm}}
\newcommand{\hop}{\vspace{1mm}}
\newcommand{\jump}{\vspace{5mm}}
\newcommand{\R}{\mathbb{R}}
\newcommand{\C}{\mathbb{C}}
\newcommand{\N}{\mathbb{N}}
\newcommand{\G}{\mathbb{G}}
\renewcommand{\S}{\mathbb{S}}
\newcommand{\bt}{\textbf}
\newcommand{\xdot}{\dot{x}}
\newcommand{\ydot}{\dot{y}}
\newcommand{\lm}{\mathrm{\lambda}}
\renewcommand{\th}{\theta}
\newcommand{\id}{\mathbb{I}}
\newcommand{\si}{\Sigma}
\newcommand{\Si}{\si}
\newcommand{\inv}{^{-1}}
\newcommand{\T}{^{\intercal}}
\renewcommand{\tr}{\text{tr}}
\newcommand{\ep}{\varepsilon}
\newcommand{\ph}{\varphi}
\newcommand{\range}{\text{range}}
\newcommand{\scP}{\mathcal{P}}
\newcommand{\scT}{\mathbb{T}}
\newcommand{\into}{\rightarrow}
\newcommand{\scM}{\mathcal{M}}
\newcommand{\scH}{\mathcal{H}}
\newcommand{\scN}{\mathcal{N}}
\newcommand{\scV}{\mathcal{V}}
\newcommand{\scW}{\mathcal{W}}
\renewcommand{\grad}{\nabla}
\renewcommand{\star}{^{*}}

\definecolor{cit}{rgb}{0.05,0.2,0.45}
\addtolength{\jot}{1em}
\newcommand{\solution}[1]{


\noindent{\color{cit}\textbf{Solution:} #1}}

\newcounter{tmpctr}
\newcommand\fancyRoman[1]{%
  \setcounter{tmpctr}{#1}%
  \setbox0=\hbox{\kern0.3pt\textsf{\Roman{tmpctr}}}%
  \setstackgap{S}{-.9pt}%
  \Shortstack{\rule{\dimexpr\wd0+.1ex}{.9pt}\\\copy0\\
              \rule{\dimexpr\wd0+.1ex}{.9pt}}%
}

\newcommand{\Id}{\fancyRoman{2}}

% Enter the specific assignment number and topic of that assignment below, and replace "Your Name" with your actual name.
\title{STAT 31210: Homework 5}
\author{Caleb Derrickson}
\date{February 8, 2024}

\begin{document}
\onehalfspacing
\maketitle
\allowdisplaybreaks
{\color{cit}\vspace{2mm}\noindent\textbf{Collaborators:}} The TA's of the class, as well as Kevin Hefner, and Alexander Cram.

\tableofcontents

\newpage
\section{Problem 12.6}
Use the Dominated convergence Theorem to prove Corollary 12.36 for differentiation under an integral sign. 

\newpage
\section{Problem 12.8}
Let $f_n : X \into \C$. be a sequence of measurable functions converging to $f$ pointwise almost everywhere. Suppose there exists $g \in L^p(X)$ such that $|f_n| \leq g$ almost everywhere. Then $f_n \into f$ in the $L^p$-norm. 

\newpage
\section{Problem 12.12}
Prove the following generalization of H\"older's inequality: if $1 \leq p_i \leq \infty$, where $i = 1, ..., n$ satisfy
\[\sum_{i = 1}^n \frac{1}{p_i} = 1\]
and $f_i \in L^{p_i}(X, \mu)$, then $f_1 \cdots f_n \in L^1(X, \mu)$ and 
\[\left| \int f_1 \cdots f_n \ d\mu\right| \leq \norm{f_1}_{p_1} \cdots \norm{f_n}_{p_n}.\]

\newpage
\section{Problem 12.15}
If $f \in L^p (\R^n) \cap L^q(\R^n)$, where $p < q$, prove that $f \in L^r(\R^n)$ for any $p < r < q$, and show that 
\[\norm{f}_r \leq (\norm{f}_p)^{\frac{1 / r - 1 / q}{1 / p - 1 / q}} (\norm{f}_q)^{\frac{1 / p - 1 / r}{1 / p - 1 / q}},\]
This result is one of the simplest examples of an \textit{interpolation inequality}.

\newpage
\section{Problem 12.17}
Prove that the unit ball in $L^p([0, 1])$, where $1 \leq p \leq \infty$, is not strongly compact.

\newpage
\section{Problem 12.18}
Give an example of a bounded sequence in $L^1([0, 1])$ that does not have a weakly convergent subsequence. Why does this not contract the Banach-Analoglu Theorem?

\newpage
\section{Problem 6.2}
Consider $C([0, 1])$ with the sup-norm. Let 
\[N = \left \{ f \in C([0, 1]) : \int_0^1 f(x) \ dx = 0\right\}\]
be the closed linear subspace of $C([0, 1])$ of functions with zero mean. Let 
\[X = \left\{ f \in C([0, 1]) : f(0) = 0\right\}\]
and define $M = N \cap X$. 
\subsection{Problem 6.2, part a}
If $u \in C([0, 1])$, prove that 
\[d(u, N) = \inf_{n \in N} \norm{u - n} = |\overline{u}|\]
where $|\overline{u}| = \int_0^1 u(x) \ dx$ is the mean of $u$, so the infimum is attained when $n = u - \overline{u} \in N$. 

\newpage
\subsection{Problem 6.2, part b}
If $u(x) = x \in X$, show that
\[d(x, M) = \inf_{m \in M}\norm{u - m} = 1/2,\]
but that the infimum is not attained for any $m \in M$.

\newpage
\section{Problem 6.5}
Suppose that $\{H_n : n \in \N\}$ is a set of orthogonal closed subspaces of a Hilbert space $H$. We define the infinite direct sum
\[\bigoplus_{n = 1}^\infty H_n = \left \{ x_n : x_n \in H_n \text{ and } \sum_{n=1}^\infty \norm{x_n}^2 < \infty\right\}.\]
Prove that $\bigoplus_{n = 1}^\infty H_n$ is a closed linear subspace of $H$.

\newpage
\section{Problem 6.11}
Prove that if $M$ is a dense linear subspace of a Hilbert space $H$, then $H$ has an orthonormal basis consisting of elements in $M$. Does the same result hold for arbitrary dense subsets of $H$?

\newpage
\section{Problem 6.14}
Define the Hermite polynomials $H_n$ by 
\[H_n(x) = (-1)^ne^{x^2} \frac{d^n}{dx^n}\left( e^{-x^2}\right).\]
\subsection{Problem 6.14, part a}
Show that 
\[\ph_n(x)  = e^{-x^2/2}H_n(x)\]
is an orthogonal set in $L^2(\R)$.

\newpage
\subsection{Problem 6.14, part b}
show that the $n$-th Hermite function $\ph_n$ is an eigenfunction of the linear operator
\[H = -\frac{d^2}{dx^2} + x^2\]
with eigenvalue $\lm_n = 2n+1$.
\end{document}