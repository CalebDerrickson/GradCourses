\documentclass[12pt]{article}
\usepackage[paper=letterpaper,margin=1.5cm]{geometry}
\usepackage{amsmath}
\usepackage{amssymb}
\usepackage{amsfonts}
\usepackage{mathtools}
%\usepackage[utf8]{inputenc}
%\usepackage{newtxtext, newtxmath}
\usepackage{lmodern}     % set math font to Latin modern math
\usepackage[T1]{fontenc}
\renewcommand\rmdefault{ptm}
%\usepackage{enumitem}
\usepackage[shortlabels]{enumitem}
\usepackage{titling}
\usepackage{graphicx}
\usepackage[colorlinks=true]{hyperref}
\usepackage{setspace}
\usepackage{subfigure} 
\usepackage{braket}
\usepackage{color}
\usepackage{tabularx}
\usepackage[table]{xcolor}
\usepackage{listings}
\usepackage{mathrsfs}
\usepackage{stackengine}
\usepackage{physics}
\usepackage{afterpage}
\usepackage{pdfpages}
\usepackage[export]{adjustbox}
\usepackage{biblatex}

\setstackEOL{\\}

\definecolor{dkgreen}{rgb}{0,0.6,0}
\definecolor{gray}{rgb}{0.5,0.5,0.5}
\definecolor{mauve}{rgb}{0.58,0,0.82}


\lstset{frame=tb,
  language=Python,
  aboveskip=3mm,
  belowskip=3mm,
  showstringspaces=false,
  columns=flexible,
  basicstyle={\small\ttfamily},
  numbers=none,
  numberstyle=\tiny\color{gray},
  keywordstyle=\color{blue},
  commentstyle=\color{dkgreen},
  stringstyle=\color{mauve},
  breaklines=true,
  breakatwhitespace=true,
  tabsize=3
}
\setlength{\droptitle}{-6em}

\makeatletter
% we use \prefix@<level> only if it is defined
\renewcommand{\@seccntformat}[1]{%
  \ifcsname prefix@#1\endcsname
    \csname prefix@#1\endcsname
  \else
    \csname the#1\endcsname\quad
  \fi}
% define \prefix@section
\newcommand\prefix@section{}
\newcommand{\prefix@subsection}{}
\newcommand{\prefix@subsubsection}{}
\renewcommand{\thesubsection}{\arabic{subsection}}
\makeatother
\DeclareMathOperator*{\argmin}{argmin}
\newcommand{\partbreak}{\begin{center}\rule{17.5cm}{2pt}\end{center}}
\newcommand{\alignbreak}{\begin{center}\rule{15cm}{1pt}\end{center}}
\newcommand{\tightalignbreak}{\vspace{-5mm}\alignbreak\vspace{-5mm}}
\newcommand{\hop}{\vspace{1mm}}
\newcommand{\jump}{\vspace{5mm}}
\newcommand{\R}{\mathbb{R}}
\newcommand{\C}{\mathbb{C}}
\newcommand{\N}{\mathbb{N}}
\newcommand{\G}{\mathbb{G}}
\renewcommand{\S}{\mathbb{S}}
\newcommand{\bt}{\textbf}
\newcommand{\xdot}{\dot{x}}
\renewcommand{\star}{^{*}}
\newcommand{\ydot}{\dot{y}}
\newcommand{\lm}{\mathrm{\lambda}}
\renewcommand{\th}{\theta}
\newcommand{\id}{\mathbb{I}}
\newcommand{\si}{\Sigma}
\newcommand{\Si}{\si}
\newcommand{\inv}{^{-1}}
\newcommand{\T}{^\intercal}
\renewcommand{\tr}{\text{tr}}
\newcommand{\ep}{\varepsilon}
\newcommand{\ph}{\varphi}
%\renewcomand{\norm}[1]{\left\lVert#1\right\rVert}
\definecolor{cit}{rgb}{0.05,0.2,0.45}
\addtolength{\jot}{1em}
\newcommand{\solution}[1]{

\noindent{\color{cit}\textbf{Solution:} #1}}

\newcounter{tmpctr}
\newcommand\fancyRoman[1]{%
  \setcounter{tmpctr}{#1}%
  \setbox0=\hbox{\kern0.3pt\textsf{\Roman{tmpctr}}}%
  \setstackgap{S}{-.9pt}%
  \Shortstack{\rule{\dimexpr\wd0+.1ex}{.9pt}\\\copy0\\
              \rule{\dimexpr\wd0+.1ex}{.9pt}}%
}

\newcommand{\Id}{\fancyRoman{2}}

% Enter the specific assignment number and topic of that assignment below, and replace "Your Name" with your actual name.
\title{STAT 31020: Homework 3}
\author{Caleb Derrickson}
\date{January 24, 2024}

\begin{document}
\onehalfspacing
\maketitle
\allowdisplaybreaks
\tableofcontents

\newpage
\section{Problem 1}
Let $f(x): \R^n \rightarrow \R$ be a three times continuously differentiable function and $x\star$ a local minimum of $f(x)$ that satisfies the sufficient second order-conditions. Consider the iteration
\begin{align}
    x_{k+1} = x_k - \left(\grad^2_{xx}f(x_k) + \norm{\grad_x f(x_k)}^p \ \id_n \right)\inv \grad f(x_k).\label{p1: first}
\end{align}
Here, $p > 0$ and $\id_n$ is the identity matrix of order $n$. Assume that you know that $x_k \rightarrow x\star$.
\subsection{Problem 1, part 1}
Show that the sequence $x_k$ converges to $x\star$ faster than superlinearly.
\partbreak
\begin{solution}

    This will be given as a consequence to the next part. 
\end{solution}
\subsection{Problem 1, part 2}
Determine the largest value $q$ such that 
\[\limsup \frac{\norm{x_{k+1} - x\star}}{\norm{x_k - x\star}^q} < \infty\] 
as a function of $p$ for $p > 0$.
\partbreak
\begin{solution}

    I will break this derivation up into sections dedicated to simple steps (which will be bundled together) and other sections which describe a particular step in more detail Note that I will be occasionally short-handing arguments of functions (such as $\grad f(x_k)$ into $\grad f_k$) to condense lines. We first start by adding a $-x\star$ to both sides of \ref{p1: first}. 
    \[\implies x_{k+1} - x\star = x_k - x\star - (\grad^2f_k + \norm{\grad f_k}^p\id_n)\inv \grad f_k.\]
    We will then group up terms by bringing out the inverted matrix on the left hand side.
    \[\implies x_{k+1} - x\star = \left ( \grad^2f_k + \norm{\grad f_k}^p \id_n\right)\inv\left[ \left(\grad^2f_k + \norm{\grad f_k}^p \id_n\right)(x_k - x\star) - \grad f_k\right]\]
    Next, define a helper function $\psi$ as $\psi(t) = \grad f(x\star + t(x_k - x\star))$. Note that $\psi(1) = \grad f_k$ and $\psi(0) = \grad f(x\star)$. Since $x\star$ is a strict local minimum of $f$, $\grad f(x\star) = 0$, so we can include this in whenever we wish. We can therefore say that $\grad f_k = \psi(1) - \psi(0)$, which by the Fundamental Theorem of Calculus,\footnote{This is well defined since the derivative of $\psi(t)$ is well defined. } 
    \[\grad f_k = \psi(1) - \psi(0) =\int_0^1 \psi '(t) \ dt = \int_0^1 \grad^2 f(x\star + t(x_k - x\star)) (x_k - x\star) \ dt.\]
    We will then do some intermediate steps below.
{\footnotesize
    \begin{align*}
        x_{k+1} - x\star &= \left ( \grad^2f_k + \norm{\grad f_k}^p \id_n\right)\inv\left[ \left(\grad^2f_k + \norm{\grad f_k}^p \id_n\right)(x_k - x\star) - \grad f_k\right] \\
         &= \left ( \grad^2f_k + \norm{\grad f_k}^p \id_n\right)\inv\left[ \left(\grad^2f_k + \norm{\grad f_k}^p \id_n\right)(x_k - x\star) - \int_0^1 \grad^2 f(x\star + t(x_k - x\star)) (x_k - x\star) \ dt\right]\\ 
         &= \left ( \grad^2f_k + \norm{\grad f_k}^p \id_n\right)\inv\left[ \int_0^1 \left[ \grad^2 f_k - \grad^2f(x\star + t(x_k - x\star)) + \norm{\grad f_k}^p\right](x_k - x\star) \ dt\right]\\ 
    \end{align*}}
    \vspace{-20mm}
    
    The steps above were just some simple rearranging and adding in the helper function. We can then take norms on both sides to get
    \begin{align}
        \norm{x_{k+1} - x\star} = \norm{\left ( \grad^2f_k + \norm{\grad f_k}^p \id_n\right)\inv\left[ \int_0^1 \left[ \grad^2 f_k - \grad^2f(x\star + t(x_k - x\star)) + \norm{\grad f_k}^p\right](x_k - x\star) \ dt\right]}.\label{p1:star}
    \end{align}
    From here we will be taking advantage of three notable things:
    \begin{enumerate}[(a)]
        \item We wish to find an upper bound to $\norm{\left ( \grad^2f_k + \norm{\grad f_k}^p \id_n\right)\inv}$. Note that for any neighborhood of the local minimum $x\star$, $\lm_{\min}(\grad^2f(x)) \geq \frac{1}{2} \lm_{\min}(\grad^2f(x\star))$ for any $x$ in that neighborhood. Furthermore, we can take $\left( \grad^2f_k + \norm{\grad f_k}^p \id_n\right)\inv \succ 0$, since its inverse is positive definite.\footnote{Or within the region such that It can be bounded below by $\grad^2f(x\star)$, which is positive definite by the second sufficient condition.} The norm of this matrix is then its largest eigenvalue (assuming we're in the $2$ norm). So that
        \[\norm{\left( \grad^2f_k + \norm{\grad f_k}^p \id_n\right)\inv} = \frac{1}{\lm_{\min}(\grad^2f_k) + \norm{\grad f_k}^p}.\]
        We can then again bound this ratio by removing the additional term, to get 
        \[\norm{\left( \grad^2f_k + \norm{\grad f_k}^p \id_n\right)\inv} \leq \frac{1}{\lm_{\min}(\grad^2 f_k)}.\]
        By the argument above, this can then be bounded again to get
        \[\norm{\left( \grad^2f_k + \norm{\grad f_k}^p \id_n\right)\inv} \leq \frac{2}{\lm_{\min}(\grad^2f_k)} = 2\norm{\grad^2f(x\star)\inv}.\]
        \item We can note that 
        \[\norm{\int \psi \ dt} \leq \int \norm{\psi} \ dt\]
        for any integrable function $\psi$. 
        \item We will take the norm to be consistent. That is, for a matrix $A$ and suitable vector $b$, $\norm{Ab} \leq \norm{A}\norm{b}$.
    \end{enumerate}
    
    \newpage
    We will then apply $(c)$ to \ref{p1:star} to get 
    \[\norm{x_{k+1} - x\star} \leq \norm{\left ( \grad^2f_k + \norm{\grad f_k}^p \id_n\right)\inv}\norm{\left[ \int_0^1 \left[ \grad^2 f_k - \grad^2f(x\star + t(x_k - x\star)) + \norm{\grad f_k}^p\id_n\right](x_k - x\star) \ dt\right]}.\]
    Then, applying $(a)$ and $(b)$, 
    \[\norm{x_{k+1} - x\star} \leq 2\norm{\grad^2f(x\star)\inv} \int_0^1 \left[ \norm{\grad^2 f_k - \grad^2f(x\star + t(x_k - x\star)) + \norm{\grad f_k}^p\id_n}\norm{(x_k - x\star)} \right] \ dt.\]
    We can then successively bound $\norm{x_{k+1} - x\star}$ via the following steps.
    \begin{align*}
        \norm{x_{k+1} - x\star} &\leq 2\norm{\grad^2f(x\star)\inv}\int_0^1 \left[ \norm{\grad^2f_k + \norm{\grad f_k}^p\id_n - \grad^2 f(x\star + t(x_k - x\star))}\right]\norm{x_k - x\star} \ dt \\
        &\leq 2\norm{\grad^2f(x\star)\inv}\int_0^1 \left[ \norm{\grad^2f_k - \grad^2 f(x\star + t(x_k - x\star))}  + \norm{\grad f_k}^p\right]\norm{x_k - x\star} \ dt \\
        &\leq 2\norm{\grad^2f(x\star)\inv}\int_0^1 \left[ L(1 - t)\norm{x_k - x\star}  + \norm{\grad f_k}^p\right]\norm{x_k - x\star} \ dt \\
        &= 2\norm{\grad^2f(x\star)\inv}\left[ \frac{L}{2}\norm{x_k - x\star}^2  + \norm{\grad f_k}^p\norm{x_k - x\star}\right] \\
        &= 2\norm{\grad^2f(x\star)\inv}\left[ \frac{L}{2}\norm{x_k - x\star}^2  + \norm{\grad f(x_k) - \grad f(x\star)}^p\norm{x_k - x\star}\right] \\
        &\leq 2\norm{\grad^2f(x\star)\inv}\left[ \frac{L}{2}\norm{x_k - x\star}^2  + (L')^p\norm{x_k - x\star}^{(1+p)}\right] \\
        &=  L\norm{\grad^2f(x\star)\inv}\norm{x_k - x\star}^2  + 2(L')^p\norm{\grad^2f(x\star)\inv}\norm{x_k - x\star}^{(1+p)}
    \end{align*}
    Note that $\grad^2 f$ is deferentially continuous and its derivative is bounded, thus $\grad^2 f$ is continuously Lipschitz. This holds as well for $\grad f$. Denote their Lipschitz constants as $L$ and $L'$, respectively. To group the terms together, define a constant $C = \max \{ L, 2(L')^p\}$. We then get
    \[\norm{x_{k+1} - x\star} \leq C\norm{\grad^2f(x\star)\inv}\left[ \norm{x_k - x\star}^2 + \norm{x_k - x\star}^{(1 + p)}\right].\]
    
    \newpage
    This gives us a fair enough bound for lower values of $k$, but for sufficiently large $k$, we will see one of these terms dominate. In particular, we will see the term which is raised by the least exponent to dominate. Thus, we can effectively ignore the term being raised to the larger power for sufficiently large $k$. \footnote{This can also be interpreted as taking the limsup of the function.} If we define $q = \min \{ 2, 1 + p \}$, then we can see that 
    \[\frac{\norm{x_{k+1} - x\star}}{\norm{x_k - x\star}^q} \leq C\norm{\grad^2f(x\star)\inv}\]
    for all $k$. Then taking the limit as $k$ goes to $\infty$, we see that this ratio is bounded above by some constant. Thus, the iteration converges faster than superlinearly for any value $p > 0$ (satisfying part 1), and the largest $q$ value is 2.
\end{solution}

\newpage
\subsection{Problem 1, part 3}
Repeat now the problem for the iteration
\[x_{k+1} = x_k - \left( \grad^2_{xx}f(x_k) + \grad f(x_k) \grad f(x_k)\T + \norm{\grad_x f(x_k)}^p \id_n\right)\inv \grad f(x_k)\]
\partbreak
\begin{solution}

    This is almost the same derivation as above. I will stop when something is meaningfully changed below.
{\footnotesize
    \begin{align*}
        &x_{k+1} = x_k - \left( \grad^2_{xx}f(x_k) + \grad f(x_k) \grad f(x_k)\T + \norm{\grad_x f(x_k)}^p \id_n\right)\inv \grad f(x_k)\\
        &x_{k+1} - x\star = x_k -x\star - \left( \grad^2_{xx}f(x_k) + \grad f(x_k) \grad f(x_k)\T + \norm{\grad_x f(x_k)}^p \id_n\right)\inv \grad f(x_k)\\
        &= \left( \grad^2_{xx}f(x_k) + \grad f(x_k) \grad f(x_k)\T + \norm{\grad_x f(x_k)}^p \id_n\right)\inv \left[\left( \grad^2_{xx}f(x_k) + \grad f(x_k) \grad f(x_k)\T + \norm{\grad_x f(x_k)}^p \id_n\right)(x_k -x\star) - \grad f(x_k)\right]\\
        &= \left( A\right)\inv \left[\left( \grad^2_{xx}f(x_k) + \grad f(x_k) \grad f(x_k)\T + \norm{\grad_x f(x_k)}^p \id_n\right)(x_k -x\star) - \int_0^1\grad^2 f(x\star + t(x_k - x\star))(x_k - x\star) \ dt\right]\\
        &= \left( A\right)\inv \left[ \int_0^1 \left( \grad^2_{xx}f(x_k) + \grad f(x_k) \grad f(x_k)\T + \norm{\grad_x f(x_k)}^p \id_n + \grad^2 f(x\star + t(x_k - x\star))\right)(x_k - x\star) \ dt\right]\\
        &\implies \norm{x_{k+1} - x\star}=\\
        &\norm{\left( A\right)\inv \left[ \int_0^1 \left( \grad^2_{xx}f(x_k) + \grad f(x_k) \grad f(x_k)\T + \norm{\grad_x f(x_k)}^p \id_n + \grad^2 f(x\star + t(x_k - x\star))\right)(x_k - x\star) \ dt\right]}\\
        &\leq \norm{A\inv} \int_0^1 \left[ \norm{\grad^2f(x_k) - \grad^2f(x\star + t(x_k - x\star))} + \norm{\grad f_k \grad f_k\T} + \norm{\grad f_k}^p \ dt \right]\norm{x_k - x\star}\\
        &\leq \norm{A\inv} \int_0^1 \left[ L \norm{x_k - x\star}(1 - t) + \norm{\grad f_k }\norm{\grad f_k } + \norm{\grad f_k}^p \ dt \right]\norm{x_k - x\star}\\
        &\leq \norm{A\inv}  \left[ \frac{L}{2} \norm{x_k - x\star} + \norm{\grad f_k}^2 + \norm{\grad f_k}^p\right]\norm{x_k - x\star}\\
        &= \norm{A\inv}  \left[ \frac{L}{2} \norm{x_k - x\star} + \norm{\grad f(x_k) - \grad f(x\star)}^2 + \norm{\grad f(x_k) - \grad f(x\star)}^p\right]\norm{x_k - x\star}\\
        &\leq \norm{A\inv}  \left[ \frac{L}{2} \norm{x_k - x\star} + (L')^2\norm{x_k - x\star}^2 + (L')^p\norm{x_k - x\star}^p\right]\norm{x_k - x\star}\\
        &= \norm{A\inv}  \left[ \frac{L}{2} \norm{x_k - x\star}^2 + (L')^2\norm{x_k - x\star}^3 + (L')^p\norm{x_k - x\star}^{(1+p)}\right]\\ 
    \end{align*}
}
    \newpage
    The derivation requires more than the width of the page to write in its entirety. Because of this, I substituted $A$ as the inverted norm on the left of the integral. This term will be considered below. By the same reasoning from the previous part, we will take
    \[\norm{A\inv} = \frac{1}{\lm_{\min}(\grad^2f_k) + \lm_{\min} (\grad f_k \grad f_k\T ) + \norm{\grad f_k}^p}.\]
    This \textit{might} not be the correct interpretation or the max eigenvalue of $\norm{A\inv}$, because It might not be the case that we are taking the smallest eigenvalues of both matrices. Either way, it doesn't matter since we will bound it like we did in the last part. Note that $\grad f_k \grad f_k\T$ is a rank one matrix, so its minimum eigenvalue is zero \footnote{If we are in two dimensions or higher.}, so this can be rewritten as 
    \[\norm{A\inv} = \frac{1}{\lm_{\min}(\grad^2f_k) + \norm{\grad f_k}^p}.\]
    This is the same as in the last part, which we found to be bounded by $2\norm{\grad f(x\star)\inv}$. Substituting this in, we get
    \[\norm{x_{k+1} - x\star} \leq \norm{\grad f(x\star)\inv} \left[ L \norm{x_k - x\star}^2 + 2(L')^2\norm{x_k - x\star}^3 + 2(L')^p\norm{x_k - x\star}^{(1+p)}\right].\]
    Take $C = \max \{ L, 2(L')^2, 2(L')^p \}$. Then,
    \[\norm{x_{k+1} - x\star} \leq \norm{\grad f(x\star)\inv} C\left[ \norm{x_k - x\star}^2 + \norm{x_k - x\star}^3 + \norm{x_k - x\star}^{(1+p)}\right].\]
    This will again be dominated by the least power term, so take $q = \min \{ 2, 3, 1 + p\} = \min \{ 2, 1 + p\}$. Note that $p > 0$, so the limit as $k$ goes to $\infty$ is zero, which means the sequence converges more than superlinearly. The fastest it can converge is superquadratically, or when $q = 2$.
\end{solution}

\newpage
\section{Problem 2}
I will try to "walk you" through a justification for the equations 4.8 in Theorem 4.1 that we will use when we characterize the solution of the trust region problem. The approach I used can be used to prove the optimality conditions for constrained optimization, using the barrier method- which is intrinsic to a class of algorithms called interior point methods. 

\jump
Under the conditions of Theorem 4.1 in the 2013 (?) textbook edition, consider the problem $ \\ \min_{p \in \R^n} m(p, \mu) := m(p) - \mu \log (\Delta^2 - p\T p),$ where $m(p) := f + g\T p + \frac{1}{2}p\T Bp$ is a quadratic function model. Observe that the domain of $m(p, \mu)$ is the interior of the trust region.
\subsection{Problem 2, part a}
Show that, for any $0 < \mu \leq 1$, you have that $m(p, \mu)$ is bounded below in its domain, uniformly with respect to $\mu$.
\partbreak
\begin{solution}

    Note the domain of $m(p, \mu)$ is restricted on the values taken by $\Delta ^2 - p\T p$. In particular, since we don't want the log to blow up, we want $\Delta^2 - p\T p > 0$, in particular, we want $\norm{p} < \Delta$. FINISH THIS ONE.
\end{solution}

\newpage
\subsection{Problem 2, part b}
Conclude that $m(p, \mu)$ must have a global minimum. denote that point by $p(\mu)$ if there is more than one such point, choose one of them at random.
\partbreak
\begin{solution}
    
    Since $p(p, \mu)$ is bounded below on its domain, we have that there must be some finite minimum value of $m$ achieved by some $p\star$. Since the additional log term is strictly increasing, this would not affect the number of minimums achieved by the function $m(p)$. Since $m(p)$ is given as convex, there is only one global minimum, which we will denote by $p\star$. Therefore, the global minimum is unique. 
\end{solution}
\subsection{Problem 2, part c}
Prove the following lemma: If A is a matrix of full column rank, then $b = A\lm \implies \norm{\lm} \leq \frac{\norm{b}}{\si_m}$, where $\sigma_m$ is the smallest singular value of A, and the norms used are 2 norms.
\partbreak
\begin{solution}

    By consistency of the 2-norm, we have that $\norm{A\lm}_2 \leq \norm{A}_2\norm{\lm}_2$. Thus $\norm{b}_2 = \norm{A\lm}_2 \leq \norm{A}_2 \norm{\lm}_2$. This implies $\norm{\lm}_2 \leq \frac{\norm{b}_2}{\norm{A}_2}$. Note that $\norm{A}_2$ is the largest singular value of $A$, which has reciprocal less than or equal to the smallest singular value (assuming it is nonzero). Therefore,  $\norm{\lm}_2 \leq \frac{\norm{b}_2}{\sigma_m}$, which is what we wanted. 
\end{solution}
\subsection{Problem 2, part d}
Conclude that, when writing the optimality conditions of the problem stated, you must have $\grad_p m(p) + \lm(\mu)p = 0$ at the global minimum. Use part $(c)$ to chow that $\lm(\mu)$ must be uniformly bounded in $\mu$ (above).
\partbreak
\begin{solution}

    By the first optimality condition, $\lm(\mu)p = -g - Bp$. By taking the gradient of $m(p)$, we can see that the right hand side is equal to $-\grad_p m(p)$. Then $\lm(\mu)p = -\grad_p m(p)$. Therefore, $\lm(\mu)p + \grad_p m(p)= 0$. To show that $\lm$ is bounded above, we take norms on the first optimality condition, and rewrite it in the same form as in part $(c)$ to get
    \[\norm{p} \leq \frac{\norm{g}}{\sigma_{m}(B + \lm(\mu)\id )}\]
    By the third optimality condition, we have that the matrix $B + \lm(\mu)\id$ is positive semidefinite. If the smallest singular value is zero, then we have a problem. So we will just look at the case when the smallest singular value is not zero. Note that $\lm \geq 0$, so when evaluating the smallest singular value of $B + \lm \id$, we can look at them individually. This means
    \[\norm{p} \leq \frac{\norm{g}}{\sigma_{m}(B) + \lm(\mu) }\]
    Then rearranging for $\lm$, we get
    \[\lm(\mu) \leq \frac{\norm{g}}{\norm{p}}\]
    So $\lm$ is bounded above. Note that if $\norm{p} = 0$, then $\grad_p m(p = 0) + 0 = 0$. This means the gradient is zero, which is a contradiction.
\end{solution}
\end{document}