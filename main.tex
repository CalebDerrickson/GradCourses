\documentclass[12pt]{article}
\usepackage[paper=letterpaper,margin=1.5cm]{geometry}
\usepackage{amsmath}
\usepackage{amssymb}
\usepackage{amsfonts}
\usepackage{mathtools}
%\usepackage[utf8]{inputenc}
%\usepackage{newtxtext, newtxmath}
\usepackage{lmodern}     % set math font to Latin modern math
\usepackage[T1]{fontenc}
\renewcommand\rmdefault{ptm}
%\usepackage{enumitem}
\usepackage[shortlabels]{enumitem}
\usepackage{titling}
\usepackage{graphicx}
\usepackage[colorlinks=true]{hyperref}
\usepackage{setspace}
\usepackage{subfigure} 
\usepackage{braket}
\usepackage{color}
\usepackage{tabularx}
\usepackage[table]{xcolor}
\usepackage{listings}
\usepackage{mathrsfs}
\usepackage{stackengine}
\usepackage{physics}
\usepackage{afterpage}
\usepackage{pdfpages}
\usepackage[export]{adjustbox}
\usepackage{biblatex}

\setstackEOL{\\}

\definecolor{dkgreen}{rgb}{0,0.6,0}
\definecolor{gray}{rgb}{0.5,0.5,0.5}
\definecolor{mauve}{rgb}{0.58,0,0.82}


\lstset{frame=tb,
  language=Python,
  aboveskip=3mm,
  belowskip=3mm,
  showstringspaces=false,
  columns=flexible,
  basicstyle={\small\ttfamily},
  numbers=none,
  numberstyle=\tiny\color{gray},
  keywordstyle=\color{blue},
  commentstyle=\color{dkgreen},
  stringstyle=\color{mauve},
  breaklines=true,
  breakatwhitespace=true,
  tabsize=3
}
\setlength{\droptitle}{-6em}

\makeatletter
% we use \prefix@<level> only if it is defined
\renewcommand{\@seccntformat}[1]{%
  \ifcsname prefix@#1\endcsname
    \csname prefix@#1\endcsname
  \else
    \csname the#1\endcsname\quad
  \fi}
% define \prefix@section
\newcommand\prefix@section{}
\newcommand{\prefix@subsection}{}
\newcommand{\prefix@subsubsection}{}
\renewcommand{\thesubsection}{\arabic{subsection}}
\makeatother
\DeclareMathOperator*{\argmin}{argmin}
\newcommand{\partbreak}{\begin{center}\rule{17.5cm}{2pt}\end{center}}
\newcommand{\alignbreak}{\begin{center}\rule{15cm}{1pt}\end{center}}
\newcommand{\tightalignbreak}{\vspace{-5mm}\alignbreak\vspace{-5mm}}
\newcommand{\hop}{\vspace{1mm}}
\newcommand{\jump}{\vspace{5mm}}
\newcommand{\R}{\mathbb{R}}
\newcommand{\C}{\mathbb{C}}
\newcommand{\N}{\mathbb{N}}
\newcommand{\G}{\mathbb{G}}
\renewcommand{\S}{\mathbb{S}}
\newcommand{\bt}{\textbf}
\newcommand{\xdot}{\dot{x}}
\renewcommand{\star}{^{*}}
\newcommand{\ydot}{\dot{y}}
\newcommand{\lm}{\mathrm{\lambda}}
\renewcommand{\th}{\theta}
\newcommand{\id}{\mathbb{I}}
\newcommand{\si}{\Sigma}
\newcommand{\Si}{\si}
\newcommand{\inv}{^{-1}}
\newcommand{\T}{^\intercal}
\renewcommand{\tr}{\text{tr}}
\newcommand{\ep}{\varepsilon}
\newcommand{\ph}{\varphi}
%\renewcomand{\norm}[1]{\left\lVert#1\right\rVert}
\definecolor{cit}{rgb}{0.05,0.2,0.45}
\addtolength{\jot}{1em}
\newcommand{\solution}[1]{

\noindent{\color{cit}\textbf{Solution:} #1}}

\newcounter{tmpctr}
\newcommand\fancyRoman[1]{%
  \setcounter{tmpctr}{#1}%
  \setbox0=\hbox{\kern0.3pt\textsf{\Roman{tmpctr}}}%
  \setstackgap{S}{-.9pt}%
  \Shortstack{\rule{\dimexpr\wd0+.1ex}{.9pt}\\\copy0\\
              \rule{\dimexpr\wd0+.1ex}{.9pt}}%
}

\newcommand{\Id}{\fancyRoman{2}}

% Enter the specific assignment number and topic of that assignment below, and replace "Your Name" with your actual name.
\title{STAT 31210: Homework 7}
\author{Caleb Derrickson}
\date{February 23, 2024}

\begin{document}
\onehalfspacing
\maketitle
\allowdisplaybreaks
{\color{cit}\vspace{2mm}\noindent\textbf{Collaborators:}} The TA's of the class, as well as Kevin Hefner, and Alexander Cram.

\tableofcontents

\newpage
\section{Problem 8.12}
Suppose that $A: \scH \into \scH$ is a bounded, self-adjoint, linear operator such that there is a constant $c > 0$ with 
\[c \norm{x} \leq \norm{Ax} \quad \text{ for all } x \in \scH.\]
Prove that there is a unique solution $x$ of the equation $Ax = y$ for every $y \in \scH$.
\partbreak
\begin{solution}

    For this proof, we have to show two things: that there is a solution for all $x \in \scH$ and that solution is unique. First, we wish to show that $\range (A)$ is closed, which we can then apply Theorem 8.18, which will give us that $Ax$ has a solution for $y$ orthogonal to $\ker (A\star)$, which is $\ker(A)$ since $A$ is self-adjoint. To show $\range (A)$ is closed, take a Cauchy sequence $Ax_n \in\scH$. Then, by the property given of $A$, we have that 
    \[\norm{x_n - x_m} \leq \frac{1}{c}\norm{Ax_n - Ax_m}\]
    Since $Ax_n$ is Cauchy, we have that $\norm{x_n - x_m} \into 0$ as $n, m \into \infty$. Therefore, $x_n$ is a Cauchy sequence in $\scH$. Since $\scH$ is complete, then $x_n \into x \in \scH$, as $n \into \infty$. Note then that
    \[\norm{x_n - x} = \frac{\norm{A}}{\norm{A}}\norm{x_n - x} \geq \frac{1}{\norm{A}}\norm{Ax_n - Ax}\]
    The last step is an application of Cauchy Schwartz. The multiplication and division of the norm of $A$ can be done since $A$ is bounded (and assumed not the $0$-matrix)\footnote{Even if we assumed that $A$ could be the zero-matrix, then the solutions of $Ax$ would not be unique, therefore we exclude this case.}. The line above then implies that $\norm{Ax_n - Ax} \into 0$ as $n \into \infty$, since it is bounded by $\norm{x_n - x}$. Therefore, $x\in \range (A)$, so it is closed. By Theorem 8.18, we have that $Ax = y$ as solutions for all $y$ orthogonal to $\ker (A)$, since $A$ is self adjoint. I claim that $\ker (A) = \{0\}$, since if it weren't, then there would exist some $x \in \scH$ nonzero such that $Ax = 0$. But, 
    \[\norm{x} \leq c\norm{Ax} = 0 \implies \norm{x} = 0 \iff x = 0.\]
    Therfore, we see a contradiction on $x$, so $\ker (A) = \{0\}$. Therefore, $Ax = y$ has a solution for all $y \in \scH$, since all of $\scH$ is orthogonal to the zero set. Finally, we need to show that the solution is unique for all $x \in \scH$. Suppose there were $x_1, x_2 \in \scH$ such that $Ax_1 = Ax_2 = y$. But then $A(x_1 - x_2) = 0$, which implies that $x_1 - x_2 = 0$, since the kernel of $A$ is the zero set. Therefore, $x_1 = x_2$, so the solution is unique for all $x \in \scH$, proving the statement. 
\end{solution}

\newpage
\section{Problem 8.13}
Prove that an orthogonal set of vectors $\{u_\alpha : \alpha \in \scA \}$ in a Hilbert space $\scH$ is an orthonormal basis if and only if 
\[\sum_{\alpha \in \scA} u_\alpha \otimes u_\alpha = \id. \]

\newpage
\section{Problem 8.14}
Suppose that $A, B \in \mfB (\scH)$ satisfy
\[\braket{x}{Ay} = \braket{x}{By} \quad \text{ for all } x, y \in \scH.\]
Prove that $A = B$. Use a polarization type identity to prove that if $\scH$ is a complex Hilbert space and 
\[\braket{x}{Ax} = \braket{x}{Bx} \quad \text{ for all } x\in \scH,\]
then $A = B$. What can you say about $A$ and $B$ for real Hilbert spaces?

\newpage
\section{Problem 8.17}
Prove that strong convergence implies weak convergence. Also prove that strong and weak convergence are equivalent in a finite-dimensional Hilbert space.

\newpage
\section{Problem 8.18}
Let $u_n$ be a sequence of orthonormal vectors in a Hilbert space. Prove that $u_n \rightharpoonup 0$ weakly.  


\newpage
\section{Problem 8.20}
\newcommand{\xbar}{\Bar{x}}
Let $\scH$ be a real Hilbert space and $\ph \in \scH\star$. Define the quadratic functional $f: \scH \into \R$ by 
\[f(x) = \frac{1}{2}\norm{x}^2 - \ph(y).\]
Prove that there is a unique element $\xbar \in \scH$ such that 
\[f(\xbar) = \inf_{x \in \scH} f(x).\]

\end{document}