\documentclass[12pt]{article}
\usepackage[paper=letterpaper,margin=1.5cm]{geometry}
\usepackage{amsmath}
\usepackage{amssymb}
\usepackage{amsfonts}
\usepackage{mathtools}
\usepackage[utf8]{inputenc}
\usepackage{newtxtext, newtxmath}
\usepackage{enumitem}
\usepackage{titling}
\usepackage{graphicx}
\usepackage[colorlinks=true]{hyperref}
\usepackage{setspace}
\usepackage{braket}
\usepackage{color}
\usepackage{listings}

\definecolor{dkgreen}{rgb}{0,0.6,0}
\definecolor{gray}{rgb}{0.5,0.5,0.5}
\definecolor{mauve}{rgb}{0.58,0,0.82}

\lstset{frame=tb,
  language=Python,
  aboveskip=3mm,
  belowskip=3mm,
  showstringspaces=false,
  columns=flexible,
  basicstyle={\small\ttfamily},
  numbers=none,
  numberstyle=\tiny\color{gray},
  keywordstyle=\color{blue},
  commentstyle=\color{dkgreen},
  stringstyle=\color{mauve},
  breaklines=true,
  breakatwhitespace=true,
  tabsize=3
}
\setlength{\droptitle}{-6em}

\newcommand{\hop}{\vspace{1mm}}
\newcommand{\jump}{\vspace{5mm}}
\newcommand{\R}{\mathbb{R}}
\newcommand{\C}{\mathbb{C}}
\newcommand{\bt}{\textbf}
\newcommand{\lm}{\lambda}
\newcommand{\ep}{\varepsilon}
\definecolor{cit}{rgb}{0.05,0.2,0.45}
\addtolength{\jot}{1em}
\newcommand{\solution}[1]{

\vspace{5mm}
\medskip\noindent{\color{cit}\textbf{Solution:} #1}}

% Enter the specific assignment number and topic of that assignment below, and replace "Your Name" with your actual name.
\title{STAT 31410: Homework 1}
\author{Caleb Derrickson}
\date{October 8, 2023}

\begin{document}
\onehalfspacing
\maketitle

For this problem set, we consider the equation for an idealized pendulum
\begin{align}
    \frac{d^2\theta}{dt^2} = -\frac{g}{\ell(t)}sin(\theta), \label{Original ode}
\end{align}

where the length of the moment arm, $l$, varies periodically according to

\begin{align}
    \ell (t) = \ell_0(1 + \ep \cos(\omega t)), \hspace{5mm} \ep \ll 1.    \label{length approx}
\end{align}

This might be a good model of a child on a swing, trying to pump energy into it by repeatedly
standing up and then squatting back down with some appropriate rhythm, to be determined.

\jump
\centerline{\noindent}%
\makebox[\textwidth]{
\includegraphics[scale = 0.6]{Images/swing.PNG}
}
\jump

\begin{enumerate}[]
    \item Make an approximation where you keep only the leading order term in $\ep$, i.e linearize about $\varepsilon$ = 0. Then choose a non-dimensionalization of time that puts (1) into this form:

    \begin{align}
        \ddot{\theta} = -(\alpha + \beta \cos(\tau))\sin(\theta),   \label{linearish}
    \end{align}

        where you define the dimensionless parameters $\alpha$ and $\beta$ in terms of the original parameters $g$, $\ell_0$, $\varepsilon$ and $\omega$. Re-write the second order equation \ref{linearish} as two first order equations for the angle $\theta$ and the angular speed $\omega \equiv \dot{\theta}$. For $\beta$ = 0, derive a conserved quantity, the energy of the pendulum, denoted $E(\theta, \Omega)$. Write an equation for $\dot{E}$ that applies if $\beta \neq 0$.

        {\color{cit}\vspace{2mm}\noindent\textbf{Collaborators:}} The TA's of the class, as well as Kevin Hefner, Nathan Suhr, and Steven Lee.
        \begin{solution}

        Since we are taking $\ep \ll 1$ from \ref{Original ode}, we can define a function $f(\ep)$ on which we will expand using ordinary Taylor expansion. Here, I will define $f(\ep)$ as:
        \begin{align}
            f(\ep) = \frac{1}{1 + \ep \cos(\omega t)},
        \end{align}

        where it is understood that $\cos(\omega t)$ is a constant in this sense. Note that we are \textit{linearizing} with respect to $\theta$, so we can drop all terms of higher order. Thus, 

        \begin{align}
            f(\ep) &= \sum_{n = 0}^\infty \ep^n\frac{d^nf}{d\ep^n} \Big\rvert_{\ep = 0} \nonumber \\
            &= f(\ep = 0) + \ep\frac{df}{d\ep} \Big\rvert_{\ep = 0} + O(n^2)  \nonumber \\
            &= 1 - \ep \bigg[ \frac{\cos(\omega t)}{(1 + \ep\cos(\omega t))^2} \bigg]_{\ep = 0}  \nonumber \\
            & = 1 - \ep \cos(\omega t) \label{Expansion} 
        \end{align}

        We can then plug \ref{Expansion} into \ref{Original ode} to obtain,

        \begin{align}
            \ddot{\theta} = -\frac{g}{\ell_0}\Big(1 - \ep \cos(\omega t)\Big)\sin(\theta). \label{before parameters}
        \end{align}

        From \ref{before parameters}, we can define three dimensionless parameters: $\tau, \alpha, \beta$. $\tau$ will take the place of the argument inside the cosine term of \ref{before parameters}, while $\alpha$ and $\beta$ will clean up any extraneous terms. We will first redefine the ratio of $g$ and $\ell_0$ as the resonance frequency of our state, $\omega_0$.
        
        \begin{align*}
            \omega_0^2 = \frac{\ell_0}{g}.    
        \end{align*}

        Then we will take $\tau$ as $\tau = \omega t$. Note that we are taking time derivatives, so this re-parameterization affects what we are differentiating by. 

        \begin{align*}
            \tau = \omega t \implies d\tau = \omega dt \iff \frac{1}{\omega}\frac{d}{d\tau} = \frac{d}{dt}.
        \end{align*}

        Applying these to \ref{before parameters}, we get,

        \begin{align}
            \ddot{\theta} = -\Big(\frac{\omega_0}{\omega} \Big)^2 \big( 1 - \ep \cos(\tau) \big) \sin(\theta).  \label{before alphabeta}
        \end{align}

        We can then define $\alpha, \beta$ as,

        \begin{align}
            \alpha = \bigg(\frac{\omega_0}{\omega}\bigg)^2, \hspace{10mm} \beta = \alpha\ep
        \end{align}

        Note that, as suggested by the TA's, to take $\beta$ as a positive constant. The negative can be recovered by shifting $\tau$ by $\pi$, meaning $\tau = \omega t + \pi$. This shift will change nothing else in our analysis, hopefully. Also, when taking time derivates for the rest of this homework, I am specifically referring to taking the $\tau$ derivative. Thus, our second order differential equation is, 

        \begin{align}
            \ddot{\theta} = -\big(\alpha + \beta \cos(\tau)\big)\sin(\theta),   \label{second order ode with new parameters}
        \end{align}

        matching \ref{linearish}. 

        Our next goal is to re-write \ref{second order ode with new parameters} as two first order equations for $\theta$ and the angular speed $\Omega \equiv \dot{\theta}$. This can be achieved quite easily - our system is now. 
        \begin{equation}
        \begin{cases}
            \dot{\theta} = \Omega, \\
            \dot{\Omega} = -\big(\alpha + \beta \cos(\tau)\big)\sin(\theta) \label{Nonlinear system}
        \end{cases}    
        \end{equation}

        Our next goal is to find a conserved quantity for $\beta = 0$. This quantity will be the energy of our system (or rather, an approximation of it) and will be denoted as $E$. From the lectures, our energy is the sum of kinetic and potential energies, in just the $\theta$ dimension. Potential can be found by integrating the (negative) right-hand-side of \ref{second order ode with new parameters} by our spacial parameter. 

        \begin{align}
            \implies V(\theta) &= \int \alpha\sin(\theta) \ d\theta \nonumber\\
            &= -\big(\alpha + \beta\cos(\tau))\cos(\theta)
        \end{align}

        The energy is now,

        \begin{align}
            E(\theta, \Omega) &= \frac{1}{2} \dot{\theta}^2 + V(\theta),    \nonumber   \\
            E(\theta, \Omega) &= \frac{1}{2} \Omega^2 -\alpha\cos(\theta).
        \end{align}

        Where the instantaneous change in energy (with respect to time) is,

        \begin{align}
            \dot{E}(\theta, \Omega) &=  \frac{d}{d\tau}\bigg[ \frac{1}{2}\Omega^2 - \alpha\cos(\theta) \bigg]  \nonumber\\
            &= \Omega\dot{\Omega} - \alpha\frac{d}{d\tau}\bigg[ \cos(\theta) \bigg] \nonumber\\
            &= -\Omega\big(\alpha + \beta \cos(\tau)\big)\sin(\theta) + \alpha\sin(\theta)\dot{\theta}\nonumber\\
            &= -\beta\Omega\cos(\tau)\sin(\theta)
        \end{align}
        \end{solution}

        \hrule
        
        \jump
        For the child in the swing picture, is the energy $E$ (instantaneously) increasing or decreasing or neither in each of the frames shown? Try making a rough sketch of the time course of each of these quantities: $\theta(t), \Omega(t), \ell(t)$ and $E(t)$, assuming a periodic motion of the swing as well as the swinger. What is the period of the swing motion compared to the period of $\ell(t)$? Are they the same?

        \begin{solution}
            As a means of faithfully showing the plots, I used matplotlib to create them. Note I am not showing the numerical values of the functions, since they are not enlightening. Note the function $\ell(t)$ is bounded by $1 \pm \ep$. Here is my code:

            \begin{lstlisting}

            import matplotlib.pyplot as plt
            import numpy as np
            import pandas as pd
            %matplotlib inline

            def ell(tau, ep):
                return (1 + ep * np.cos(2 * tau))
            def Omega(tau, ep):
                return -1*(np.sin(tau))
            def energy(tau, ep):
                return 0.5 * Omega(tau, ep) **2 - np.cos(theta(tau))
            def theta(tau):
                return np.cos(tau)

            ep = 0.1
            tau = np.linspace(-np.pi * 2, np.pi * 2, 200)
            tau_ticks = np.arange(-np.pi * 2, np.pi * 2, np.pi/2)
            tau_ticks_label = [f'0' if n == 0 else f'-pi/2' if n == -1 else f'pi/2' if n == 1 else f'{int(n/2)}pi' if n%2 == 0 else f'{int(n)}pi/2' for n in tau_ticks / (np.pi / 2)] 
            
            plt.figure(figsize=(10, 8))
            plt.subplots_adjust(hspace=0.5)
            
            plt.subplot(2, 2, 1)
            plt.plot(tau, ell(tau, ep), 'b-', label = "Length")
            plt.plot(tau, [np.mean(ell(tau, ep)) for _ in range(len(tau))], 'k--')
            plt.xlabel("tau")
            plt.title("Length")
            plt.yticks([])  
            plt.xticks(tau_ticks, tau_ticks_label)
            
            plt.subplot(2, 2, 2)
            plt.plot(tau, Omega(tau, ep), 'r-', label = "Omega")
            plt.plot(tau, 0 * tau, 'k--')
            plt.xlabel("tau")
            plt.title("Omega")
            plt.yticks([])  
            plt.xticks(tau_ticks, tau_ticks_label)
            
            plt.subplot(2, 2, 3)
            plt.plot(tau, energy(tau, ep), 'g-', label = "Energy")
            plt.plot(tau, [np.mean(energy(tau, ep)) for _ in range(len(tau))], 'k--')
            plt.xlabel("tau")
            plt.title("Energy")
            plt.yticks([])  
            plt.xticks(tau_ticks, tau_ticks_label)
            
            plt.subplot(2, 2, 4)
            plt.plot(tau, theta(tau),  'm-', label = "Theta")
            plt.plot(tau, 0 * tau, 'k--')
            plt.xlabel("tau")
            plt.title("Theta")
            plt.yticks([])  
            plt.xticks(tau_ticks, tau_ticks_label)
            
            plt.show()

            \end{lstlisting}

            Here are the resulting plots (I tried making this as horizontally aligned as possible):
            
            \jump
            \centerline{\noindent}%
            \makebox[\textwidth]{
            \includegraphics[scale = 0.6]{Images/homework1 heuristic plots.PNG}
            }
            \jump

            Note that these are here purely as a heuristic. As we expect the length of the pendulum arm to be maximal when
            \textbf{GIVE INTERPRETATION OF PLOTS!!}
        \end{solution}

        \item Linearize your system of first order equations about the equilibrium position $\theta = \Omega = 0$ to obtain a linear non-autonomous system to study:

        \begin{align}
            \frac{d\textbf{x}}{d\tau} = A(\tau)\textbf{x}, \label{Linearized system}
        \end{align}

        where $A(\tau) = A(\tau + T)$ is a 2$\times$2 T-periodic matrix, and $\textbf{x} \in \R^{2}$ is your state space vector. What is T for your non-dimensional problem?
        
        \jump
        \hrule

        \begin{solution}
        
            In preparation for this, we wrote out our "pseudo system" as \ref{Nonlinear system}. From this, we are only interested in small perturbations of $\theta$ and $\Omega$ around $\theta = \Omega = 0$. Thus, we can replace functions dependent on $\theta$ and $\Omega$ with their Taylor Expanded equivalent, up to linear term. From \ref{Nonlinear system}, the only function is $\sin(\theta)$, which can be approximated by $\sin(\theta) \approx \theta$ under our assumptions. Therefore we have,

            \begin{align}
                \begin{pmatrix}
                    \dot{\theta}    \\
                    \dot{\Omega}
                \end{pmatrix}
                \hspace{3mm}
                =
                \hspace{3mm}
                \begin{pmatrix}
                    0   &1  \\
                    -\big( \alpha + \beta\cos(\tau)\big) &0
                \end{pmatrix}
                \begin{pmatrix}
                    \theta  \\
                    \Omega
                \end{pmatrix}   \label{Linear system}
            \end{align}

            It is quite easy to see that our state vector, \textbf{x} is just $(\theta, \Omega)^t \in \R^2$, where $t$ denotes the transpose. It is also noted that our system $A(\tau)$ is periodic, with period $T = 2\pi$.
        \end{solution}

        \jump
        \hrule
        
        \jump
        Briefly summarize what Floquet theory tells us about the form of the solutions of \ref{Linear system}.

        \begin{solution}
        
            \textbf{WHAT DOES FLOQUET THEORY TELL US?}
        \end{solution}
        
\end{enumerate}
\end{document}