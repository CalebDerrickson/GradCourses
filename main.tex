\documentclass[12pt]{article}
\usepackage[paper=letterpaper,margin=1.5cm]{geometry}
\usepackage{amsmath}
\usepackage{amssymb}
\usepackage{amsfonts}
\usepackage{mathtools}
%\usepackage[utf8]{inputenc}
%\usepackage{newtxtext, newtxmath}
\usepackage{lmodern}     % set math font to Latin modern math
\usepackage[T1]{fontenc}
\renewcommand\rmdefault{ptm}
%\usepackage{enumitem}
\usepackage[shortlabels]{enumitem}
\usepackage{titling}
\usepackage{graphicx}
\usepackage[colorlinks=true]{hyperref}
\usepackage{setspace}
\usepackage{subfigure} 
\usepackage{braket}
\usepackage{color}
\usepackage{tabularx}
\usepackage[table]{xcolor}
\usepackage{listings}
\usepackage{mathrsfs}
\usepackage{stackengine}
\usepackage{physics}
\usepackage{afterpage}
\usepackage{pdfpages}
\usepackage[export]{adjustbox}
\usepackage{biblatex}

\setstackEOL{\\}

\definecolor{dkgreen}{rgb}{0,0.6,0}
\definecolor{gray}{rgb}{0.5,0.5,0.5}
\definecolor{mauve}{rgb}{0.58,0,0.82}


\lstset{frame=tb,
  language=Python,
  aboveskip=3mm,
  belowskip=3mm,
  showstringspaces=false,
  columns=flexible,
  basicstyle={\small\ttfamily},
  numbers=none,
  numberstyle=\tiny\color{gray},
  keywordstyle=\color{blue},
  commentstyle=\color{dkgreen},
  stringstyle=\color{mauve},
  breaklines=true,
  breakatwhitespace=true,
  tabsize=3
}
\setlength{\droptitle}{-6em}

\makeatletter
% we use \prefix@<level> only if it is defined
\renewcommand{\@seccntformat}[1]{%
  \ifcsname prefix@#1\endcsname
    \csname prefix@#1\endcsname
  \else
    \csname the#1\endcsname\quad
  \fi}
% define \prefix@section
\newcommand\prefix@section{}
\newcommand{\prefix@subsection}{}
\newcommand{\prefix@subsubsection}{}
\renewcommand{\thesubsection}{\arabic{subsection}}
\makeatother
\DeclareMathOperator*{\argmin}{argmin}
\newcommand{\partbreak}{\begin{center}\rule{17.5cm}{2pt}\end{center}}
\newcommand{\alignbreak}{\begin{center}\rule{15cm}{1pt}\end{center}}
\newcommand{\tightalignbreak}{\vspace{-5mm}\alignbreak\vspace{-5mm}}
\newcommand{\hop}{\vspace{1mm}}
\newcommand{\jump}{\vspace{5mm}}
\newcommand{\R}{\mathbb{R}}
\newcommand{\C}{\mathbb{C}}
\newcommand{\N}{\mathbb{N}}
\newcommand{\G}{\mathbb{G}}
\renewcommand{\S}{\mathbb{S}}
\newcommand{\bt}{\textbf}
\newcommand{\xdot}{\dot{x}}
\renewcommand{\star}{^{*}}
\newcommand{\ydot}{\dot{y}}
\newcommand{\lm}{\mathrm{\lambda}}
\renewcommand{\th}{\theta}
\newcommand{\id}{\mathbb{I}}
\newcommand{\si}{\Sigma}
\newcommand{\Si}{\si}
\newcommand{\inv}{^{-1}}
\newcommand{\T}{^\intercal}
\renewcommand{\tr}{\text{tr}}
\newcommand{\ep}{\varepsilon}
\newcommand{\ph}{\varphi}
%\renewcomand{\norm}[1]{\left\lVert#1\right\rVert}
\definecolor{cit}{rgb}{0.05,0.2,0.45}
\addtolength{\jot}{1em}
\newcommand{\solution}[1]{

\noindent{\color{cit}\textbf{Solution:} #1}}

\newcounter{tmpctr}
\newcommand\fancyRoman[1]{%
  \setcounter{tmpctr}{#1}%
  \setbox0=\hbox{\kern0.3pt\textsf{\Roman{tmpctr}}}%
  \setstackgap{S}{-.9pt}%
  \Shortstack{\rule{\dimexpr\wd0+.1ex}{.9pt}\\\copy0\\
              \rule{\dimexpr\wd0+.1ex}{.9pt}}%
}

\newcommand{\Id}{\fancyRoman{2}}

% Enter the specific assignment number and topic of that assignment below, and replace "Your Name" with your actual name.
\title{STAT 31210: Homework 4}
\author{Caleb Derrickson}
\date{February 2, 2024}

\begin{document}
\onehalfspacing
\maketitle
\allowdisplaybreaks
{\color{cit}\vspace{2mm}\noindent\textbf{Collaborators:}} The TA's of the class, as well as Kevin Hefner, and Alexander Cram.

\tableofcontents

\newpage
\section{Problem 5.2}
Suppose that $\{ e_1, e_2, ..., e_n\}$ and $\{\Bar{e}_1, \Bar{e}_2, ..., \Bar{e}_n\}$ are two bases of the n-dimensional linear space $X$, with
\[\Bar{e}_i = \sum_{j = 1}^nL_{ij}e_j, \quad e_i = \sum_{j = 1}^n \Bar{L}_{ij}\Bar{e}_j\]
where $L$ is an invertible matrix with inverse $\Bar{L}$, i.e., $\sum_{j = 1}^n L_{ij}\Bar{L}_{jk} = \delta_{ik}$. Let $\{\omega_1, \omega_2, ..., \omega_n\}$ and $\{ \Bar{\omega}_1, \Bar{\omega}_2, ..., \Bar{\omega}_n\}$ be the associated dual bases of $X\star$.
\subsection{Problem 5.2, part a}
If $x = \sum x_ie_i = \sum \Bar{x}_i \Bar{e}_i \in X$, then prove that the components of $x$ transform under a change of basis according to 
\[\Bar{x}_i = \Bar{L}_{ji}x_j, \quad \forall \ i = 1, ..., n\]
\partbreak
\begin{solution}

    If we start with the first expansion of $x$ in the basis $\{ e_1, e_2, ..., e_n\}$ and rewrite $e_i$ into the given form, we have 
    \[x = \sum_{i = 1}^n x_i e_i = \sum_{i = 1}^n x_i \sum_{j = 1}^n \Bar{L}_{ij}\Bar{e}_j.\]
    After some rearranging, we have
    \[x = \sum_{i = 1}^n \left(\sum_{j = 1}^n \Bar{L}_{ij}x_i \right)\Bar{e}_j.\]
    Note that the summation indices can be freely interchanged, so swapping the two indices, we have,
    \[x = \sum_{i = 1}^n \left(\sum_{j = 1}^n \Bar{L}_{ji}x_j \right)\Bar{e}_i.\]
    By the other expansion for $x$ in the basis $\{\Bar{e}_1, \Bar{e}_2, ..., \Bar{e}_n\}$, we can compare the coefficients in each expansion to get 
    \[\Bar{x}_i = \Bar{L}_{ji}x_j. \]
    Note that this holds for all $i = 1, ..., n$, which is what we wanted to show.
\end{solution}

\newpage
\subsection{Problem 5.2, part b}
If $\ph = \sum_{i = 1}^n \ph_i \omega_i = \sum_{i = 1}^n \Bar{\ph}_i \Bar{\omega}_i \in X\star$, then prove that the components of $\ph$ transform under a change of basis according to
\[\Bar{\ph}_i = \sum_{j  =1}^n L_{ij}\ph_j, \quad \forall \ i = 1, ..., n\]
\partbreak
\begin{solution}

    Investigating the action of $\ph$ onto the basis vector $\Bar{e_i}$, by the given mapping from $\Bar{e} \mapsto e$, we have 
    \[\ph(\Bar{e}_i) = \ph \left( \sum_{j = 1}^n L_{ij} e_j\right).\]
    Since $\ph$ is a linear operator, which only acts on coordinates, we have that 
    \[\ph(\Bar{e}_i) = \sum_{j = 1}^n L_{ij} \ph (e_j).\]
    By our notation, we write $\ph(\Bar{e}_i) = \Bar{\ph}$ and $\ph (e_j) = \ph_j$, thus we can write
    \[\Bar{\ph}_i = \sum_{j = 1}^n L_{ij} \ph_j.\]
    This formula will hold for any chosen $i \leq n$, since its choice was arbitrary. Thus, we have proven the statement. 
\end{solution}

\newpage
\section{Problem 5.6}
Let $X$ be a normed linear space. Use the Hahn-Banach Theorem to prove the following statements:
\begin{itemize}[a)]
    \item For any $x \in X$, there is a bounded linear functional $\ph \in X\star$ such that $\norm{\ph} = 1$ and $\ph (x) = \norm{x}$.
    \item[b)] If $x, y \in X$ and $\ph(x) = \ph(y) $ for any $\ph \in X\star$, then $x = y$.
\end{itemize}
\end{document}