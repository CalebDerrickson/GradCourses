\documentclass[12pt]{article}
\usepackage[paper=letterpaper,margin=1.5cm]{geometry}
\usepackage{amsmath}
\usepackage{amssymb}
\usepackage{amsfonts}
\usepackage{mathtools}
%\usepackage[utf8]{inputenc}
%\usepackage{newtxtext, newtxmath}
\usepackage{lmodern}     % set math font to Latin modern math
\usepackage[T1]{fontenc}
\renewcommand\rmdefault{ptm}
%\usepackage{enumitem}
\usepackage[shortlabels]{enumitem}
\usepackage{titling}
\usepackage{graphicx}
\usepackage[colorlinks=true]{hyperref}
\usepackage{setspace}
\usepackage{subfigure} 
\usepackage{braket}
\usepackage{color}
\usepackage{tabularx}
\usepackage[table]{xcolor}
\usepackage{listings}
\usepackage{mathrsfs}
\usepackage{stackengine}
\usepackage{physics}
\usepackage{afterpage}
\usepackage{pdfpages}
\usepackage[export]{adjustbox}
\usepackage{biblatex}

\setstackEOL{\\}

\definecolor{dkgreen}{rgb}{0,0.6,0}
\definecolor{gray}{rgb}{0.5,0.5,0.5}
\definecolor{mauve}{rgb}{0.58,0,0.82}


\lstset{frame=tb,
  language=Python,
  aboveskip=3mm,
  belowskip=3mm,
  showstringspaces=false,
  columns=flexible,
  basicstyle={\small\ttfamily},
  numbers=none,
  numberstyle=\tiny\color{gray},
  keywordstyle=\color{blue},
  commentstyle=\color{dkgreen},
  stringstyle=\color{mauve},
  breaklines=true,
  breakatwhitespace=true,
  tabsize=3
}
\setlength{\droptitle}{-6em}

\makeatletter
% we use \prefix@<level> only if it is defined
\renewcommand{\@seccntformat}[1]{%
  \ifcsname prefix@#1\endcsname
    \csname prefix@#1\endcsname
  \else
    \csname the#1\endcsname\quad
  \fi}
% define \prefix@section
\newcommand\prefix@section{}
\newcommand{\prefix@subsection}{}
\newcommand{\prefix@subsubsection}{}
\renewcommand{\thesubsection}{\arabic{subsection}}
\makeatother
\DeclareMathOperator*{\argmin}{argmin}
\newcommand{\partbreak}{\begin{center}\rule{17.5cm}{2pt}\end{center}}
\newcommand{\alignbreak}{\begin{center}\rule{15cm}{1pt}\end{center}}
\newcommand{\tightalignbreak}{\vspace{-5mm}\alignbreak\vspace{-5mm}}
\newcommand{\hop}{\vspace{1mm}}
\newcommand{\jump}{\vspace{5mm}}
\newcommand{\R}{\mathbb{R}}
\newcommand{\C}{\mathbb{C}}
\newcommand{\N}{\mathbb{N}}
\newcommand{\G}{\mathbb{G}}
\renewcommand{\S}{\mathbb{S}}
\newcommand{\bt}{\textbf}
\newcommand{\xdot}{\dot{x}}
\renewcommand{\star}{^{*}}
\newcommand{\ydot}{\dot{y}}
\newcommand{\lm}{\mathrm{\lambda}}
\renewcommand{\th}{\theta}
\newcommand{\id}{\mathbb{I}}
\newcommand{\si}{\Sigma}
\newcommand{\Si}{\si}
\newcommand{\inv}{^{-1}}
\newcommand{\T}{^\intercal}
\renewcommand{\tr}{\text{tr}}
\newcommand{\ep}{\varepsilon}
\newcommand{\ph}{\varphi}
%\renewcomand{\norm}[1]{\left\lVert#1\right\rVert}
\definecolor{cit}{rgb}{0.05,0.2,0.45}
\addtolength{\jot}{1em}
\newcommand{\solution}[1]{

\noindent{\color{cit}\textbf{Solution:} #1}}

\newcounter{tmpctr}
\newcommand\fancyRoman[1]{%
  \setcounter{tmpctr}{#1}%
  \setbox0=\hbox{\kern0.3pt\textsf{\Roman{tmpctr}}}%
  \setstackgap{S}{-.9pt}%
  \Shortstack{\rule{\dimexpr\wd0+.1ex}{.9pt}\\\copy0\\
              \rule{\dimexpr\wd0+.1ex}{.9pt}}%
}

\newcommand{\Id}{\fancyRoman{2}}

% Enter the specific assignment number and topic of that assignment below, and replace "Your Name" with your actual name.
\title{STAT 31020: Homework 1}
\author{Caleb Derrickson}
\date{Wednesday, January 10, 2024}

\begin{document}
\onehalfspacing
\maketitle
\allowdisplaybreaks
%{\color{cit}\vspace{2mm}\noindent\textbf{Collaborators:}} The TA's of the class, as well as Kevin Hefner, and Alexander Cram.

\tableofcontents

\newpage
\section{Problem 1}
Let $f(x): \R \rightarrow \R$ be an infinitely differentiable function. Let $k(x)$ be the smallest integer that satisfies $f^{(j)}(x) = 0, 1 \leq k \leq k - 1, f^{(k)}(x) \neq 0$ (that is the smallest index of the derivative that is nonzero. Here, $f^{(l)}(x)$ is the $l$-th derivative of $f$ at $x$.

\subsection{Problem 1, Part 1}
Prove that, if $k(x)$ is odd, then $x$ is neither a local minimum nor a local maximum. 
\partbreak

\begin{solution}

    Since $f$ is given as arbitrarily differentiable, Taylor's Theorem can be applied. I will provide the theorem below, for reference.
    
    \begin{center}\rule{17cm}{0.1mm}\end{center}
    \begin{quote}
        \vspace{-5mm}
        \underline{Taylor's Theorem} \footnote{This theorem is taken from the Wikipedia page for Taylor's Theorem, directly under the section titled ``Taylor's theorem in one real variable."}
        
        Let $k \geq 1$ be an integer and let the function $f :\R \rightarrow \R$ be $k$ times differentiable at the point $a \in \R$. Then there exists a function $h_k : \R \rightarrow \R$ such that 
        \[f(x) = f(a) + f'(a) + \frac{1}{2!}\frac{d^2f}{dx^2}\bigg|_{x = a}(x - a)^2 + ... + \frac{1}{k!}\frac{d^kf}{dx^k}\bigg|_{x = a}(x - a)^k + h_k(x)(x - a)^k,\]
        where $\lim_{x \rightarrow a} h_k(x) = 0.$
    \vspace{-5mm}
    \end{quote}
    \begin{center}\rule{17cm}{0.1mm}\end{center}
    Supposing $k$ is odd, then $k = 2m+1$, for $m \in \N \cup \{ 0 \}$.\footnote{I am writing it like this to avoid ambiguity, since I take the natural numbers without 0.} Not also by assumption, all derivatives of $f$ evaluated at $a$ are zero, thus the Taylor expansion of $f(x)$ is simplified to
    \[
    f(x) = f(a) + f'(a) + \frac{1}{(2m+1)!}\frac{d^{(2m+1)}f}{dx^{(2m+1)}}\bigg|_{x = a}(x - a)^{(2m+1)} + h_{(2m+1)}(x)(x - a)^{(2m+1)}.
    \]
    Assuming we can truncate $f$ (i.e., discard the function $h(x)$) around a neighborhood $U_a$ of $a$, then we can approximate $f(x)$ for $x \in U_a$ as 
    \[
    f(x) \approx f(a) + \frac{1}{(2m+1)!}f^{(2m+1)}(x = a) (x - a)^{(2m+1)}.
    \]
    Thus, we can analyze $f$ according to the extremality of $f(a)$.
    \newpage
    \begin{itemize}[-]
        \item \underline{\textbf{Case 1:}} $ \ f(a)$ is a local minimum.
        
        \hop
        Then $f(x) - f(a) \geq 0$ should be true for all $x \in U_a$. This means $\frac{1}{(2m+1)!}f^{(2m+1)}(x = a) (x - a)^{(2m+1)} \geq 0$, so $(x - a)^{(2m+1)} \geq 0$. Note that this is an odd function, so for points less than $a$, this inequality cannot hold. Thus $f(a)$ cannot be a local minimum.
        
        \item \underline{\textbf{Case 2:}} $ \ f(a)$ is a local maximum.

        \hop
        Then the same argument can be made in the previous case, replacing the inequality with $f(x) - f(a) \leq 0$. Then $(x - a)^{(2m+1)} \leq 0$, and since this is an odd function, we cannot find a neighborhood around $a$ such that this inequality is satisfied for all points within the neighborhood. Thus $f(a)$ cannot be a local maximum.
    \end{itemize}
\end{solution}

\newpage
\subsection{Problem 1, part 2}
Prove that if $k(a)$ is even, then $a$ is either a strict local minimum or a strict local maximum.  
\partbreak
\begin{solution}

    Given we are in the same environment as in the previous part, we will truncate $f$ so that around a neighborhood $U_a$ of $a$, $f(x)$ can be reasonably approximated as 
    \[f(x \in U_a) \approx f(a) + \frac{1}{(k)!}f^{(k)}(x = a) (x - a)^{(k)}.\]
    Since $k$ is even, it can be expressed as $k = 2m$, for $m \in \N \cup \{0\}$. Then we can analyze $f(a)$ based on the value of $f(x) - f(a)$ Note that the case where $f(x) = f(a)$ is not considered, since this would imply $f^{(k)}(x = a) = 0$, which under our assumption of $k$ cannot happen.

    \begin{itemize}[-]
        \item \underline{\textbf{Case 1:}} $ \ f(x) - f(a) > 0 $. 

        \hop
         This implies $\frac{1}{(2m)!}f^{(2m)}(x = a) (x - a)^{(2m)} > 0$, so $f^{(2m)}(x = a) > 0$, since $\frac{1}{(2m)!}(x - a)^{(2m)} \geq 0$ for any $x \in U_a, m \in \N \cup \{0\}.$ Therefore, since $f^{(2m)}(x = a) > 0$, $a$ is a strict local minimum. 

         \item \underline{\textbf{Case 2:}} $ \ f(x) - f(a) < 0$.

         \hop
         Then $\frac{1}{(2m)!}f^{(2m)}(x = a) (x - a)^{(2m)} < 0$, so $f^{(2m)}(x = a) < 0$, since $\frac{1}{(2m)!}(x - a)^{(2m)} \geq 0$ for any $x \in U_a, m \in \N \cup \{0\}.$ Therefore, since $f^{(2m)}(x = a) < 0$, $a$ is a strict local maximum. 
    \end{itemize}

    Therefore, $a$ can either be a strict local maximum or minimum, depending on the sign of $f(x) - f(a)$.  
\end{solution}

\newpage
\subsection{Problem 1, part 3}
In the same situation, prove that $a$ is an isolated extremum.
\partbreak
\begin{solution}

    Assuming that $a$ is a local extremum of $f$, then we can apply the first order necessary conditions to get that $f'(a) = 0$. Applying Taylor's Theorem to $f'(x)$ then gives
    \[
    \deriv{f}{x}{1} (x) = \deriv{f}{x}{1} \bigg|_{x = a} + \deriv{f}{x}{2}\bigg|_{x = a} (x - a) + \frac{1}{2!}\deriv{f}{x}{3}\bigg|_{x = a} (x - a)^2 + ... + \frac{1}{(k-1)!}\deriv{f}{x}{k} \bigg|_{x = a} (x - a)^{(k-1)} + h_k(x) 
    \]
    This is allowed in some neighborhood $U_a$ of $a$. Note that all derivatives prior to the $k$-th are zero. Suppose further that $U_a$ also satisfies the approximation performed in the previous parts, such that
    \[
    \deriv{f}{x}{1} (x \in U_a) \approx \frac{1}{(k-1)!}\deriv{f}{x}{k} \bigg|_{x = a}(x - a)^{(k-1)} .
    \]
    Note that if $x' \in U_a$ is an extremum, then $\deriv{f}{x}{1}(x') = 0$. When evaluating $\deriv{f}{x}{1}$ at $x = x'$ and setting it equal to zero, we get
    \[
    \deriv{f}{x}{1}(x') = 0 = \frac{1}{(k-1)!}\deriv{f}{x}{k} \bigg|_{x = a}(x' - a)^{(k-1)}.
    \]
    By assumption, $\deriv{f}{x}{k} (x = a) \neq 0$. We can cancel this out, plus the factor of $(k-1)!$ to get $(x' - a)^{(k-1)} = 0$. This then implies $x' = a$. Thus there are no other extrema within the neighborhood $U_a$, meaning $a$ is an isolated extremum. 
\end{solution}

\newpage
\section{Problem 2}
 As we discussed in class, an important issue in comparing algorithms is the asymptotic order of convergence. This problems exposes several rates of convergence and examples concerning them. I suggest you also read the subsection ``Rates of Convergence” of the Appendix A2 of the textbook (pages 619 - 620 in the 2006 edition). The following points refer to the textbook problem numbering. 
 \subsection{Problem 2.13}
 
\end{document}