\documentclass[12pt]{article}
\usepackage[paper=letterpaper,margin=1.5cm]{geometry}
\usepackage{amsmath}
\usepackage{amssymb}
\usepackage{amsfonts}
\usepackage{mathtools}
%\usepackage[utf8]{inputenc}
%\usepackage{newtxtext, newtxmath}
\usepackage{lmodern}     % set math font to Latin modern math
\usepackage[T1]{fontenc}
\renewcommand\rmdefault{ptm}
%\usepackage{enumitem}
\usepackage[shortlabels]{enumitem}
\usepackage{titling}
\usepackage{graphicx}
\usepackage[colorlinks=true]{hyperref}
\usepackage{setspace}
\usepackage{subfigure} 
\usepackage{braket}
\usepackage{color}
\usepackage{tabularx}
\usepackage[table]{xcolor}
\usepackage{listings}
\usepackage{mathrsfs}
\usepackage{stackengine}
\usepackage{physics}
\usepackage{afterpage}
\usepackage{pdfpages}
\usepackage[export]{adjustbox}
\usepackage{biblatex}

\setstackEOL{\\}

\definecolor{dkgreen}{rgb}{0,0.6,0}
\definecolor{gray}{rgb}{0.5,0.5,0.5}
\definecolor{mauve}{rgb}{0.58,0,0.82}


\lstset{frame=tb,
  language=Python,
  aboveskip=3mm,
  belowskip=3mm,
  showstringspaces=false,
  columns=flexible,
  basicstyle={\small\ttfamily},
  numbers=none,
  numberstyle=\tiny\color{gray},
  keywordstyle=\color{blue},
  commentstyle=\color{dkgreen},
  stringstyle=\color{mauve},
  breaklines=true,
  breakatwhitespace=true,
  tabsize=3
}
\setlength{\droptitle}{-6em}

\makeatletter
% we use \prefix@<level> only if it is defined
\renewcommand{\@seccntformat}[1]{%
  \ifcsname prefix@#1\endcsname
    \csname prefix@#1\endcsname
  \else
    \csname the#1\endcsname\quad
  \fi}
% define \prefix@section
\newcommand\prefix@section{}
\newcommand{\prefix@subsection}{}
\newcommand{\prefix@subsubsection}{}
\renewcommand{\thesubsection}{\arabic{subsection}}
\makeatother
\DeclareMathOperator*{\argmin}{argmin}
\newcommand{\partbreak}{\begin{center}\rule{17.5cm}{2pt}\end{center}}
\newcommand{\alignbreak}{\begin{center}\rule{15cm}{1pt}\end{center}}
\newcommand{\tightalignbreak}{\vspace{-5mm}\alignbreak\vspace{-5mm}}
\newcommand{\hop}{\vspace{1mm}}
\newcommand{\jump}{\vspace{5mm}}
\newcommand{\R}{\mathbb{R}}
\newcommand{\C}{\mathbb{C}}
\newcommand{\N}{\mathbb{N}}
\newcommand{\G}{\mathbb{G}}
\renewcommand{\S}{\mathbb{S}}
\newcommand{\bt}{\textbf}
\newcommand{\xdot}{\dot{x}}
\renewcommand{\star}{^{*}}
\newcommand{\ydot}{\dot{y}}
\newcommand{\lm}{\mathrm{\lambda}}
\renewcommand{\th}{\theta}
\newcommand{\id}{\mathbb{I}}
\newcommand{\si}{\Sigma}
\newcommand{\Si}{\si}
\newcommand{\inv}{^{-1}}
\newcommand{\T}{^\intercal}
\renewcommand{\tr}{\text{tr}}
\newcommand{\ep}{\varepsilon}
\newcommand{\ph}{\varphi}
%\renewcomand{\norm}[1]{\left\lVert#1\right\rVert}
\definecolor{cit}{rgb}{0.05,0.2,0.45}
\addtolength{\jot}{1em}
\newcommand{\solution}[1]{

\noindent{\color{cit}\textbf{Solution:} #1}}

\newcounter{tmpctr}
\newcommand\fancyRoman[1]{%
  \setcounter{tmpctr}{#1}%
  \setbox0=\hbox{\kern0.3pt\textsf{\Roman{tmpctr}}}%
  \setstackgap{S}{-.9pt}%
  \Shortstack{\rule{\dimexpr\wd0+.1ex}{.9pt}\\\copy0\\
              \rule{\dimexpr\wd0+.1ex}{.9pt}}%
}

\newcommand{\Id}{\fancyRoman{2}}

% Enter the specific assignment number and topic of that assignment below, and replace "Your Name" with your actual name.
\title{STAT 31410: Homework 1}
\author{Caleb Derrickson}
\date{October 17, 2023}

\begin{document}
\onehalfspacing
\maketitle

{\color{cit}\vspace{2mm}\noindent\textbf{Collaborators:}} The TA's of the class, as well as Kevin Hefner, Nathan Suhr, and Steven Lee.

\tableofcontents

\newpage

\section{Problem 1}
Determine for which values of $\beta > 0$ there is ``finite-time blow-up" of solutions(i.e. there exists a finite $T > 0$ such that lim$_{t \rightarrow T}$ $\lvert x(t) \rvert = \ + \infty$) for the following scalar ode:
\[
\dot{x} = x^\beta, \hspace{3mm} x(0) = x_0 \in \R^+.
\]

Specify the domain of existence of the solution, i.e. determine $T$ as a function of $x_0$ and $\beta$.

\partbreak
\begin{solution}

Treating this as a separable ODE, we can apply the following:

\alignbreak
\begin{align}
&\dot{x} = x^\beta &\text{(Given.)} \nonumber\\
&\frac{dx}{dt} = x^\beta &\text{(Explicit form of $\dot{x}$.)} \nonumber\\
&\frac{dx}{x^\beta} = dt &\text{(Rearranging.)} \nonumber\\        
\int_{x_0}^x &\frac{dx'}{(x')^\beta} = \int_0^t dt' &\text{(Integrating with given bounds.)} \label{pb1:cases}
\end{align}
\alignbreak

From \ref{pb1:cases}, we run into an immediate issue concerning $\beta$. Since $\beta = 1$ will give a notable outcome, we can break \ref{pb1:cases} into three cases, depending on certain values of $\beta$.

\alignbreak

\underline{\textbf{Case 1:}} $\beta > 1$

Then, 
\begin{align}
 &\int_{x_0}^x \frac{dx'}{(x')^\beta} = \int_0^t dt' &\text{(Given.)}\nonumber\\
 &\frac{1}{1 - \beta}(x')^{1 - \beta}\bigg|_{x_0}^x = t' \bigg|_0^t &\text{(Integrating.)}\nonumber\\
&x^{1 - \beta} = (1 - \beta)t + x_0^{1 - \beta} &\text{(Applying bounds and Rearranging.)} \nonumber\\
&x = \sqrt[1 - \beta]{(1 - \beta)t + x_0^{1 - \beta}} &\text{(Simplifying.)}\nonumber\\
&x = \frac{1}{\sqrt[\beta - 1]{(1 - \beta)t + x_0^{1 - \beta}}} &\text{(Applying reciprocal.)} \nonumber
\end{align}

Note the last step was performed with the mind that $\beta - 1 > 0$, and $1 - \beta < 0$. Since $t$ is given as a strictly positive number, $(1 - \beta)t$ is strictly negative, while $x_0^{1 - \beta}$ is strictly positive. Thus there will be some time T where these two quantities will equal, thus making the denominator zero, and making $x$ blow up. This will happen explicitly when
\[T = \frac{1}{1 - \beta}\Bigg(\frac{1}{x_0^{\beta - 1}}\Bigg)\]

\underline{\textbf{Case 2:}} $\beta = 1$

Then,
\begin{align}
     &\int_{x_0}^x \frac{dx'}{x'} = \int_0^t dt' &\text{(Given with $\beta = 1$.)}\nonumber\\
    &\ln{x} - \ln x_0 = t &\text{(Integrating and applying bounds.)}\nonumber\\
    &x(t) = x_0e^t &\text{(Rearranging and simplifying.)}\nonumber\\
\end{align}

Since our solution is an exponential in $t$, there will be no finite time $T$ such that $x(t)$ ``blows up".

\underline{\textbf{Case 3:}} $0 < \beta < 1$

Then,
\begin{align}
    &\int_{x_0}^x \frac{dx'}{(x')^\beta} = \int_0^t dt' &\text{(Given.)}\nonumber\\
    &\frac{1}{1 - \beta}(x')^{1 - \beta}\bigg|_{x_0}^x = t' \bigg|_0^t &\text{(Integrating.)}\nonumber\\
    &x^{1 - \beta} = (1 - \beta)t + x_0^{1 - \beta} &\text{(Applying bounds and Rearranging.)} \nonumber\\
    &x = \sqrt[1 - \beta]{(1 - \beta)t + x_0^{1 - \beta}} &\text{(Simplifying.)}\nonumber\\
\end{align}

Note that $0 < \beta < 1$, i.e. $1 - \beta > 0$. Thus, $(1 - \beta)t$ is strictly positive, as is $x_0^{1 - \beta}$. Thus there is no time such that $x(t)$ ``blows up".

\alignbreak

Therefore, it has been shown the only case where there is a finite time in which $x(t)$ ``blows up" is when $\beta > 1$, giving an explicit time $T$ where we approach $\infty$.

\jump
\end{solution}% End of problem 1 part 1

\partbreak
\subsection{Problem 1, part 2}
In the remainder of this problem, let's focus on $\beta = 2$  and $x_0 = 1$. What is the time $T$ to ``blow up" in this case? Introduce the following transformation from $t$ to a new time variable $\tau$:
\[
\tau = \int_0^t 1 + x^2(s) ds, \hspace{3mm} t\in [0,T)
\]
where $x(s)$ satisfies the original problem $dx/ds = x^2$. Show that this is a one-to-one mapping between $t \in [0, T)$ and $\tau$, and specify the range of $\tau$. Let $y(\tau) = x(t(\tau))$; what differential equation does $y(\tau)$ satisfy? Does $y(\tau)$ have finite time blow-up? 

\begin{solution}

    We are given $\beta = 2, x_0 = 1$. From the first case of problem 1, we have
    \[
    x = \frac{1}{1 - t}, \hspace{10mm} T = \frac{1}{2 - 1}\Bigg(\frac{1}{1^{2 - 1}}\Bigg) = 1
    \]

    Therefore, since we have an explicit function for $x(t)$, we can plug this in to the transformation to $\tau$.

    \alignbreak
    \begin{align}
        \tau &= \int_0^{t} 1 + \frac{1}{(1 - t')^2}dt' &\text{(Plugging in $x(t)$.)}\nonumber\\
        &= \int_0^{t} \frac{(1 - t')^2 + 1}{(1 - t')^2}dt' &\text{(Simplifying.)}\nonumber\\
        &= \int_1^{1-t} \frac{u^2 + 1}{u^2} du &\text{($u$ sub. with $u = 1 - t'$.)} \nonumber\\
        &= \ u\bigg|_1^{1 - t} + \frac{1}{u}\bigg|^1_{1 - t} &\text{(Integrating.)}\nonumber\\
        &= 1 - t - 1 + 1 - \frac{1}{1 - t} &\text{(Applying bounds.)}\nonumber\\
        &= t + 1 - \frac{1}{1-t} &\text{(Simplifying.)}\nonumber\\
        &= \frac{(1 - t)^2 - 1}{1 - t} &\text{(Simplifying.)}\nonumber\\
        \tau&= \frac{t(t - 2)}{1 - t} &\text{(Simplifying.)} \label{p1:tau in t}
    \end{align}
    \alignbreak

    By \ref{p1:tau in t}, we have an explicit function for $\tau$ in terms of $t$. To show injectivity, it suffices to show that $\tau'(t) > 0$ for all $t > 0$.

    \begin{center}\rule{15cm}{0.5pt}\end{center}
    \begin{align}
        \frac{d\tau}{dt} &= \frac{d}{dt}\bigg[ t + 1 - \frac{1}{1 - t}\bigg] &\text{(Third to last line above.)}\nonumber\\
        &= 1 + \frac{1}{(1 - t)^2} &\text{(Differentiating.)} \label{p1:diff}
    \end{align}
    \alignbreak
    We can note that \ref{p1:diff} is strictly positive for all $t > 0$, thus $\tau$ is injective.

    Next, we wish to show what differential equation does $y(\tau)$ satisfy. Note that from the lectures, $y(\tau)$ should satisfy
    \[\frac{dy}{d\tau} = F(y) = \frac{f(y)}{1 + |f(y)|}\]

    From the chain rule,
    \[\frac{dy}{d\tau} = \frac{dx}{dt}\frac{dt}{d\tau}\]

    We have an explicit form for $dx/dt$, and we have the reciprocal for $dt/d\tau$. Thus, we will simply flip $d\tau/dt$ to get $dt/d\tau$. This implies,

    \alignbreak
    \begin{align}
        \frac{d\tau}{dt} &= \frac{(1 - t)^2 + 1}{(1 - t)^2} \nonumber\\
        \frac{dt}{d\tau} &= \frac{(1 - t)^2}{(1 - t)^2 + 1} \label{p1:dtdtau}\\  
        \frac{dx}{dt} &= \frac{d}{dt}\Bigg[ \frac{1}{1 - t}\Bigg] = \frac{1}{(1 - t)^2} \label{p1:dxdt}
    \end{align}
    \alignbreak

    Thus, \ref{p1:dtdtau} can be applied to \ref{p1:dxdt} to get, 

    \alignbreak
    \begin{align}
        \frac{dy}{d\tau} &= \frac{dx}{dt}\frac{dt}{d\tau} &\text{(Given.)}\nonumber\\
        &= \Bigg(\frac{1}{(1 - t)^2}\Bigg)\Bigg( \frac{(1 - t)^2}{(1 - t)^2 + 1}\Bigg) &\text{(Plugging in \ref{p1:dxdt} and \ref{p1:dtdtau}.)}\nonumber\\
        &= \frac{1}{(1 - t)^2 + 1} &\text{(Simplifying.)}\nonumber\\
        &= \frac{\frac{1}{(1 - t)^2}}{1 + \frac{1}{(1 - t)^2}} &\text{(Dividing by $(1 - t)^2$.)}\nonumber\\
        &= \frac{y^2}{1 + y^2} &\text{(Applying $y(\tau) = x(t(\tau))$.)}\nonumber\\
        \frac{dy}{d\tau} &= \frac{f(y)}{1 + |f(y)|} &\text{($f(y) = y^2$.)}\nonumber
    \end{align}
    \alignbreak

    With initial condition $y(\tau = 0) = x(t = 0)$???
\end{solution}% End of problem 1 part 2

\newpage

\section{Problem 2}
Consider the scalar initial value problem
\[
\dot{x} = -|x|^\beta, \hspace{5mm} x(0) = 0.
\]
For which values of $\beta > 0$ is the solution $x(t) = 0$ guaranteed to be unique on some interval $J = [-a, a], \ a > 0$? If there is no such guarantee, explicitly construct a family of solutions to the initial value problem on the interval $J$.   
\partbreak

\end{document}