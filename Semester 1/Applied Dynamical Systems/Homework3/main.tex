\documentclass[12pt]{article}
\usepackage[paper=letterpaper,margin=1.5cm]{geometry}
\usepackage{amsmath}
\usepackage{amssymb}
\usepackage{amsfonts}
\usepackage{mathtools}
%\usepackage[utf8]{inputenc}
%\usepackage{newtxtext, newtxmath}
\usepackage{lmodern}     % set math font to Latin modern math
\usepackage[T1]{fontenc}
\renewcommand\rmdefault{ptm}
%\usepackage{enumitem}
\usepackage[shortlabels]{enumitem}
\usepackage{titling}
\usepackage{graphicx}
\usepackage[colorlinks=true]{hyperref}
\usepackage{setspace}
\usepackage{subfigure} 
\usepackage{braket}
\usepackage{color}
\usepackage{tabularx}
\usepackage[table]{xcolor}
\usepackage{listings}
\usepackage{mathrsfs}
\usepackage{stackengine}
\usepackage{physics}
\usepackage{afterpage}
\usepackage{pdfpages}
\usepackage[export]{adjustbox}
\usepackage{biblatex}

\setstackEOL{\\}

\definecolor{dkgreen}{rgb}{0,0.6,0}
\definecolor{gray}{rgb}{0.5,0.5,0.5}
\definecolor{mauve}{rgb}{0.58,0,0.82}


\lstset{frame=tb,
  language=Python,
  aboveskip=3mm,
  belowskip=3mm,
  showstringspaces=false,
  columns=flexible,
  basicstyle={\small\ttfamily},
  numbers=none,
  numberstyle=\tiny\color{gray},
  keywordstyle=\color{blue},
  commentstyle=\color{dkgreen},
  stringstyle=\color{mauve},
  breaklines=true,
  breakatwhitespace=true,
  tabsize=3
}
\setlength{\droptitle}{-6em}

\makeatletter
% we use \prefix@<level> only if it is defined
\renewcommand{\@seccntformat}[1]{%
  \ifcsname prefix@#1\endcsname
    \csname prefix@#1\endcsname
  \else
    \csname the#1\endcsname\quad
  \fi}
% define \prefix@section
\newcommand\prefix@section{}
\newcommand{\prefix@subsection}{}
\newcommand{\prefix@subsubsection}{}
\renewcommand{\thesubsection}{\arabic{subsection}}
\makeatother
\DeclareMathOperator*{\argmin}{argmin}
\newcommand{\partbreak}{\begin{center}\rule{17.5cm}{2pt}\end{center}}
\newcommand{\alignbreak}{\begin{center}\rule{15cm}{1pt}\end{center}}
\newcommand{\tightalignbreak}{\vspace{-5mm}\alignbreak\vspace{-5mm}}
\newcommand{\hop}{\vspace{1mm}}
\newcommand{\jump}{\vspace{5mm}}
\newcommand{\R}{\mathbb{R}}
\newcommand{\C}{\mathbb{C}}
\newcommand{\N}{\mathbb{N}}
\newcommand{\G}{\mathbb{G}}
\renewcommand{\S}{\mathbb{S}}
\newcommand{\bt}{\textbf}
\newcommand{\xdot}{\dot{x}}
\renewcommand{\star}{^{*}}
\newcommand{\ydot}{\dot{y}}
\newcommand{\lm}{\mathrm{\lambda}}
\renewcommand{\th}{\theta}
\newcommand{\id}{\mathbb{I}}
\newcommand{\si}{\Sigma}
\newcommand{\Si}{\si}
\newcommand{\inv}{^{-1}}
\newcommand{\T}{^\intercal}
\renewcommand{\tr}{\text{tr}}
\newcommand{\ep}{\varepsilon}
\newcommand{\ph}{\varphi}
%\renewcomand{\norm}[1]{\left\lVert#1\right\rVert}
\definecolor{cit}{rgb}{0.05,0.2,0.45}
\addtolength{\jot}{1em}
\newcommand{\solution}[1]{

\noindent{\color{cit}\textbf{Solution:} #1}}

\newcounter{tmpctr}
\newcommand\fancyRoman[1]{%
  \setcounter{tmpctr}{#1}%
  \setbox0=\hbox{\kern0.3pt\textsf{\Roman{tmpctr}}}%
  \setstackgap{S}{-.9pt}%
  \Shortstack{\rule{\dimexpr\wd0+.1ex}{.9pt}\\\copy0\\
              \rule{\dimexpr\wd0+.1ex}{.9pt}}%
}

\newcommand{\Id}{\fancyRoman{2}}

% Enter the specific assignment number and topic of that assignment below, and replace "Your Name" with your actual name.
\title{STAT 31410: Homework 3}
\author{Caleb Derrickson}
\date{October 25, 2023}

\begin{document}
\onehalfspacing
\maketitle

{\color{cit}\vspace{2mm}\noindent\textbf{Collaborators:}} The TA's of the class, as well as Kevin Hefner, Alexander Cram, and Steven Lee.

\tableofcontents

\newpage
\section{Problem 1}
For this problem, let $\dot{x} = f(x)$, where $x \in \R$ and $f(x)$ is a $C^1$ function satisfying $f(0) = 0$, i.e. $x = 0$ is an equilibrium. 

\subsection{Problem 1, part a}
Find an $f(x)$ that shows that $x = 0$ may be unstable even though it appears to by Lyapunov stable for the corresponding linearized problem.

\partbreak
\begin{solution}

    For the sake of completion, I will provide the definition of Lyapunov stability.

    \alignbreak
    \vspace{-8mm}
    \begin{quote}
        If $x^*$ is an equilibrium point of the system $\dot{x} = f(x)$, then $x^*$ is Lyapunov stable if for every neighborhood $N$ of $x^*$ there exists a neighborhood $S\subset N$ such that $\forall \ x_0 \in S$, $x(t) \in N \ \forall \ t > 0$.   
    \end{quote}
    \vspace{-8mm}
    \alignbreak

    Our equilibrium point will be at 0, i.e. $f(x^*) = f(0) = 0$, so our function should match accordingly. If we let $f(x) = -sin(x)$, then our linearized problem about $x = 0$ is $\dot{x} = -x$. Thus, for any $x$ on which our function occupies will be attracted to the origin (but never reach it, but we are not examining that in this part). 
    However, we can notice that the non-linear system will have points not attracted to the origin, i.e., any points beyond $x = \pm\pi$. 
\end{solution}

\newpage
\subsection{Problem 1, part b}
Find an $f(x)$ that demonstrates that $x = 0$ may be asymptotically stable even when it is not linearly asymptotically stable.
\partbreak
\begin{solution}

    I will again include the definition of asymptotic stability.

    \alignbreak
    \vspace{-8mm}
    \begin{quote}
        If $x^*$ is an equilibrium point of the system $\dot{x} = f(x)$, then $x^*$ is asymptotically stable if it is Lyapunov stable and there exists a neighborhood $N$ such that if $x_0 \in N$, then $x(t) \rightarrow x^*$ as $t \rightarrow \infty$. 
    \end{quote}
    \vspace{-8mm}
    \alignbreak

    If we take $f(x) = -x^2\sin(x)$, then we see that there exists some neighborhood (roughly in the interval (-2.89, 2.89)) around 0 such that if $x_0 \in (-2.89, 2.89)$, then it will asymptotically approach $x^* = 0$. However, its linear term is zero, thus when solving the differential equation, we see that $x$ can be any constant function. Then the only point which will asymptotically approach $x^* = 0$ is $x_0 = 0$, which is not an open set, thus not Lyapunov stable, so not asymptotically stable. 
\end{solution} 

\newpage
\section{Problem 2}
The proof that linear asymptotic stability implies asymptotic stability in the text (Page 117) has a large number of positive constants introduced in it: $K, \delta, \ep, \alpha$. Describe the role of each in the proof, as well as any restrictions on them, ordering of them, etc. 

\partbreak
\begin{solution}

    \begin{itemize}
        \item $\alpha$ caps the eigenvalues of $A$ such that $\Re(\lm) < -\alpha < 0$, thereby ensuring all eigenvalues have real part less than zero.
        \item $K \geq 1$ is the bound obtained in Lemma 2.29, which bounds any solution with initial condition within the stable space of $A$. 
        \item $\ep$ and $\delta$ come from the fact that our system $f \in C^1$.
    \end{itemize}

    \jump
    From the above descriptions, their applications in the proof in various ways. In particular, the combination $K\delta$ bounds the distance from our solution to the equilibrium point, i.e., $|y| = |x - x^*| < K\delta$. Furthermore, by Gr\"{o}nwall's Lemma, this distance is further bounded in time by 
    \[
    |x - x^*| \leq K\delta e^{-(\alpha - K\ep)t} 
    \]
    Thus for any $\ep < \alpha K$, our solution is bounded. The proof then concludes with the statement, `` if S is the ball of radius $\delta$, then $N$ is the ball of radius $K\delta$", meaning Lyapunov stability is assured for neighborhood of radius $K\delta$ of our equilibrium $x^*$. 
\end{solution}

\newpage
\section{Problem 3}
Use the Hartman-Grobman theorem to prove that linear asymptotic stability implies asymptotic stability for an equilibrium $\textbf{x}^*$ of $\dot{\textbf{x}} = \textbf{f}(\textbf{x})$, where $\textbf{f}: \R^n \rightarrow \R^n$ is a $C^1$ vector filed. Can you also use it to show linear instability implies instability?

\partbreak

\begin{solution}
    
    Again, for completion, I will include the Hartman-Grobman here:

    \alignbreak
    \vspace{-8mm}
    \begin{quote}
        If $x^*$ is a hyperbolic equilibrium point of the system $\dot{x} = f(x)$, $f \in C^1$ with flow $\ph_t(x)$, the there is a neighborhood $N$ of $x^*$ such that $\ph_t(x)$ is topologically conjugate to its linearization on $N$.
    \end{quote}
    \vspace{-8mm}
    \alignbreak

    Since we are supposing an equilibrium $x^*$ is linear asymptotically stable, then all eigenvalues of the Jacobian $Df(x^*) = A$ has real part less than zero. Thus $x^*$ is a hyperbolic equilibrium of the linearized system. By Hartman-Grobman, we are guaranteed the existence of a neighborhood $N$ of $x^*$ such that the flow $\ph_t(x)$ is topologically conjugate to linearized counterpart $\psi_t(x)$ for any choice $x \in N$. Thus, picking $x_0 \in N$, we can say there exists a homeomorphism $h$ satisfying
    \[
    h(\ph_t(x_0)) = \psi_t(h(x_0)).
    \]

    Thus, we want to show that by our assumption of linearized asymptotic stability, our equilibrium $x_*$ is asymptotically stable. That is, for any $x_0 \in N$, $x(t) \rightarrow x^*$ as $t \rightarrow \infty$, where $x$ is any solution to our nonlinear system. Since $\ph_t(x_0)$ solves our system (as was demonstrated in class, as well as the texts), then we wish to show 
    \[
    \lim_{t \rightarrow \infty} \ph_t(x_0) = x^*.
    \]
    
    Note that since sour system is linearly asymptotically stable, then $h(x^*) = \lim_{t\rightarrow \infty} \psi_t(h(x_0))$. Thus the following steps are justified:

    \alignbreak
    \begin{align}
        h(x^*) &= \lim_{t \rightarrow \infty} \psi_t(h(x_0)) &\text{(By above.)}\nonumber\\
        &= \lim_{t \rightarrow \infty} h(\ph_t(x_0)) &\text{(Hartman-Grobman Theorem.)}\nonumber\\
        h(x^*) &= h(\lim_{t \rightarrow \infty} \ph_t(x_0)) &\text{(Continuity of $h$.)}\nonumber\\
        \implies x^* &= \lim_{t \rightarrow \infty} \ph_t(x_0) &\text{($h$ is one to one.)}\nonumber
    \end{align}
    \alignbreak

    \newpage
    Thus, it is shown that linear asymptotic stability implies asymptotic stability. Note linear instability means that there will exist some initial condition $x_0$ inside any neighborhood $N$ such that $\psi_t(h(x_0)) \notin N$ for some $t \in \R$ (note that Hartman-Grobman is still applicable in this case). As such this will imply $h(\ph_t(x_0)) \notin N \iff \ph_t(x_0) \notin N$ for any neighborhood $N$. Thus Linear instability implies instability.   
\end{solution}

\newpage
\section{Problem 4}
In section 4.8 of the textbook, on the Hartman-Grobman Theorem, there is an example of constructing a homeomorphism $H(x, y) = (x, y-x^2/3)$ that satisfies topological conjugacy condition $\psi_t \circ H = H \circ \ph_t$, where $\ph_t$ is generated by
\begin{align}
    \dot{x} &= x\nonumber\\
    \dot{y} &= -y+x^2.\nonumber
\end{align}

Solve these differential equations to obtain $\ph_t$ and $\psi_t$, where $\psi_t$ applies to the linearized flow. Then check, by explicit computation, that indeed $\psi_t \circ H = H \circ \ph_t$. 
\partbreak
\begin{solution}

    Solving this nonlinear system, we get

    \begin{align}
        \begin{pmatrix}x\\y\end{pmatrix} 
        \hspace{3mm} = \hspace{3mm}
        \begin{pmatrix}
            c_1 e^t\\
            \frac{1}{3} c_1^2 e^{2 t} + c_2 e^{-t}
        \end{pmatrix}\nonumber
    \end{align}

    Whereas in the linear system, i.e.
    \begin{align}
        \begin{pmatrix}\dot{x}\\\dot{y}\end{pmatrix} 
        \hspace{3mm} = \hspace{3mm}
        \begin{pmatrix}1 &0\\0 &-1\end{pmatrix}
        \begin{pmatrix}x\\y\end{pmatrix} 
        \nonumber
    \end{align}

    we see that 
    \begin{align}
        \begin{pmatrix}x\\y\end{pmatrix} 
        \hspace{3mm} = \hspace{3mm}
        \begin{pmatrix}
            c_1 e^t\\
            c_2 e^{-t}
        \end{pmatrix}\nonumber
    \end{align}

    Thus, our flow for the nonlinear system and the linear system are given above, where $c_1, c_2$ can be found by abusing what we know about the flow of a differential equation, i.e., 

    \[
    f(x) = \frac{d}{dt}\ph_t(x)\Big|_{t = 0}
    \]

    Where this coniditon will need to be applied to both systems independently. Applying this our nonlinear system, we get,

    \begin{align}
        \begin{pmatrix}c_1\\ \frac{2}{3}c_1^2 - c_2\end{pmatrix} 
        \hspace{3mm} = \hspace{3mm}
        \begin{pmatrix}x \\ -y + x^2\end{pmatrix} \nonumber
    \end{align}

    It is easy to see that $c_1 = x$. Then we can plug this into the second equation and, when simplifying, get $c_2 = y - \frac{1}{3}x^2$. Thus our flow for the nonlinear system is given as 

    \begin{align}
        \ph_t(x)
        \hspace{3mm} = \hspace{3mm}
        \begin{pmatrix}
            x_1 e^t\\
            \frac{1}{3} x_1^2 e^{2 t} + (x_2 - \frac{1}{3}x_1^2) e^{-t}
        \end{pmatrix}\label{p4: nonlinear flow}
    \end{align}

    A similar procedure can be done for the linear system, in which case, we get

    \begin{align}
        \begin{pmatrix}c_1\\ - c_2\end{pmatrix} 
        \hspace{3mm} = \hspace{3mm}
        \begin{pmatrix}x \\ -y\end{pmatrix} \nonumber
    \end{align}

    Then our linear flow $\psi_t(x)$ is given as 

    \begin{align}
        \psi_t(x)
        \hspace{3mm} = \hspace{3mm}
        \begin{pmatrix}
            x_1 e^t\\
            x_2e^{-t}
        \end{pmatrix}\label{p4: linear flow}
    \end{align}

    where $x = x_1, y = x_2$.
\end{solution}

\end{document}
