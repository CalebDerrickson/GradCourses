\documentclass[12pt]{article}
\usepackage[paper=letterpaper,margin=1.5cm]{geometry}
\usepackage{amsmath}
\usepackage{amssymb}
\usepackage{amsfonts}
\usepackage{mathtools}
%\usepackage[utf8]{inputenc}
%\usepackage{newtxtext, newtxmath}
\usepackage{lmodern}     % set math font to Latin modern math
\usepackage[T1]{fontenc}
\renewcommand\rmdefault{ptm}
%\usepackage{enumitem}
\usepackage[shortlabels]{enumitem}
\usepackage{titling}
\usepackage{graphicx}
\usepackage[colorlinks=true]{hyperref}
\usepackage{setspace}
\usepackage{subfigure} 
\usepackage{braket}
\usepackage{color}
\usepackage{tabularx}
\usepackage[table]{xcolor}
\usepackage{listings}
\usepackage{mathrsfs}
\usepackage{stackengine}
\usepackage{physics}
\usepackage{afterpage}
\usepackage{pdfpages}
\usepackage[export]{adjustbox}
\usepackage{biblatex}

\setstackEOL{\\}

\definecolor{dkgreen}{rgb}{0,0.6,0}
\definecolor{gray}{rgb}{0.5,0.5,0.5}
\definecolor{mauve}{rgb}{0.58,0,0.82}


\lstset{frame=tb,
  language=Python,
  aboveskip=3mm,
  belowskip=3mm,
  showstringspaces=false,
  columns=flexible,
  basicstyle={\small\ttfamily},
  numbers=none,
  numberstyle=\tiny\color{gray},
  keywordstyle=\color{blue},
  commentstyle=\color{dkgreen},
  stringstyle=\color{mauve},
  breaklines=true,
  breakatwhitespace=true,
  tabsize=3
}
\setlength{\droptitle}{-6em}

\makeatletter
% we use \prefix@<level> only if it is defined
\renewcommand{\@seccntformat}[1]{%
  \ifcsname prefix@#1\endcsname
    \csname prefix@#1\endcsname
  \else
    \csname the#1\endcsname\quad
  \fi}
% define \prefix@section
\newcommand\prefix@section{}
\newcommand{\prefix@subsection}{}
\newcommand{\prefix@subsubsection}{}
\renewcommand{\thesubsection}{\arabic{subsection}}
\makeatother
\DeclareMathOperator*{\argmin}{argmin}
\newcommand{\partbreak}{\begin{center}\rule{17.5cm}{2pt}\end{center}}
\newcommand{\alignbreak}{\begin{center}\rule{15cm}{1pt}\end{center}}
\newcommand{\tightalignbreak}{\vspace{-5mm}\alignbreak\vspace{-5mm}}
\newcommand{\hop}{\vspace{1mm}}
\newcommand{\jump}{\vspace{5mm}}
\newcommand{\R}{\mathbb{R}}
\newcommand{\C}{\mathbb{C}}
\newcommand{\N}{\mathbb{N}}
\newcommand{\G}{\mathbb{G}}
\renewcommand{\S}{\mathbb{S}}
\newcommand{\bt}{\textbf}
\newcommand{\xdot}{\dot{x}}
\renewcommand{\star}{^{*}}
\newcommand{\ydot}{\dot{y}}
\newcommand{\lm}{\mathrm{\lambda}}
\renewcommand{\th}{\theta}
\newcommand{\id}{\mathbb{I}}
\newcommand{\si}{\Sigma}
\newcommand{\Si}{\si}
\newcommand{\inv}{^{-1}}
\newcommand{\T}{^\intercal}
\renewcommand{\tr}{\text{tr}}
\newcommand{\ep}{\varepsilon}
\newcommand{\ph}{\varphi}
%\renewcomand{\norm}[1]{\left\lVert#1\right\rVert}
\definecolor{cit}{rgb}{0.05,0.2,0.45}
\addtolength{\jot}{1em}
\newcommand{\solution}[1]{

\noindent{\color{cit}\textbf{Solution:} #1}}

\newcounter{tmpctr}
\newcommand\fancyRoman[1]{%
  \setcounter{tmpctr}{#1}%
  \setbox0=\hbox{\kern0.3pt\textsf{\Roman{tmpctr}}}%
  \setstackgap{S}{-.9pt}%
  \Shortstack{\rule{\dimexpr\wd0+.1ex}{.9pt}\\\copy0\\
              \rule{\dimexpr\wd0+.1ex}{.9pt}}%
}

\newcommand{\Id}{\fancyRoman{2}}

% Enter the specific assignment number and topic of that assignment below, and replace "Your Name" with your actual name.
\title{STAT 31410: Homework 3}
\author{Caleb Derrickson}
\date{October 25, 2023}

\begin{document}
\onehalfspacing
\maketitle

{\color{cit}\vspace{2mm}\noindent\textbf{Collaborators:}} The TA's of the class, as well as Kevin Hefner, and Alexander Cram.

\tableofcontents

\newpage
\section{Problem 1}
For this problem, let $\dot{x} = f(x)$, where $x \in \R$ and $f(x)$ is a $C^1$ function satisfying $f(0) = 0$, i.e. $x = 0$ is an equilibrium. 

\subsection{Problem 1, part a}
Find an $f(x)$ that shows that $x = 0$ may be unstable even though it appears to by Lyapunov stable for the corresponding linearized problem.

\partbreak
\begin{solution}

    For the sake of completion, I will provide the definition of Lyapunov stability.

    \alignbreak
    \vspace{-8mm}
    \begin{quote}
        If $x^*$ is an equilibrium point of the system $\dot{x} = f(x)$, then $x^*$ is Lyapunov stable if for every neighborhood $N$ of $x^*$ there exists a neighborhood $S\subset N$ such that $\forall \ x_0 \in S$, $x(t) \in N \ \forall \ t > 0$.   
    \end{quote}
    \vspace{-8mm}
    \alignbreak

    Our equilibrium point will be at 0, i.e. $f(x^*) = f(0) = 0$, so our function should match accordingly. If we let $f(x) = x^3$, then our linearized problem about $x = 0$ is just $\hat{f}(x) = $ 0. Thus, for any $x$ on which our function occupies will just be stationary, thus not attracted to the origin.
    However, we can notice that the non-linear system, $f(x) = x^3$ will be unstable for all points $\neq 0$, as solutions will be repelled from the origin. 
\end{solution}

\newpage
\subsection{Problem 1, part b}
Find an $f(x)$ that demonstrates that $x = 0$ may be asymptotically stable even when it is not linearly asymptotically stable.
\partbreak
\begin{solution}

    I will again include the definition of asymptotic stability.

    \alignbreak
    \vspace{-8mm}
    \begin{quote}
        If $x^*$ is an equilibrium point of the system $\dot{x} = f(x)$, then $x^*$ is asymptotically stable if it is Lyapunov stable and there exists a neighborhood $N$ such that if $x_0 \in N$, then $x(t) \rightarrow x^*$ as $t \rightarrow \infty$. 
    \end{quote}
    \vspace{-8mm}
    \alignbreak

    If we take $f(x) = -x^2\sin(x)$, then we see that there exists some neighborhood (roughly in the interval (-2.89, 2.89)) around 0 such that if $x_0 \in (-2.89, 2.89)$, then it will asymptotically approach $x^* = 0$. However, its linear term is zero, thus when solving the differential equation, we see that $x$ can be any constant function. Then the only point which will asymptotically approach $x^* = 0$ is $x_0 = 0$, which is not an open set, thus not Lyapunov stable, so not asymptotically stable. 
\end{solution} 

\newpage
\section{Problem 2}
The proof that linear asymptotic stability implies asymptotic stability in the text (Page 117) has a large number of positive constants introduced in it: $K, \delta, \ep, \alpha$. Describe the role of each in the proof, as well as any restrictions on them, ordering of them, etc. 

\partbreak
\begin{solution}

    \begin{itemize}
        \item $\alpha$ caps the eigenvalues of $A$ such that $\Re(\lm) < -\alpha < 0$, thereby ensuring all eigenvalues have real part less than zero.
        \item $K \geq 1$ is the bound obtained in Lemma 2.29, which bounds any solution with initial condition within the stable space of $A$. 
        \item $\ep$ and $\delta$ come from the fact that our system $f \in C^1$.
    \end{itemize}

    \jump
    From the above descriptions, their applications in the proof in various ways. In particular, the combination $K\delta$ bounds the distance from our solution to the equilibrium point, i.e., $|y| = |x - x^*| < K\delta$. Furthermore, by Gr\"{o}nwall's Lemma, this distance is further bounded in time by 
    \[
    |x - x^*| \leq K\delta e^{-(\alpha - K\ep)t} 
    \]
    Thus for any $\ep < \alpha K$, our solution is bounded. The proof then concludes with the statement, `` if S is the ball of radius $\delta$, then $N$ is the ball of radius $K\delta$", meaning Lyapunov stability is assured for neighborhood of radius $K\delta$ of our equilibrium $x^*$. 
\end{solution}

\newpage
\section{Problem 3}
Use the Hartman-Grobman theorem to prove that linear asymptotic stability implies asymptotic stability for an equilibrium $\textbf{x}^*$ of $\dot{\textbf{x}} = \textbf{f}(\textbf{x})$, where $\textbf{f}: \R^n \rightarrow \R^n$ is a $C^1$ vector filed. Can you also use it to show linear instability implies instability?

\partbreak

\begin{solution}
    
    Again, for completion, I will include the Hartman-Grobman here:

    \alignbreak
    \vspace{-8mm}
    \begin{quote}
        If $x^*$ is a hyperbolic equilibrium point of the system $\dot{x} = f(x)$, $f \in C^1$ with flow $\ph_t(x)$, the there is a neighborhood $N$ of $x^*$ such that $\ph_t(x)$ is topologically conjugate to its linearization on $N$.
    \end{quote}
    \vspace{-8mm}
    \alignbreak

    Since we are supposing an equilibrium $x^*$ is linear asymptotically stable, then all eigenvalues of the Jacobian $Df(x^*) = A$ has real part less than zero. Thus $x^*$ is a hyperbolic equilibrium of the linearized system. By Hartman-Grobman, we are guaranteed the existence of a neighborhood $N$ of $x^*$ such that the flow $\ph_t(x)$ is topologically conjugate to linearized counterpart $\psi_t(x)$ for any choice $x \in N$. Thus, picking $x_0 \in N$, we can say there exists a homeomorphism $h$ satisfying
    \[
    h(\ph_t(x_0)) = \psi_t(h(x_0)).
    \]

    Thus, we want to show that by our assumption of linearized asymptotic stability, our equilibrium $x_*$ is asymptotically stable. That is, for any $x_0 \in N$, $x(t) \rightarrow x^*$ as $t \rightarrow \infty$, where $x$ is any solution to our nonlinear system. Since $\ph_t(x_0)$ solves our system (as was demonstrated in class, as well as the texts), then we wish to show 
    \[
    \lim_{t \rightarrow \infty} \ph_t(x_0) = x^*.
    \]
    
    Note that since sour system is linearly asymptotically stable, then $h(x^*) = \lim_{t\rightarrow \infty} \psi_t(h(x_0))$. Thus the following steps are justified:

    \alignbreak
    \begin{align}
        h(x^*) &= \lim_{t \rightarrow \infty} \psi_t(h(x_0)) &\text{(By above.)}\nonumber\\
        &= \lim_{t \rightarrow \infty} h(\ph_t(x_0)) &\text{(Hartman-Grobman Theorem.)}\nonumber\\
        h(x^*) &= h(\lim_{t \rightarrow \infty} \ph_t(x_0)) &\text{(Continuity of $h$.)}\nonumber\\
        \implies x^* &= \lim_{t \rightarrow \infty} \ph_t(x_0) &\text{($h$ is one to one.)}\nonumber
    \end{align}
    \alignbreak

    \newpage
    Thus, it is shown that linear asymptotic stability implies asymptotic stability. Note linear instability means that there will exist some initial condition $x_0$ inside any neighborhood $N$ such that $\psi_t(h(x_0)) \notin N$ for some $t \in \R$ (note that Hartman-Grobman is still applicable in this case). As such this will imply $h(\ph_t(x_0)) \notin N \iff \ph_t(x_0) \notin N$ for any neighborhood $N$. Thus Linear instability implies instability.   
\end{solution}

\newpage
\section{Problem 4}
In section 4.8 of the textbook, on the Hartman-Grobman Theorem, there is an example of constructing a homeomorphism $H(x, y) = (x, y-x^2/3)$ that satisfies topological conjugacy condition $\psi_t \circ H = H \circ \ph_t$, where $\ph_t$ is generated by
\begin{align}
    \dot{x} &= x\nonumber\\
    \dot{y} &= -y+x^2.\nonumber
\end{align}

Solve these differential equations to obtain $\ph_t$ and $\psi_t$, where $\psi_t$ applies to the linearized flow. Then check, by explicit computation, that indeed $\psi_t \circ H = H \circ \ph_t$. 
\partbreak
\begin{solution}

    Solving this nonlinear system, we get

    \begin{align}
        \begin{pmatrix}x\\y\end{pmatrix} 
        \hspace{3mm} = \hspace{3mm}
        \begin{pmatrix}
            c_1 e^t\\
            \frac{1}{3} c_1^2 e^{2 t} + c_2 e^{-t}
        \end{pmatrix}\nonumber
    \end{align}

    Whereas in the linear system, i.e.
    \begin{align}
        \begin{pmatrix}\dot{x}\\\dot{y}\end{pmatrix} 
        \hspace{3mm} = \hspace{3mm}
        \begin{pmatrix}1 &0\\0 &-1\end{pmatrix}
        \begin{pmatrix}x\\y\end{pmatrix} 
        \nonumber
    \end{align}

    we see that 
    \begin{align}
        \begin{pmatrix}x\\y\end{pmatrix} 
        \hspace{3mm} = \hspace{3mm}
        \begin{pmatrix}
            c_1 e^t\\
            c_2 e^{-t}
        \end{pmatrix}\nonumber
    \end{align}

    Thus, our flow for the nonlinear system and the linear system are given above, where $c_1, c_2$ can be found by abusing what we know about the flow of a differential equation, i.e., 

    \[
    f(x) = \frac{d}{dt}\ph_t(x)\Big|_{t = 0}
    \]

    Where this coniditon will need to be applied to both systems independently. Applying this our nonlinear system, we get,

    \begin{align}
        \begin{pmatrix}c_1\\ \frac{2}{3}c_1^2 - c_2\end{pmatrix} 
        \hspace{3mm} = \hspace{3mm}
        \begin{pmatrix}x \\ -y + x^2\end{pmatrix} \nonumber
    \end{align}

    It is easy to see that $c_1 = x$. Then we can plug this into the second equation and, when simplifying, get $c_2 = y - \frac{1}{3}x^2$. Thus our flow for the nonlinear system is given as 

    \begin{align}
        \ph_t(x)
        \hspace{3mm} = \hspace{3mm}
        \begin{pmatrix}
            x_1 e^t\\
            \frac{1}{3} x_1^2 e^{2 t} + (x_2 - \frac{1}{3}x_1^2) e^{-t}
        \end{pmatrix}\label{p4: nonlinear flow}
    \end{align}

    A similar procedure can be done for the linear system, in which case, we get

    \begin{align}
        \begin{pmatrix}c_1\\ - c_2\end{pmatrix} 
        \hspace{3mm} = \hspace{3mm}
        \begin{pmatrix}x \\ -y\end{pmatrix} \nonumber
    \end{align}

    Then our linear flow $\psi_t(x)$ is given as 

    \begin{align}
        \psi_t(x)
        \hspace{3mm} = \hspace{3mm}
        \begin{pmatrix}
            x_1 e^t\\
            x_2e^{-t}
        \end{pmatrix}\label{p4: linear flow}
    \end{align}

    where $x = x_1, y = x_2$. Thus, by function composition, 

    \alignbreak
    \begin{align}
        \psi_t(H(x)) &= \psi_t \begin{pmatrix} x\\ y - \frac{1}{3}x^2\end{pmatrix}\nonumber =  \begin{pmatrix}x_1e^t\\ x_2e^{-t} - \frac{1}{3}x_1^2e^{2t}\end{pmatrix}\nonumber\\
        H(\ph_t(x)) &= H\begin{pmatrix}x_1e^t\\\frac{1}{3}x_1^2e^{2t} + \frac{1}{3}(3x_2 - x_1^2)e^{-t}\end{pmatrix} = \begin{pmatrix} x_1e^t\\ \frac{1}{3}x_1^2e^{2t} + \frac{1}{3}(3x_2 - x_1^2)e^{-t} - \frac{1}{3}x_1^2e^{2t}\end{pmatrix}\nonumber\\
        &= \begin{pmatrix}x_1e^t\\ x_2e^{-t} - \frac{1}{3}x_1^2e^{2t}\end{pmatrix}\nonumber
    \end{align}
    \alignbreak

    So we see that, indeed, $\psi_t \circ H = H \circ \ph_t$.
\end{solution}

\newpage
\section{Problem 5.6}
Which of the ODEs $\dot{x} = Ax$ with the following matrices are topologically conjugate? Which are diffeomorphic? Which are linearly conjugate?

\partbreak
\begin{solution}

    Given on page 130 of our text book, Theorem 4.33 states that two flows corresponding to two linearized systems $A, B$ are diffeomorphic iff the matrices $A, B$ are similar. That is, $A$ and $B$ are the same linear mapping, just expressed in different bases. Therefore, all we are looking for is similarity between matrices. One surefire way to check this is if we wrote the Jordan decomposition of all matrices, but there will be easier checks do do first. Two similar matrices will have the same trace, rank, determinant, and eigenvalues (this is not an exhaustive list). If one of these fails to hold, then we can say they are not similar. Thus, I will write out a table of these matrices and their respective values. I will label each matrix by its part letter

\begin{table}[h]
  \centering
  \caption{Matrix Similarity}
  \begin{tabularx}{\textwidth}{|>{\columncolor{gray!25}}X|X|X|X|X|X|}
    \hline
    \rowcolor{gray!25}Matrix & Trace & Rank & Determinant & Eigenvalues & Similar? \\
    \hline
    (a) & 0 & 2 & -1 & -1, 1 & h\\
    \hline
    (b) & 4 & 2 & 4 & 2, 2 &f?\\
    \hline
    (c) & -4 & 2 & 5 & $\sqrt{19} + 2, \sqrt{19} - 2$ &None \\
    \hline
    (d) & 4 & 2 & 2 & 3, 1 &None\\
    \hline
    (e) & -1 & 2 & -6 & -3, 2 &None\\
    \hline
    (f) & 4 & 2 & 4 & 2, 2 &b?\\
    \hline
    (g) & -5 & 2 & 6 & -3, -2 &None\\
    \hline
    (h) & 0 & 2 & -1 & -1, 1 &a\\
    \hline
  \end{tabularx}
\end{table}

Note that if two matrices share all same values, it doesn't necessarily mean they are similar. Only when both matrices are diagonalizable are they similar. Note that the pair (a) and (h) are similar, since they have equal trace, rank, determinant, eigenvalues, and are diagonalizable. The pair (b) and (f), although seem to be similar, are not since (f) is not diagonalizable. Thus the only pair which are topologically conjugate, diffeomorphic, and linearly conjugate are (a) and (h).  
\end{solution}
\newpage
\section{Problem 5.7}
Construct a topological conjugacy between the linear systems with the matrices
\[
A = \begin{pmatrix}1 &-1\\ 1&1\end{pmatrix}, \hspace{5mm} B = \begin{pmatrix}2 &0\\ 0&2\end{pmatrix}
\]
\partbreak

\begin{solution}

    Solving the first system, we first note the eigenvalues and eigenvectors. By observation,

    \begin{align}
        &\lm_1 = 1 + i, &\textbf{v}_1 = \begin{pmatrix} \hspace{3mm} i  &1\end{pmatrix}^t\nonumber\\
        &\lm_2 = 1 - i, &\textbf{v}_2 = \begin{pmatrix} -i &1\end{pmatrix}^t\nonumber
    \end{align}


    Thus, by the fundamental matrix, our general solution is the following: 
    
    \begin{align}
        \begin{pmatrix} x \\ y\end{pmatrix} \hspace{3mm} = \hspace{3mm} 
        \begin{pmatrix}
            c_1e^t\cos t - c_2e^t\sin t\\
            c_1e^t\sin t +\ c_2e^t\cos t
        \end{pmatrix}\label{p5.7: general soln}
    \end{align}

    Thus, by our reasoning in problem 4, we can remove these constants $c_1, c_2$ via initial conditions to retrieve our flow $\ph$. 

    \alignbreak
    \begin{align}
        x - y &= \frac{d}{dt}\ph_t \Big|_{t = 0}\nonumber\\
        &= c_1( - e^t\sin t + \cos t e^t) - c_2(e^t \cos t + e^t \sin t) \Big|_{t = 0}\nonumber\\
        x - y&= c_1 - c_2\nonumber\\
        x + y&= \frac{d}{dt}\ph_t \Big|_{t = 0}\nonumber\\
        &= c_1(e^t \cos t + e^t \sin t) + c_2 (-\sin t e^t + e^t \cos t)\Big|_{t = 0}\nonumber\\
        x + y&= c_1 + c_2 \nonumber
    \end{align}
    \alignbreak


    We can easily see that $c_1 = x, c_2 = y$. Thus our flow $\ph_t(x)$

    \begin{align}
        \ph_t(x)  \hspace{3mm} = \hspace{3mm} \begin{pmatrix} \ph_{t1}(x) \\ \ph_{t2}(x)\end{pmatrix} \hspace{3mm} = \hspace{3mm} 
        \begin{pmatrix}
            x_1e^t\cos t -  x_2e^t\sin t\\
            x_1e^t\sin t +\ x_2e^t\cos t
        \end{pmatrix}\label{p5.7: Aflow}
    \end{align}

    The second system is fortunately easier to find. Since the system is a multiple of the identity matrix, we can see the following:

        \begin{align}
        &\lm_1 = 2, &\textbf{v}_1 = \begin{pmatrix}  1  &0\end{pmatrix}^t\nonumber\\
        &\lm_2 = 2, &\textbf{v}_2 = \begin{pmatrix} 0 &1\end{pmatrix}^t\nonumber
    \end{align}

    Thus our flow $\psi_t(x)$, with constants, equals

    \begin{align}
        \begin{pmatrix} x \\ y\end{pmatrix} \hspace{3mm} = \hspace{3mm} 
        \begin{pmatrix}
            c_1e^{2t}\\
            c_2e^{2t}
        \end{pmatrix}\label{p5.7: general soln psi}
    \end{align}

    Following the previous steps, we see that $x = c_1, y = c_2$. Thus, 

        \begin{align}
        \psi_t(x)  \hspace{3mm} = \hspace{3mm} \begin{pmatrix} \psi_{t1}(x) \\ \psi_{t2}(x)\end{pmatrix} \hspace{3mm} = \hspace{3mm} 
        \begin{pmatrix}
            x_1e^{2t}\\
            x_2e^{2t}
        \end{pmatrix}\label{p5.7: Bflow}
    \end{align}

    Then the two flows between which we wish to find a homeomorphism, are 

    \alignbreak
    \[
    \ph_t(x)  =  
        \begin{pmatrix}
            x_1e^t\cos t -  x_2e^t\sin t\\
            x_1e^t\sin t +\ x_2e^t\cos t
        \end{pmatrix}\
        ,
        \hspace{5mm}
    \psi_t(x)   =   
        \begin{pmatrix}
            x_1e^{2t}\\
            x_2e^{2t}
        \end{pmatrix}\
    \]
    \alignbreak

\newpage
    Transforming these into polar coordinates, the following steps are justified:

    \alignbreak
    \begin{align}
        \ph_t(x) &= 
        \begin{pmatrix}
            r\cos \th e^t \cos t - r \sin \th e^t \sin t\\
            r\cos \th e^t \sin t + r \sin \th e^t \cos t
        \end{pmatrix} &\text{(Using $x_1 = r\cos \th, x_2 = r\sin \th$.)}\nonumber\\
        &= \begin{pmatrix}
            re^t(\cos \th  \cos t - \sin \th \sin t)\\
            re^t(\cos \th  \sin t + \sin \th \cos t)
        \end{pmatrix} &\text{(Rearranging.)}\nonumber\\
        \ph_t(x) &= \begin{pmatrix}
            re^t\cos(\th + t)\\
            re^t\sin (\th  + t) 
        \end{pmatrix} &\text{(Trig - Addition identities.)}\nonumber\\
        \psi_t(x) &=   
        \begin{pmatrix}
            re^{2t}\cos \th \\
            re^{2t}\sin \th 
        \end{pmatrix}&\text{(Using $x_1 = r\cos \th, x_2 = r\sin \th.$)}\nonumber
    \end{align}

    These are still ``encoded" in cartesian coordinates, although written in terms of $r$ and $\th$. I.e., we want to rewrite these in terms of \textbf{r}, rather than \textbf{x}. If $\ph_t(r, \th) = \ph_t(x, y)$, then we can say the following:
    \begin{align}
        \ph_{t1}(\textbf{r}) &= \sqrt{\ph_{t1}^2(x) + \ph_{t2}^2(x)} = re^t\sqrt{\cos^2(\th + t) + \sin^2(\th + t)} = re^t \nonumber\\
        \ph_{t2}(\textbf{r}) &= \arctan\bigg( \frac{\ph_{t2}(x)}{\ph_{t1}(x)} \bigg) = \arctan\bigg( \frac{\sin(\th + t)}{\cos (\th + t)} \bigg) = \th + t\nonumber\\
        \psi_{t1}(\textbf{r}) &= \sqrt{\psi_{t1}^2(x) + \psi_{t2}^2(x)} = re^{2t}\sqrt{\cos^2(\th) + \sin^2(\th)} = re^{2t}\nonumber\\
        \psi_{t2}(\textbf{r}) &= \arctan\bigg( \frac{\psi_{t2}(x)}{\psi_{t1}(x)} \bigg) = \arctan\bigg( \frac{\sin\th}{\cos \th} \bigg) = \th \nonumber
    \end{align}


    Note that I took the value of $\th \in [-\pi/2, \pi/2]$ implicitly. Therefore, in polar coordinates, we can write,
    \alignbreak
    \begin{align}
            \ph_t(r)  =  
        \begin{pmatrix}
            re^t\\
            \th + t
        \end{pmatrix}
        ,
        \hspace{5mm}
    \psi_t(r)   =   
        \begin{pmatrix}
            re^{2t}\\
            \th
        \end{pmatrix}\nonumber
    \end{align}
    \alignbreak

    \newpage
    Therefore, in order to find a homeomorphism $H$, it should satisfy $\psi_t \circ H = H \circ \ph_t$. Writing this out explicitly, 
    \[
    H(\psi_t(\textbf{r})) = \ph_t(H(\textbf{r})).
    \]

    Thus the following steps are justified:

    \alignbreak
    \begin{align}
        \begin{pmatrix}H_r(re^{2t})\\ H_\th(re^{2t}, \th)\end{pmatrix} &= \begin{pmatrix}H_r(r)e^t\\H_\th(r, \th)+ t\end{pmatrix} &\text{(We were hinted $H$ is of this form.)}\nonumber\\
        H_r(re^{2t}) &= H_r(r)e^t &\text{(First coordinate.)}\nonumber\\
        H_r(re^{2t})^{-1}&= H^{-1}(H(r)e^t) &\text{(Taking inverse of $H$.)}\nonumber\\
        \implies& H_r(r) = \sqrt{r} &\text{($H^{-1}$ should undo $H$, this will work.)}\nonumber\\
        H_\th(re^{2t}, \th) &= H_\th(r, \th) + t &\text{(Second coordinate.)}\nonumber\\
        \implies& H_\th(r, \th) = \frac{1}{2}\ln r + \th &\text{(Taking in $re^{2t}$, should output $\th$.)}\nonumber
    \end{align}
    \alignbreak

    Therefore, our homeomorphism $H$ is found as 
    \[
    H = \begin{pmatrix}H_r(r)\\H_\th(r, \th)\end{pmatrix} = \begin{pmatrix}\sqrt{r}\\ \frac{1}{2}\ln r + \th\end{pmatrix}
    \]

    where it is understood that $r > 0$. On this region, both components of $H$ are one to one (square root is one to one, as well as natural log and the identity) and onto (there will be some value in $\R$ onto which each component will map). Thus, the mapping is bijective, with is inverse given by 
    \[
    H^{-1} = \begin{pmatrix}r^2\\ -\frac{1}{2}\ln r + \th\end{pmatrix}.
    \]

    As a sanity check, we will check explicitly that $\psi_t \circ H = H \circ \ph_t$.

    \newpage
    \alignbreak
    \begin{align}
        \psi_t(H(r)) &= \psi_t\begin{pmatrix}\sqrt{r}\\ \frac{1}{2}\ln r + \th\end{pmatrix}\nonumber\\
                     &= \begin{pmatrix}\sqrt{r} e^t \\ \frac{1}{2}\ln r + t + \th \end{pmatrix}\nonumber\\
        H(\ph_t(r))  &= H\begin{pmatrix}re^t\\ \th + t\end{pmatrix}\nonumber\\
                     &= \begin{pmatrix}\sqrt{r}e^t\\ \frac{1}{2}\ln r + \th + t\end{pmatrix}\nonumber
    \end{align}
    \alignbreak

    Thus we see that $H$ is a homeomorphism.

\end{solution}

\end{document}
