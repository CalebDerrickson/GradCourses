\documentclass[12pt]{article}
\usepackage[paper=letterpaper,margin=1.5cm]{geometry}
\usepackage{amsmath}
\usepackage{amssymb}
\usepackage{amsfonts}
\usepackage{mathtools}

\usepackage{lmodern}     % set math font to Latin modern math
\usepackage[T1]{fontenc}
\renewcommand\rmdefault{ptm}
\usepackage[shortlabels]{enumitem}
\usepackage{titling}
\usepackage{graphicx}
\usepackage[colorlinks=true]{hyperref}
\usepackage{setspace}
\usepackage{subfigure} 
\usepackage{braket}
\usepackage{color}
\usepackage{tabularx}
\usepackage[table]{xcolor}
\usepackage{listings}
\usepackage{mathrsfs}
\usepackage{stackengine}
\usepackage{physics}
\usepackage{afterpage}
\usepackage{tikz}
\usepackage{pdfpages}
\usepackage[export]{adjustbox}
\usepackage{biblatex}

\setstackEOL{\\}

\definecolor{dkgreen}{rgb}{0,0.6,0}
\definecolor{gray}{rgb}{0.5,0.5,0.5}
\definecolor{mauve}{rgb}{0.58,0,0.82}


\lstset{frame=tb,
  language=Python,
  aboveskip=3mm,
  belowskip=3mm,
  showstringspaces=false,
  columns=flexible,
  basicstyle={\small\ttfamily},
  numbers=none,
  numberstyle=\tiny\color{gray},
  keywordstyle=\color{blue},
  commentstyle=\color{dkgreen},
  stringstyle=\color{mauve},
  breaklines=true,
  breakatwhitespace=true,
  tabsize=3
}
\setlength{\droptitle}{-6em}

\makeatletter
% we use \prefix@<level> only if it is defined
\renewcommand{\@seccntformat}[1]{%
  \ifcsname prefix@#1\endcsname
    \csname prefix@#1\endcsname
  \else
    \csname the#1\endcsname\quad
  \fi}
% define \prefix@section
\newcommand\prefix@section{}
\newcommand{\prefix@subsection}{}
\newcommand{\prefix@subsubsection}{}
\renewcommand{\thesubsection}{\arabic{subsection}}
\makeatother
\DeclareMathOperator*{\argmin}{argmin}
\newcommand{\partbreak}{\begin{center}\rule{17.5cm}{2pt}\end{center}}
\newcommand{\alignbreak}{\begin{center}\rule{15cm}{1pt}\end{center}}
\newcommand{\tightalignbreak}{\vspace{-5mm}\alignbreak\vspace{-5mm}}
\newcommand{\hop}{\vspace{1mm}}
\newcommand{\jump}{\vspace{5mm}}
\newcommand{\R}{\mathbb{R}}
\newcommand{\C}{\mathbb{C}}
\newcommand{\N}{\mathbb{N}}
\newcommand{\G}{\mathbb{G}}
\renewcommand{\S}{\mathbb{S}}
\newcommand{\bt}{\textbf}
\newcommand{\xdot}{\dot{x}}
\newcommand{\ydot}{\dot{y}}
\newcommand{\lm}{\mathrm{\lambda}}
\renewcommand{\th}{\theta}
\newcommand{\id}{\mathbb{I}}
\newcommand{\si}{\Sigma}
\newcommand{\Si}{\si}
\newcommand{\inv}{^{-1}}
\newcommand{\T}{^{\intercal}}
\renewcommand{\tr}{\text{tr}}
\newcommand{\ep}{\varepsilon}
\newcommand{\ph}{\varphi}
\newcommand{\range}{\text{range}}
\newcommand{\scP}{\mathcal{P}}
\newcommand{\scT}{\mathbb{T}}
\newcommand{\into}{\rightarrow}
\newcommand{\scM}{\mathcal{M}}
\newcommand{\scH}{\mathcal{H}}
\newcommand{\scN}{\mathcal{N}}
\newcommand{\scV}{\mathcal{V}}
\newcommand{\scW}{\mathcal{W}}
\renewcommand{\grad}{\nabla}
\renewcommand{\star}{^{*}}

\definecolor{cit}{rgb}{0.05,0.2,0.45}
\addtolength{\jot}{1em}
\newcommand{\solution}[1]{


\noindent{\color{cit}\textbf{Solution:} #1}}

\newcounter{tmpctr}
\newcommand\fancyRoman[1]{%
  \setcounter{tmpctr}{#1}%
  \setbox0=\hbox{\kern0.3pt\textsf{\Roman{tmpctr}}}%
  \setstackgap{S}{-.9pt}%
  \Shortstack{\rule{\dimexpr\wd0+.1ex}{.9pt}\\\copy0\\
              \rule{\dimexpr\wd0+.1ex}{.9pt}}%
}

\newcommand{\Id}{\fancyRoman{2}}

% Enter the specific assignment number and topic of that assignment below, and replace "Your Name" with your actual name.
\title{STAT 31410: Homework 4}
\author{Caleb Derrickson}
\date{November 3, 2023}

\begin{document}
\onehalfspacing
\maketitle

{\color{cit}\vspace{2mm}\noindent\textbf{Collaborators:}} The TA's of the class, as well as Kevin Hefner, and Alexander Cram.

\tableofcontents

\newpage
\section{Problem 1}

\newpage
\section{Problem 2}
This is meant to be a simple problem in which you numerically computer a Poincar\'e return map, and use it to determine the Floquet multiplier associated with a limit cycle. Let

\begin{align}
    \xdot &= 2y\nonumber\\
    \ydot &= -2x + \frac{1}{2}(1 - x^2)y\nonumber
\end{align}

Define a surface of section as $\Si = \{(x, y) : x \in [1, 3], y = 0\}$ for your return map, and numerically compute the Poincar\'e return map $P: \si \rightarrow \si $ by computing orbits for initial conditions in $(x_0, 0) \in \si$, and finding the $x$ values where they next return to $\si$, i.e. if the next point is $(x_1, 0) \in \si$, then the map should satisfy $x_1 = P(x_0)$ for each $x_0 \in [1, 3]$. You will find that the map has a fixed point $(P(x^*) = x^*)$, and the $x^*$ value associated with it gives a point $(x^*, 0)$ on a limit cycle for the flow. You can further obtain an estimate of the Floquet multipliers associated with the limit cycle by estimating $\mu = P'(x^*)$. How does this compare with what you obtain from the Monodromy matrix? (According to results proved in Meiss, you should find that the eigenvalues of the Monodromy matrix are exactly $\mu, 1$. You can either prove this for this 2-d example, or compute the Monodromy matrix numerically to confirm the claim.)
\end{document}