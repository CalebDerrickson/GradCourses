\documentclass[12pt]{article}
\usepackage[paper=letterpaper,margin=1.5cm]{geometry}
\usepackage{amsmath}
\usepackage{amssymb}
\usepackage{amsfonts}
\usepackage{mathtools}
%\usepackage[utf8]{inputenc}
%\usepackage{newtxtext, newtxmath}
\usepackage{lmodern}     % set math font to Latin modern math
\usepackage[T1]{fontenc}
\renewcommand\rmdefault{ptm}
%\usepackage{enumitem}
\usepackage[shortlabels]{enumitem}
\usepackage{titling}
\usepackage{graphicx}
\usepackage[colorlinks=true]{hyperref}
\usepackage{setspace}
\usepackage{subfigure} 
\usepackage{braket}
\usepackage{color}
\usepackage{tabularx}
\usepackage[table]{xcolor}
\usepackage{listings}
\usepackage{mathrsfs}
\usepackage{stackengine}
\usepackage{physics}
\usepackage{afterpage}
\usepackage{pdfpages}
\usepackage[export]{adjustbox}
\usepackage{biblatex}

\setstackEOL{\\}

\definecolor{dkgreen}{rgb}{0,0.6,0}
\definecolor{gray}{rgb}{0.5,0.5,0.5}
\definecolor{mauve}{rgb}{0.58,0,0.82}


\lstset{frame=tb,
  language=Python,
  aboveskip=3mm,
  belowskip=3mm,
  showstringspaces=false,
  columns=flexible,
  basicstyle={\small\ttfamily},
  numbers=none,
  numberstyle=\tiny\color{gray},
  keywordstyle=\color{blue},
  commentstyle=\color{dkgreen},
  stringstyle=\color{mauve},
  breaklines=true,
  breakatwhitespace=true,
  tabsize=3
}
\setlength{\droptitle}{-6em}

\makeatletter
% we use \prefix@<level> only if it is defined
\renewcommand{\@seccntformat}[1]{%
  \ifcsname prefix@#1\endcsname
    \csname prefix@#1\endcsname
  \else
    \csname the#1\endcsname\quad
  \fi}
% define \prefix@section
\newcommand\prefix@section{}
\newcommand{\prefix@subsection}{}
\newcommand{\prefix@subsubsection}{}
\renewcommand{\thesubsection}{\arabic{subsection}}
\makeatother
\DeclareMathOperator*{\argmin}{argmin}
\newcommand{\partbreak}{\begin{center}\rule{17.5cm}{2pt}\end{center}}
\newcommand{\alignbreak}{\begin{center}\rule{15cm}{1pt}\end{center}}
\newcommand{\tightalignbreak}{\vspace{-5mm}\alignbreak\vspace{-5mm}}
\newcommand{\hop}{\vspace{1mm}}
\newcommand{\jump}{\vspace{5mm}}
\newcommand{\R}{\mathbb{R}}
\newcommand{\C}{\mathbb{C}}
\newcommand{\N}{\mathbb{N}}
\newcommand{\G}{\mathbb{G}}
\renewcommand{\S}{\mathbb{S}}
\newcommand{\bt}{\textbf}
\newcommand{\xdot}{\dot{x}}
\renewcommand{\star}{^{*}}
\newcommand{\ydot}{\dot{y}}
\newcommand{\lm}{\mathrm{\lambda}}
\renewcommand{\th}{\theta}
\newcommand{\id}{\mathbb{I}}
\newcommand{\si}{\Sigma}
\newcommand{\Si}{\si}
\newcommand{\inv}{^{-1}}
\newcommand{\T}{^\intercal}
\renewcommand{\tr}{\text{tr}}
\newcommand{\ep}{\varepsilon}
\newcommand{\ph}{\varphi}
%\renewcomand{\norm}[1]{\left\lVert#1\right\rVert}
\definecolor{cit}{rgb}{0.05,0.2,0.45}
\addtolength{\jot}{1em}
\newcommand{\solution}[1]{

\noindent{\color{cit}\textbf{Solution:} #1}}

\newcounter{tmpctr}
\newcommand\fancyRoman[1]{%
  \setcounter{tmpctr}{#1}%
  \setbox0=\hbox{\kern0.3pt\textsf{\Roman{tmpctr}}}%
  \setstackgap{S}{-.9pt}%
  \Shortstack{\rule{\dimexpr\wd0+.1ex}{.9pt}\\\copy0\\
              \rule{\dimexpr\wd0+.1ex}{.9pt}}%
}

\newcommand{\Id}{\fancyRoman{2}}

% Enter the specific assignment number and topic of that assignment below, and replace "Your Name" with your actual name.
\title{STAT 30900: Homework 1}
\author{Caleb Derrickson}
\date{October 20, 2023}

\begin{document}
\onehalfspacing
\maketitle

{\color{cit}\vspace{2mm}\noindent\textbf{Collaborators:}} The TA's of the class, as well as Kevin Hefner, and Steven Lee.

\tableofcontents

\newpage
\section{Problem 1}

\subsection{Problem 1, part a} Let $A \in \C^{m \times n}$ and $1 \leq p \leq \infty$. Find the closed form expressions for 
\[
\lVert A \rVert_{1, p} := \max_{\textbf{x} \neq \textbf{0}} \frac{\lVert A\textbf{x} \rVert_{p}}{\lVert \textbf{x} \rVert_1} \hspace{5mm} \text{and} \hspace{5mm} \lVert A \rVert_{p, \infty} := \max_{\textbf{x} \neq \textbf{0}} \frac{\lVert A\textbf{x} \rVert_{\infty}}{\lVert \textbf{x} \rVert_p} 
\]
Your expressions should agree with the matrix 1-norm and matrix $\infty$-norm when $p = 1$ and $\infty$ respectively.
\partbreak
\begin{solution}

    Let $A_k$ denote the k-th column of $A$. Then, 
    \alignbreak
    \begin{align}
        \norm{A\textbf{x}}_p &= \norm{\sum_{k = 1}^n x_kA_k}_p &\text{(Definition of vector norm.)}\nonumber\\
        &\leq \sum_{k = 1}^n \norm{x_kA_k}_p &\text{(Triangle Inequality.)}\nonumber\\
        &= \sum_{k = 1}^n|x_k|\norm{A_k}_p &\text{(Absolute Homogeneity of norm.)}\nonumber\\
        &\leq \max \{ \norm{A_k}_p : 1 \leq k \leq n \} \Bigg(\sum_{k = 1}^n |x_k| \Bigg) &\text{(All columns of A are $\leq$ max column.)}\nonumber\\
        &=\max \{ \norm{A_k}_p : 1 \leq k \leq n \} \norm{\textbf{x}_1} &\text{(Definition of 1-norm.)}\nonumber\\
        \implies \frac{\norm{A\textbf{x}}_p}{\norm{\textbf{x}}_1} &\leq \max \{\norm{A_k}_p : 1 \leq k \leq n \} &\text{(Rearranging.)}\label{p1a:pnorm}
    \end{align}
    \alignbreak

    From \ref{p1a:pnorm}, this means the argument inside the max is less than or equal to the max column of $A$. The question now is when this is equal to $\norm{A}_{1,p}$. We are free to choose any nonzero \textbf{x}, so a natural one to choose would be the one to pick the max column when multiplied to $A$. Meaning if \textbf{x} $= e_M$, where M is the index corresponding to the max column of $A$, then equality it obtained. Note this is given as the definition of $\norm{A}_1$ when $p = 1$. 
\newpage
    We next wish to find a closed form expression for $\norm{A}_{p, \infty}$. It follows the same processes done in the previous section, with some slight modifications.

    \alignbreak
    \begin{align}
        \norm{A\textbf{x}}_\infty &= \max \{ |(A\textbf{x})_j | : 1 \leq j \leq n\} &\text{(Given.)}\nonumber\\
        &= \max \{ \Big|\sum_{k = 1}^nx_kA_{ik}\Big| : 1 \leq i \leq m\} &\text{(Matrix-vector multiplication.)}\nonumber\\
        &\leq \max \{ \sum_{k = 1}^n|x_kA_{ik}| : 1 \leq i \leq m\} &\text{(Triangle inequality.)}\nonumber\\
        &= \max \{ \sum_{k = 1}^n|x_k||A_{ik}| : 1 \leq i \leq m\} &\text{(Absolute Homogeneity.)}\nonumber\\
        &= \sum_{k = 1}^n|x_k||A_{Mk}| &\text{($M$ is max row of $A$.)}\nonumber\\
        &\leq |x_M|\sum_{k = 1}^n|A_{Mk}| &\text{($M$ is max element of \textbf{x}.)}\nonumber\\
        &= \norm{\textbf{x}}_\infty\sum_{k = 1}^n|A_{Mk}| &\text{(Definition of $\infty$-norm.)}\nonumber\\
        &\leq \norm{\textbf{x}}_p\sum_{k = 1}^n|A_{Mk}| &\text{($\infty$-norm is included in all $p$-norms.)}\nonumber\\
        \implies \norm{A\textbf{x}}_\infty &\leq \norm{\textbf{x}}_p\norm{A}_\infty &\text{(Definition of $\infty$-norm for matrices.)}\nonumber
    \end{align}
    \alignbreak

    Thus, we can see $\norm{A}_{p,\infty}$ is the max of $\norm{A}_\infty$, which is just $\norm{A}_\infty$. We note that $\norm{A}_\infty$ is the max row sum of the matrix $A$. Thus, both expressions have been found.
\end{solution}

\newpage
\subsection{Problem 1, part b}
Let $\S^n := \{ A \in \R^{n\times n} : A^T = A \}$. Recall the Gram matrix from Homework \textbf{0}, Problem \textbf{4}:
\\
For \textbf{x}$_1$, ..., \textbf{x}$_n \in \R^n$, we write

\[
G(\textbf{x}_1, ...,\textbf{x}_n) := 
\begin{bmatrix}
    \textbf{x}_1^T\textbf{x}_1 &\textbf{x}_1^T\textbf{x}_2 &... &\textbf{x}_1^T\textbf{x}_n\\
    \textbf{x}_2^T\textbf{x}_1 &\textbf{x}_2^T\textbf{x}_2 &... &\textbf{x}_2^T\textbf{x}_n\\
    \vdots    &\vdots     &\ddots    &\vdots\\
    \textbf{x}_n^T\textbf{x}_1 &\textbf{x}_n^T\textbf{x}_2 &... &\textbf{x}_n^T\textbf{x}_n\\
\end{bmatrix}
\in \S^n.
\]

Consider the set $\G^n := \{G(\textbf{x}_1, ..., \textbf{x}_n) \in \S^n : \norm{\textbf{x}_1}_2 \leq 1, ..., \norm{\textbf{x}_n}_2 \leq 1 \}$. Prove that for any $A \in \S^n$,

\[
\norm{A}_G := \max \{ |\text{tr}(AG)| : G \in \G^n \}
\]

defines a norm on $S^n$. Show that if $A$ = diag($a_{11}, ..., a_{nn}$) $\in \S^n$, then 
\[
\norm{A}_G = \max\bigg( \sum_{i = 1}^n \max (a_{ii}, 0), -\sum_{i = 1}^n \min (a_{ii}, 0)\bigg).
\]

\partbreak
\begin{solution}
    
\end{solution}
\newpage
\section{Problem 2}
Let $A \in \C^{n\times n}$ and $\lVert \cdot \rVert_p : \C^{n \times n} \rightarrow [0, \infty )$ be the matrix $p$-norm for some $p \in [1, \infty]$. 

\subsection{Problem 2, part a}
Show that if $\norm{A}_p < 1$, then $\Id - A$ is invertible and furthermore, 
\begin{align}
\frac{1}{1 + \lVert A\rVert_p} \leq \lVert(\Id - A)^{-1}\rVert_p \leq \frac{1}{1 - \lVert A\rVert_p}. \label{p2a:chain ineq}    
\end{align}
\partbreak
\begin{solution}

    Let \textbf{v} $\in \C^{n\times n} \backslash \{0\}$ be an eigenvector of $A$ with eigenvalue $\lm$. Then $A\textbf{v} = \lm \textbf{v} \iff \norm{A\textbf{v}}_p = \norm{\lm\textbf{v}}_p$. Note the matrix $p$-norm is taken as a max over all \textbf{x} $\neq 0$. Along with $\norm{A} < 1$, we get, 
    \[
    1 > \max_{\textbf{x} \neq 0}\frac{\norm{A\textbf{x}}}{\norm{\textbf{x}}} \leq \frac{\norm{A\textbf{v}}}{\norm{\textbf{v}}}
    \]
    then $\norm{A\textbf{v}} < \norm{\textbf{v}}$. So 1 is then not an eigenvector of $A$. This implies that $(A - \id)\textbf{v} = (\lm - 1)\textbf{v}$. Note that \textbf{v} was taken as a nonzero vector, so $(1 - \lm)\textbf{v} \neq 0$. This means that 0 is not an eigenvalue of $A - \id$, making the determinant nonzero, thus $A - \id$ is invertible.  

    \jump
    We then wish to show \ref{p2a:chain ineq},
    
    \[
    \frac{1}{1 + \lVert A\rVert_p} \leq \lVert(\Id - A)^{-1}\rVert_p \leq \frac{1}{1 - \lVert A\rVert_p}.
    \]
    
    \jump
    To prove the first inequality, we note that if \textbf{v} is an eigenvector of $A$ with eigenvalue $\lm$, (and $A$ invertible) then \textbf{v} is an eigenvector of $A$ with eigenvalue $\lm^{-1}$. This is shown by the following:
    
    \[
    A\textbf{v} = \lm\textbf{v} \iff A^{-1}A\textbf{v} = A^{-1}\lm\textbf{v} \iff A^{-1}\textbf{v} = \frac{1}{\lm} \textbf{v}.
    \]

    Thus, take \textbf{v} as an eigenvector of $A$ with eigenvalue $(1 - \lm)$. The above claim implies that \textbf{v} is also an eigenvector of $(\id - A)^{-1}$ with eigenvalue $(1 - \lm)^{-1}$. Furthermore, since $\norm{\cdot}_p$ is consistent, we have that the spectral radius $\rho(A) = \max\{ |\lm(A)|\}$ is less than or equal to $\norm{A}$, ie $\rho(A) \leq \norm{A}$. The following steps can then be justified

    \alignbreak
    \begin{align}
        &(\id - A)^{-1}\textbf{v} = \frac{1}{1 - |\lm|}\textbf{v}   &\text{(Given.)}\nonumber\\
        &\norm{(\id - A)^{-1}\textbf{v}} = \frac{1}{1 - |\lm|}\norm{\textbf{v}} &\text{(Taking norm and eigenvalue is positive.)}\nonumber\\
        &\norm{(\id - A)^{-1}\textbf{v}} \leq \norm{(\id - A)^{-1}}\norm{\textbf{v}}    &\text{(Norm Consistency of $\norm{\cdot}_p$.)} \label{p2a:normconsistency}\\
        \implies &\norm{(\id - A)^{-1}} > \frac{1}{1 - |\lm |} &\text{(Applying \ref{p2a:normconsistency} and cancelling $\norm{v}$)}\nonumber\\
        &\norm{(\id - A)^{-1}} > \frac{1}{1 + |\lm |} &\text{($1 - |\lm | < 1 + |\lm |$.)} \nonumber\\
        &\norm{(\id - A)^{-1}} \geq \frac{1}{1 + \norm{A}_p} &\text{($\rho(A) \leq \norm{A}$.)}\nonumber
    \end{align}
    \alignbreak

    This solves the first inequality of \ref{p2a:chain ineq}. We then can show the second inequality via the Taylor expansion of $\norm{(\id - A)^{-1}}_p$. Noting further that $\norm{A}_p < 1$, the following steps can be taken:

    \alignbreak
    \begin{align}
        &\norm{(\id - A)^{-1}}_p &\text{(Given.)}\nonumber\\
        &= \norm{\sum_{k = 0}^\infty A^k}_p &\text{(Taylor Expansion.)}\nonumber\\
        &\leq \sum_{k = 0}^\infty \norm{A^k}_p &\text{(Triangle inequality.)}\nonumber\\
        &\leq\sum_{k = 0}^\infty\norm{A}^k_p &\text{(Submultiplication of $\norm{\cdot}_p$.)}\nonumber\\
        \implies \norm{(\id - A)^{-1}}_p&\leq \frac{1}{1 - \norm{A}_p} &\text{(Taylor Expansion Equivalent.)}\nonumber
    \end{align}
    \alignbreak

    This proves both inequalities, thus \ref{p2a:chain ineq} has been shown. $\square$
\end{solution}

\subsection{Problem 2, part b}
Suppose $A$ is invertible. show that and $X \in \C^{n\times n}$ with
\[
\norm{X - A}_p < \frac{1}{\norm{A^{-1}}_p}
\]
must also be invertible. 
\partbreak

\begin{solution}

    We first should note that if a matrix product $AB$ is invertible, then both $A$ and $B$ are invertible. This can be reasoned by a rank argument: if $AB$ is invertible, then $AB$ has full rank. If $B$ mapped anything other than \textbf{0} to \textbf{0}, then $AB$ would map a nonzero vector to zero, which is a contradiction. Thus, $B$ has full rank, meaning that any vector in its target space can be represented by an element in the domain of $B$ scaled by matrix $B$ (i.e. $B\textbf{x} = \textbf{y}$). If we supposed that $A$ mapped $\textbf{y}$ to the zero vector for some arbitrary $\textbf{y}$, then $AB$ would map $\textbf{y}$ to zero, which can only be the zero vector. Thus $\textbf{y} = \textbf{0}$, meaning $A$ and $B$ have full rank, thus both are invertible. This will be used in the following steps:
    
    \alignbreak
    \begin{align}
        &\norm{X - A}_p < \frac{1}{\norm{A^{-1}}_p} &\text{(Given.)}\nonumber\\
        &\norm{X - A}_p\norm{A^{-1}}_p < 1  &\text{(Rearranging.)}\nonumber\\
        &\norm{(X - A)A^{-1}}_p \leq \norm{X - A}_p\norm{A^{-1}}_p &\text{(Norm submultiplicative.)}\label{p2b:normconsistency}\\
        \implies &\norm{XA^{-1} - AA^{-1}}_p < 1 &\text{(Applying \ref{p2b:normconsistency}.)}\nonumber\\
        &\norm{XA^{-1} - \id}_p < 1 &\text{(Simplifying.)}\nonumber\\
        \implies &\id - XA^{-1} + \id \text{ is invertible.} &\text{(By part a.)}\nonumber\\
        \implies &XA^{-1} \text{ is invertible.} &\text{(Simplifying.)}\nonumber\\
        \implies &X \text{ is invertible.} &\text{(By proof above.)}\nonumber
    \end{align}
    \alignbreak
    
    The steps above thus prove $X$ is invertible. $\square$
\end{solution}

\subsection{Problem 2, part c}
Let $\norm{\cdot} : \C^{n\times n} \rightarrow [0, \infty)$ be an arbitrary norm that may not be submultiplicative. Suppose $\norm{A} < 1$, can we conclude that $\id - A$ is invertible?
\partbreak

\begin{solution}
    As a counterexample, take A to be
    \[
    A = 
    \begin{bmatrix}
    1/2     &-1/2   \\
    -1/2    &1/2
    \end{bmatrix}
    \]

    Under the H\"{o}lder $\infty$-norm. Thus $\norm{A} < 1$, but 
    \[
    \id - A = 
    \begin{bmatrix}
    1/2     &1/2   \\
    1/2    &1/2
    \end{bmatrix}
    \]

    which is clearly not invertible, since $\id - A$ does not have full rank.
\end{solution}

\newpage
\section{Problem 3}
Recall that in the lectures, we mentioned that
\begin{enumerate}[(i)]
    \item there are matrix norms that are not submutiplicative and an example is the H\"{o}lder $\infty$-norm;
    \item we may always construct a norm that approximates the spectral radius of a given matrix $A$ as closely as we want.  
\end{enumerate}
\partbreak
\subsection{Problem 3, part a}
Let $\norm{\cdot} : \C^{m\times n} \rightarrow \R$ be a norm, defined for all $m, n \in \N$. show that there always exist a $c > 0$ such that the constant multiple $\norm{\cdot}_c := c\norm{\cdot}$ defines a submultiplicative norm, i.e.,
\[
\norm{AB}_c \leq \norm{A}_c\norm{B}_c
\]
for any $A \in \C^{m\times n}$ and $B \in \C^{n\times p}$ (even if $\norm{\cdot}$ does not have this property). Note that the constant $c$ will depend on $m, n, p \in \N$ in general. Find the constant $c$ for the H\"{o}lder $\infty$-norm. 
\end{document}